\documentclass[12pt,letterpaper]{article}
\usepackage[utf8]{inputenc}
\usepackage{amsmath}
\usepackage{amsfonts}
\usepackage{amssymb}
\usepackage{graphicx}
\usepackage[left=1.00in, right=1.00in, top=1.00in, bottom=1.00in]{geometry}
\usepackage{fancyhdr}
\pagestyle{fancy}
\fancyhf{}
\title{Math 534 Homework 1}
\author{Morgan Gribbins}
\lhead{Page \thepage\ of 2}
\begin{document}
	
\maketitle

\section{Prove that \((a +_{n} b) +_{n} c = a +_{n} (b +_{n} c)\) for \(a,b,c \in \{0,1,...,n-1\}\), thus finishing our proof from lecture that in the "numbers" definition of \(\mathbb{Z}/n\), addition is associative.}

	Note that \([a] +_{n} [b] = [a + b]\). Therefore, \(([a] +_{n} [b]) +_{n} [c] = ([a + b]) +_{n} [c] = [a + b] +_{n} [c] = [a + b + c].\) Additionally, \([a] +_{n} ([b] +_{n} [c]) = [a] +_{n} ([b + c] = [a] +_{n} [b + c] = [a + b + c]\), so both sides are identical.
	
\section{Show the following:}

\subsection{For \(a,a',b,b' \in \mathbb{Z}\), if \(a \cong a'\ (mod\ n)\) and \(b \cong b'\ (mod\ n)\), then \(ab \cong a'b'\ (mod\ n)\).}

	By the division algorithm, we can set \(a = nq_{1} + r_{a}\), \(a' = nq_{2} + r_{a}\), \(b = nq_{3} + r_{b}\), and \(b' = nq_{4} + r_{b}\). Therefore, \[ab = (nq_{1} + r_{a})(nq_{3} + r_{b}) = n^{2}q_{1}q_{3} + nr_{a}q_{3} + nr_{b}q_{1} + r_{a}r_{b}\] \[a'b' = (nq_{2} + r_{a})(nq_{4} + r_{b}) = n^{2}q_{2}q_{4} + nr_{a}q_{4} + nr_{b}q_{2} + r_{a}r_{b},\] which are congruent mod n as \(r_{a}r_{b} \cong r_{a}r_{b}\ (mod\ n)\) trivially.
	
\subsection{For \(a,b,c \in \{1,...,n-1\},\) show \(a \cdot_{n} (b \cdot_{n} c) = (a \cdot_{n} b) \cdot_{n} c. \)}

	By defining the binary operation \(\cdot_{n}([a],[b]) = [a\cdot b]\), we get \([a]\cdot_{n} ([b] \cdot_{n} [c])  = [a] \cdot_{n} ([b \cdot c]) = [a]\cdot_{n} [b \cdot c] = [a \cdot b \cdot c]\) and \(([a]\cdot_{n} [b]) \cdot_{n} [c]  = ([a\cdot b]) \cdot_{n} c = [a \cdot b] \cdot_{n} c = [a \cdot b \cdot c]\), so the \(\cdot_{n}\) operation is associative.
	
\subsection{For \(a,c \in \mathbb{Z}\), if \(gcd(a,n) = 1\) and \(gcd(c,n) = 1\), then \(gcd(ac,n) = 1\).}

	If \(gcd(a,n) = 1\), then a and n are relatively prime, and if \(gcd(c,n) = 1\), then c and n are relatively prime. The assertion \(gcd(ac,n) = 1\) states that ac and n are relatively prime, which directly follows from the prior statements, by the fundamental theorem of arithmetic.
	
\subsection{For \(a,c \in \mathbb{Z}\), if \(gcd(a,n) = 1\) and \(a \cong c\ (mod\ n)\), then \(gcd(c,n) = 1\).}

	The proposition that \(a \cong c\ (mod n)\) implies that a and c vary by an integer multiple of n, i.e. \(a = kn + c\) for some integer k. Now, we have \(1 = xa + yn\) for integers x,y by hypothesis. Substituting \(a = kn + c\) gives us \(1 = x(kn + c) + yn = xc + (kx+y)n = 1\), so \(gcd(c,n) = 1\) because 1 is the smallest positive integer which can be written as a linear combination of c and n.
	
\section{Let \((G, \cdot)\) be a group. For \(g \in G\) and \(k \in \mathbb{N}\), define \(g^{k}\) as the result of combining g with itself k times using the binary operation of the group. Prove that \((g \cdot h)^{2} = g^{2} \cdot h^{2}\) if and only if \(g \cdot h = h \cdot g\).}

	Note that \((g\cdot h)^{2} = (g \cdot h) \cdot (g \cdot h) = g \cdot h \cdot g \cdot h\), by the associativity of a group. \\
	
	Direct proof of \((\implies)\): \\

	Assume \((g \cdot h)^{2} = g^{2} \cdot h^{2}\). This implies that \(g\cdot h \cdot g \cdot h = g \cdot g \cdot h \cdot h\). By applying \(g^{-1}\) on the left hand side of the equation and applying \(h^{-1}\) on the right hand side of the equation (by cancellation laws), we get \[g^{-1} \cdot g\cdot h \cdot g \cdot h \cdot h^{-1} = g^{-1} \cdot g \cdot g \cdot h \cdot h \cdot h^{-1} = h \cdot g = g \cdot h.\] Therefore, \((g\cdot h)^{2} = g^{2} \cdot h^{2} \implies g \cdot h = h \cdot g. \) \\
	
	Direct proof of \((\impliedby)\): \\
	
	Assume \(g \cdot h = h\cdot g\). Multiplying on the left of this equation by \(g\) and on the right by \(h\) gives \(g \cdot g \cdot h \cdot h = g \cdot h \cdot g \cdot h\). The left side of this is equivalent to \(g^{2} \cdot h^{2}\), and the right side of this equation is equivalent to \((g \cdot h)^{2}\), which completes our proof via direct implication.
	
\section{Let \((G, \cdot)\) be a group.}

\subsection{Show that if \(h \in G\) satisfies \(h \cdot g = g\) for some \(g \in G\), then \(h\) is the identity.}

	Assume that \(h \cdot g = g\). Via cancellation laws, apply \(g^{-1}\) to the right of this equation to receive \(h \cdot g \cdot g^{-1} = g \cdot g^{-1} = h \cdot e = e\), so \(h\) is the identity.
	
\subsection{Fix \(k \in G\). Show that if \(h \cdot k = e\), then \(h = k^{-1}\).}

	Assume that \(h \cdot k = e\). Applying \(k^{-1}\) on both right sides gives \(h \cdot k \cdot k^{-1} = e \cdot k^{-1} = h \cdot e = e \cdot k^{-1} = h = k^{-1}\), which is our conclusion.
\end{document}