\documentclass[12pt,letterpaper]{article}
\usepackage[utf8]{inputenc}
\usepackage{amsmath}
\usepackage{amsfonts}
\usepackage{amssymb}
\usepackage{graphicx}
\usepackage[left=1.00in, right=1.00in, top=1.00in, bottom=1.00in]{geometry}
\usepackage{fancyhdr}
\pagestyle{fancy}
\fancyhf{}
\title{Math 534 Homework 2}
\author{Morgan Gribbins}
\lhead{Page \thepage\ of 3}
\begin{document}
	
\maketitle

\section*{(1)}
{\large \textbf{Write down the group table for \((\mathbb{Z}/4,+_{4})\) and \(((\mathbb{Z}/5)^{\times}, \times_{5})\). Are they related? If so, explain how.}
}

\begin{center}
	\begin{tabular}{|c|c|c|c|c|}
		\hline
		\(\mathbb{Z}/4\) & \textbf{0} & \textbf{1} & \textbf{2} & \textbf{3} \\
		\hline
		\textbf{0} & 0 & 1 & 2 & 3 \\
		\hline
		\textbf{1} & 1 & 2 & 3 & 0 \\
		\hline
		\textbf{2} & 2 & 3 & 0 & 1 \\
		\hline
		\textbf{3} & 3 & 0 & 1 & 2 \\
		\hline		
	\end{tabular}
\end{center}
\begin{center}
	\begin{tabular}{|c|c|c|c|c|}
		\hline
		\((\mathbb{Z}/5)^{\times}\) & \textbf{1} & \textbf{2} & \textbf{3} & \textbf{4} \\
		\hline
		\textbf{1} & 1 & 2 & 3 & 4 \\
		\hline
		\textbf{2} & 2 & 4 & 1 & 3 \\
		\hline
		\textbf{3} & 3 & 1 & 4 & 2 \\
		\hline
		\textbf{4} & 4 & 3 & 2 & 1 \\
		\hline
	\end{tabular}
\end{center}
These groups are isometric under the mapping \(\psi : \mathbb{Z}/4 \rightarrow (\mathbb{Z}/5)^{\times}\), with \(\psi(0) = 1, \psi(1) = 4, \psi(2) = 2, \psi(3) = 3\).
\section*{(2)}
{\large \textbf{For each of the following groups \(G\), find \(|G|\) and \(|g|\) for every \(g \in G\):}}
\subsection*{(a) \(G = \mathbb{Z}/12\)}
\(|0| = 1\), as \(0 = e\).\\
\(|1| = 12\), as \(1 + 1 + 1 + 1 + 1 +1 +1 +1 +1 +1 +1 +1 = 12 = 0 = e\).\\
\(|2| = 6\), as \(2 + 2 +2 +2+2+2 =  12 =0 = e\).\\
\(|3| = 4\), as \(3 + 3 +3 +3 = 12 = 0 =e\).\\
\(|4| = 3\), as \(4 + 4 + 4 = 12= 0 = e\). \\
\(|5| = 12\), as \(5 + 5 + 5 + 5 + 5 + 5 + 5 +5 + 5 +5 +5 +5 =60= 0 = e\). \\
\(|6| = 2\), as \(6 + 6 =12= 0 = e\). \\
\(|7| = 12\), as \(7+7+7+7+7+7+7+7+7+7+7+7 =84=  0 = e\).\\
\(|8| = 3\), as \(8+8+8 =24= 0 = e\).\\
\(|9| = 4\), as \(9 + 9 + 9 + 9 = 36 =0 = e\). \\
\(|10| = 6\), as \(10 + 10 + 10 + 10 + 10 + 10 = 60 = 0 = e\). \\
\(|11| = 12\), as \(11 + 11 + 11 + 11 + 11 + 11 + 11 + 11 + 11 + 11 + 11 + 11 = 132 = 0 = e\).
\subsection*{(b) \(G = (\mathbb{Z}/12)^{\times}\)}
\(|1| = 1\), as \(1 = e\). \\
\(|5| = 2\), as \(5 \times 5 = 24 = 1 = e\). \\
\(|7| = 2\), as \(7 \times 7 = 49 = 1 = e\). \\
\(|11| = 2\), as \(11 \times 11 = 121 = 1 = e\).
\subsection*{(c) \(G = (\mathbb{Z}/16)^{\times}\)}
\(|1| = 1\), as \(1 = e\). \\
\(|3| = 4\), as \(3 \times 3 \times 3 \times 3 = 81 = 1 = e\). \\
\(|5| = 4\), as \( 5\times 5 \times 5 \times 5 = 625 = 1 = e\). \\
\(|7| = 2\), as \(7 \times 7 = 49 =1 = e\). \\
\(|9| = 2\), as \(9 \times 9 = 81 = 1 = e\). \\
\(|11| = 4\), as \(11 \times 11 \times 11 \ times 11 = 14641 = 1 = e\). \\
\(|13| = 4\), as \(13 \times 13 \times 13 \times 13 = 28561 = 1 = e\). \\
\(|15| = 2\), as \(15 \times 15 = 225 = 1 = e\).
\subsection*{(d) \(G = \) the symmetries of the square}
\(|R_{0}| = 0\), as \(R_{0} = e\). \\
\(|R_{90}| = 4\), as \(R_{90}^{4} = R_{0} = e\). \\
\(|R_{180}| = 2\), as \(R_{180}^{2} = R_{0} = e\). \\
\(|R_{270}| = 4\), as \(R_{270}^{4} = R_{0} = e\). \\
\(|H| = 2\), as \(H^{2} = e\). \\
\(|V| = 2\), as \(V^{2} = e\). \\
\(|D| = 2\), as \(D^{2} = e\). \\
\(|D'| = 2\), as \(D'^{2} = e\). \\
\section*{(3)}
{\large \textbf{Recall from lecture that Wilson's Theorem states that a number \(n\) is prime if and only if \((n-1)! \cong -1\ (mod\ n)\). We proved that \(n\) being prime implies the above congruence. In this problem, we will complete the proof by showing that if \(n\) isn't prime, then \((n-1)! \ncong -1\ (mod\ n)\).}}

\subsection*{(a) Complete and then prove the following statement: if \(n\) is not prime and \(n \neq a\), then \((n-1)! \cong 0\ (mod\ n)\).}

\(a = 1\). Let \(n\) be non-prime. This implies that \(n\) can be expressed as a product of some finite amount of integers less than \(n\). As \((n-1)!\) is the product of all integers less than \(n\), it must be a multiple of \(n\), and as such, \((n-1)! \cong 0\ (mod\ n)\).

\subsection*{(b) Prove the only thing left to complete our proof of Wilson's Theorem.}

We must now prove that \(0 \ncong -1\ (mod\ n)\), for all \(n \neq 1\). The assertion that \(0 \cong -1\ (mod\ n)\) means that \(n | 0 - (-1)\) i.e. \(n | 1\). This implies that there is some \(k \in \mathbb{Z}\) that satisfies the equation \(kn = 1\), which cannot be true for natural \(n \neq 1\). Therefore, we have \(n\) prime \(\implies (n-1)! \cong -1\ (mod\ n)\). \\

Also, when \(n = 1\), \((n-1)! \cong 0 \cong -1\ (mod\ n)\). \(1\) is not prime.

\section*{(4)}
{\large \textbf{Let \(G\) be a group. The center of \(G\) is defined via: \[Z(G) = \{g\in G: gx = xg\ for\ all\ x\in G\}.\] Prove the following: if \(a \in G\) is the only element in \(G\) of order 2, then \(a\in Z(G)\).}}

Assume that \(a \in G\) is the only element in \(G\) of order 2. This implies that \(a^{2} = e\), and \(\forall b \in G,\ b \neq a\ and\ b \neq e \implies b^{2} \neq e.\). The assertion that \(a \in Z(G)\) means that \(ga = ag, g \in G\), which is true for \(g = e\) or \(g = a\) because \(ea = a = ae\) and \(aa = e = aa\). We must now show that this holds for all other cases. Let \(g \in G\) be some arbitrary element of \(g\), and let us assume \(g\) has an order \(k > 2\). Multiplying on the left of the equation \(a = a^{-1}\) by \(g^{k-1}\) gives us \(g^{k-1}a = g^{k-1}a^{-1} \). Inverting this, we get \(a^{-1}g = ag \implies ga = ag\), so \(a \in Z(G)\).

\end{document}