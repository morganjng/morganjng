\documentclass[12pt,letterpaper]{article}
\usepackage[utf8]{inputenc}
\usepackage{amsmath}
\usepackage{amsfonts}
\usepackage{amssymb}
\usepackage{graphicx}
\usepackage{lastpage}
\usepackage[left=1.00in, right=1.00in, top=1.00in, bottom=1.00in]{geometry}
\usepackage{fancyhdr}
\title{Math 534 HW 6}
\author{Morgan Gribbins}
\date{}
\pagestyle{fancy}
\fancyhf{}
\lhead{Page \thepage\ of \pageref{LastPage}}
\begin{document}
	
\maketitle

\textbf{(1)} Prove that every element in \(S_{n}\) for \(n > 1\) can be written as a product of transpositions of the form \((1\ k)\). \\

Proof by induction that \((i\ i+1)\) can be written as a product of transpositions of the form \((1\ k)\). First, we will prove the case of \(i = 2\). This is the transposition \((2\ 3)\), which is equivalent to \((12)(13)(12)\), so this case may be written as a product of transpositions of the form \((1\ k)\). \\

 Now, we will assume that \((j\ j+1)\) can be written in this form, and show that this implies \((j+1\ j+2)\) can be written in this form. The permutation \((1\ j+1)(1\ j+2)(1\ j+1)(j\ j+1) = (j+1\ j+2),\) so it is evident that this property holding for \(j\) implies it holds for \(j+1\), so this is possible for every element in \(S_{n}\). \\ 

\textbf{(2)} Prove that the 4-cycle \((1234) \in S_{n}\) cannot be written as a product of 3-cycles. \\

The 4-cycle \((1234)\) is equivalent to \((14)(13)(12)\) and as such is odd. Any 3-cycle \((abc)\) can be written as \((ac)(ab)\), so it is even. Therefore, any product of 3-cycles must be even. This means that any product of 3-cycles has a different parity than \((1234)\), so these permutations can not be equal. \\ 

\textbf{(3)} Show that a permutation of odd order must be an even permutation. Is every permutation of even order an odd permutation? \\

Consider an arbitrary permutation of order \(2n+1\), which can be written as  \((a_{1}a_{2}...a_{2n+1}\). This permutation can be written as \((a_{2n+1}a_{1})(a_{2n}a_{1})...(a_{2}a_{1})\), which has \(2n\) terms, and as such is even. Therefore, an odd ordered permutation must be even. \\

This is not true. Consider the permutation \((34)(12)\). This has even order (2) and is an odd permutation. \\

\textbf{(4)} Prove that the groups \((\mathbb{Z}, +)\) and \((\mathbb{Q}, \cdot)\) are not isomorphic. \\

Assume that there is some isomorphism between these two groups, \(\phi : (\mathbb{Z},+) \to (\mathbb{Q},\cdot)\).  As this is a isomorphism, we have \(\phi(a+b) = \phi(a) \cdot \phi(b)\). Therefore, we have \(\phi(2) = \phi(1+1) = \phi(1)\cdot\phi(1)\) and \(\phi(3) = \phi(1+1+1) = \phi(1)\cdot\phi(1)\cdot\phi(1)\). This means there is some element \(a\) in \(\mathbb{Q}\) such that \(a^{2} = a^{3}\) (with regular rules of multiplication, and with 0 not in this group), so \(a=1\). However, we have \(\phi(0) = 1\), so this contradicts the 1-1 property of an isomorphism. Therefore, this must not be an isomorphism, and these two groups cannot be isomorphic. \\

\textbf{(5)} Let \(\phi : G_{1} \to G_{2}\) be a homomorphism. \\

\textbf{(5a)} Define the \textit{image} of \(G_{1}\) under \(\phi\) to be \[\phi(G_{1}) = \{g \in G_{2} : \exists a \in G_{2},\ g = \phi(a)\}\] Prove that this is a subgroup of \(G_{2}\). \\

In order for \(\phi(G_{1})\) to be a subgroup of \(G_{2}\), it must be closed, it must contain the identity, and it must contain inverses. \\

Proof that \(\phi(G_{1})\) is closed. Let \(g_{1}, g_{2} \in \phi(G_{1})\). Due to the definition of the image of \(G_{1}\), we have \(\exists a_{1},a_{2}\) such that \(\phi(a_{1}) = g_{1}\) and \(\phi(a_{2}) = g_{2}\). By definition of a homomorphism, we have \(\phi(a_{1}a_{2}) = \phi(a_{1})\phi(a_{2}) = g_{1}g_{2}\), so \(a_{1}a_{2} \in G_{1}\) satisfies \(\phi(a_{1}a_{2}) = g_{1}g_{2}\), so \(g_{1},g_{2} \in \phi(G_{1})  \implies g_{1}g_{2} \in \phi(G_{1})\). \\

Proof that \(\phi(G_{1})\) contains the identity. Let \(e\) be the identity in \(G_{1}\) and \(i\) be the identity in \(G_{2}\). As \(e \in G_{1}\), \(\phi(e) = \phi(ee) = \phi(e)\phi(e) \in G_{2}\) (and \(\phi(e) \in \phi(G_{1})\). Therefore, we have \(\phi(e) = \phi(e)\phi(e) \implies i\phi(e) = \phi(e)\phi(e) \implies i = \phi(e)\), and \(\phi(e) \in \phi(G_{1})\), so \(\phi(G_{1})\) contains the identity. \\

Proof that \(\phi(G_{1})\) contains inverses. Let \(a \in G_{1}\). This implies that \(a^{-1} \in G_{1}\), so we have \(\phi(a)\) and \(\phi(a^{-1}\) in \(\phi(G_{1})\). Also, \(\phi(e) = \phi(aa^{-1}) = \phi(a)\phi(a^{-1}) \implies \phi(e) = \phi(a)\phi(a^{1})\) where \(\phi(e)\) is the identity of \(\phi(G_{1})\), so \(\phi(a) \in \phi(G_{1})\implies \phi(a^{-1}) = \phi(a)^{-1} \in \phi(G_{1})\), so this group has inverses. \\

Therefore, \(\phi(G_{1})\) is a subgroup of \(G_{2}\). \\

\textbf{(5b)} Define the \textit{kernel} of \(\phi\) to be \[\ker \phi = \{a \in G_{1} : \phi(a) = e\}\] where \(e\in G_{2}\) is the identity. Prove that this is a subgroup of \(G_{1}\). \\

In order for \(\ker \phi\) to be  a subgroup of \(G_{1}\), it must be closed, it must contain the identity, and it must contain inverses. \\

Proof that \(\ker \phi\) is closed. Let \(a,b \in \ker\phi\). By definition of \(\ker\phi\), we have \(\phi(a) = e\) and \(\phi(b) = e\). Consider the element \(ab \in G_{1}\). \(\phi(ab) = \phi(a)\phi(b) = ee = e\), so \(ab \in \ker\phi\). Therefore, \(\ker\phi\) is closed. \\

Proof that \(\ker \phi \) contains the identity. Let \(e_{1}\) be the identity of \(G_{1}\) and \(e_{2}\) be the identity of \(G_{2}\). \(e_{1} \in \ker\phi\) if \(\phi(e_{1}) = e_{2}\). We have \(\phi(e_{1}) = \phi(e_{1}e_{1} =  \phi(e_{1})\phi(e_{1}) \implies e_{2}\phi(e_{1}) = \phi(e_{1})\phi(e_{1}) \implies e_{2} = \phi(e_{2})\), so the identity (of \(G_{1}\)) is in \(\ker\phi)\). \\

Proof that \(\ker\phi\) contains inverses. Let \(a \in \ker\phi\). This means that \(\phi(a) = e\) and \(\phi(e) = \phi(aa^{-1}) = \phi(a)\phi(a^{-1}) = e \implies e\phi(a^{-1}) = e \implies \phi(a^{-1}) = e\), (the e-s in this equation being the identity of \(G_{2}\)) so \(a \in \ker\phi \implies a^{-1} \in \ker\phi\). \\

Therefore, \(\ker\phi\) is a subgroup of \(G_{1}\). \\

\textbf{(6)} Is the function \(\psi: \mathbb{Z}/12 \to \mathbb{Z}/10\) defined by \(\psi(k) = 3k\) a homomophism? If so, prove it, and if not, explain why not. \\

This is a homomorphism. Let \(a,b \in \mathbb{Z}/12\). Let us examine \(\phi(a+b)\). This is equal to \(3(a+b) \mod 10 = 3a+3b \mod 10\). Considering \(\phi(a) + \phi(b) = 3a + 3b \mod 10\), it is clear that these quantities are equal. Therefore, this is a homomorphism.

\end{document}