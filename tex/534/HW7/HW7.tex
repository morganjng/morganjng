\documentclass[12pt,letterpaper]{article}
\usepackage[utf8]{inputenc}
\usepackage{amsmath}
\usepackage{amsfonts}
\usepackage{amssymb}
\usepackage{graphicx}
\usepackage{lastpage}
\usepackage[left=1.00in, right=1.00in, top=1.00in, bottom=1.00in]{geometry}
\usepackage{fancyhdr}
\title{Math 534 HW 7}
\author{Morgan Gribbins}
\date{}
\pagestyle{fancy}
\fancyhf{}
\lhead{Page \thepage\ of \pageref{LastPage}}
\begin{document}
	
\maketitle

\textbf{(1)} An isomorphism from a group to itself, i.e. an isomorphism \(\alpha: G \to G\), is called an automorphism. Suppose that \(G\) is a finite abelian group which has no elements of order 2. Show that the function \(\alpha(g) = g^{2}\) is an automorphism of \(G\). Show by example that the result doesn't hold in the case that \(G\) is infinite. \\

For \(\alpha\) to be an automorphism, it must be an isomorphism from \(G\) to itself, and as \(g \in G \implies g^{2} = \alpha(g) \in G\), this function is \(G \to G\). For \(\alpha\) to be an isomorphism, it must both be a bijection and satisfy \[\alpha(gh) = \alpha(g)\alpha(h),\ \text{for all g, h in G}.\] 

\begin{itemize}
	\item Proof that \(\alpha\) is bijective. Note that, because \(G\) is finite, injectivity of \(\alpha\) is equivalent to bijectivity (and surjectivity); also note that \(G\) is abelian. Let \(g, h \in G\) with \(\alpha(g) = \alpha(h) \implies g^{2} = h^{2}\). This directly implies that \(g = h\), so this function is injective, which implies that it is bijective. \\
	\item Proof that \(\alpha\) is an homomorphism. Let \(g,h \in G\). Then, \(\alpha(gh) = ghgh = gghh = g^{2} h^{2} = \alpha(g)\alpha(h)\), so this is a homomorphism. \\
\end{itemize}

Therefore, \(\alpha\) is an automorphism on \(G\). \\

An example that the result doesn't hold in the case that \(G\) is infinite is provided by the group \((\mathbb{Z}, +)\). There is no element \(b \in \mathbb{Z}\) that satisfies \(b + b = 3\), so this cannot be an automorphism (however it is abelian without any elements of order 2). \\ 

\textbf{(2)} Consider the group \(G = (\mathbb{Z}, +)\). \\

\textbf{(2a)} Show that \(G\) is isomorphic to the proper subgroup \(H = \{2k : k \in \mathbb{Z}\}\) of even elements. \\

Let \(\phi : G \to H\) be defined by multiplying elements in \(G\) by 2. \\

Proof that \(\phi\) is injective. Let \(g, h \in G\) such that \(\phi(g) = \phi(h) \implies 2g = 2h \implies g = h.\) \\

Proof that \(\phi\) is surjective. Let \(g \in H\). This implies that \(g = 2k\), for some \(k \in \mathbb{Z}\), which means that \(k \in G\), and \(g = 2k = \phi(k)\). \\

Proof that \(\phi\) is a homomorphism. Let \(g,h \in G\). Then, \(\phi(g + h) = 2(g + h) = 2g + 2h = \phi(g) + \phi(h)\), so \(\phi\) is a homomorphism. \\

As \(\phi\) is a bijective homomorphism, it is an isomorphism, and these groups are isomorphic.

\textbf{(2b)} Show that there are in fact infinitely many subgroups of \(G\) to which it is isomorphic. \\

Let \(a\neq 0 \in \mathbb{Z}\), and let \(H_{a} = \{ak : k \in \mathbb{Z}\} \leq G\). As \(\mathbb{Z}\setminus \{0\}\) is infinite, there are an infinite amount of \(H_{a}\). We will now show that \(G\) is isomorphic to any \(H_{a}\). \\

Let \(\phi : G \to H_{a}\) be defined by multiplying elements in \(G\) by \(a\). \\

Proof that \(\phi\) is injective. Let \(g, h \in G\) such that \(\phi(g) = \phi(h) \implies ag = ah \implies g = h.\) \\

Proof that \(\phi\) is surjective. Let \(g \in H\). This implies that \(g = ak\), for some \(k \in \mathbb{Z}\), which means that \(k \in G\), and \(g = ak = \phi(k)\). \\

Proof that \(\phi\) is a homomorphism. Let \(g,h \in G\). Then, \(\phi(g + h) = a(g + h) = ag + ah = \phi(g) + \phi(h)\), so \(\phi\) is a homomorphism. \\

As \(\phi\) is a bijective homomorphism, it is an isomorphism, and \(G\) is isomorphic to an arbitrary \(H_{a}\). As there are an infinite amount of \(H_{a}\), \(G\) has an infinite amount of subgroups that are isomorphic to \(G\). \\

\textbf{(3)} Consider the group \(\mathbb{R}^{\times} = (\mathbb{R}\setminus \{0\}, \times)\) of non-zero real numbers under multiplication. Prove that this group is not isomorphic to \((\mathbb{R}, +)\). \\

Let \(\phi : \mathbb{R}\setminus \{0\} \to \mathbb{R}\) be an isomorphism between these two sets. Consider \(\phi(1) = 0 = \phi(-1 \times -1) = \phi(-1) + \phi(-1) \implies \phi(-1) = \phi(1) \text{ and } -1 \neq 1\), which means that \(\phi\) is not bijective, and as such cannot be an isomorphism. \\

\textbf{(4)} Explain how \(\mathfrak{S}_{8}\) contains subgroups isomorphic to \(\mathbb{Z}/15, (\mathbb{Z}/16)^{\times}\), and \(D_{8}\). Here, \(D_{8}\) denotes the group of symmetries of a regular convex octagon. \\ 

\(\mathfrak{S}_{8}\)  contains subgroups isomorphic to:

\begin{itemize}
	\item \(\mathbb{Z}/15\) because all elements in \(\mathbb{Z}/15\) that are even are equal to their quotient by \(2\) squared, and so permutations on the odd elements of \(\mathbb{Z}/15\) permute their corresponding even elements. 
	\item \((\mathbb{Z}/16)^{\times}\) because this group has order \(8\), and so this isomorphism exists by Cayley's theorem.
	\item \(D_{8}\) has symmetry between the ``front" and ``back" of the octagon, each with eight different positions, so it is ``similar" to an object with eight possible settings.
\end{itemize}




\end{document}