\documentclass[12pt,letterpaper]{article}
\usepackage[utf8]{inputenc}
\usepackage{amsmath}
\usepackage{amsfonts}
\usepackage{amssymb}
\usepackage{graphicx}
\usepackage{lastpage}
\usepackage[left=1.00in, right=1.00in, top=1.00in, bottom=1.00in]{geometry}
\usepackage{fancyhdr}
\title{Math 534 HW 7}
\author{Morgan Gribbins}
\date{}
\pagestyle{fancy}
\fancyhf{}
\lhead{Page \thepage\ of \pageref{LastPage}}
\begin{document}
	
\maketitle

\textbf{(1)} An isomorphism from a group to itself, i.e. an isomorphism \(\alpha: G \to G\), is called an automorphism. Suppose that \(G\) is a finite abelian group which has no elements of order 2. Show that the function \(\alpha(g) = g^{2}\) is an automorphism of \(G\). Show by example that the result doesn't hold in the case that \(G\) is infinite. \\



\textbf{(2)} Consider the group \(G = (\mathbb{Z}, +)\). \\

\textbf{(2a)} Show that \(G\) is isomorphic to the proper subgroup \(H = \{2k : k \in \mathbb{Z}\}\) of even elements. \\



\textbf{(2b)} Show that there are in fact infinitely many subgroups of \(G\) to which it is isomorphic. \\



\textbf{(3)} Consider the group \(\mathbb{R}^{\times} = (\mathbb{R}\setminus \{0\}, \times)\) of non-zero real numbers under multiplication. Prove that this group is not isomorphic to \((\mathbb{R}, +)\). \\



\textbf{(4)} Explain \(\mathfrak{S}_{8}\) contains subgroups isomorphic to \(\mathbb{Z}/15, (\mathbb{Z}/16)^{\times}\), and \(D_{8}\). Here, \(D_{8}\) denotes the group of symmetries of a regular convex octagon. \\ 




\end{document}