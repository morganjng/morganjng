\documentclass[12pt,letterpaper]{article}
\usepackage[utf8]{inputenc}
\usepackage{amsmath}
\usepackage{amsfonts}
\usepackage{amssymb}
\usepackage{graphicx}
\usepackage[left=1.00in, right=1.00in, top=1.00in, bottom=1.00in]{geometry}
\usepackage{fancyhdr}
\title{Math 534 HW 3}
\author{Morgan Gribbins}
\date{}
\pagestyle{fancy}
\fancyhf{}
\lhead{Page \thepage\ of 6}
\begin{document}
	
\maketitle

\section*{(1)}
Let \(G\) be a group, and let \(H < G\) and \(K < G\) be two proper subgroups. Show that \(G \neq H \cup K\) (as sets). \\

Let \(a,b \in G\), with the properties that \(a \in H\), \(a \notin K\), \(b \in K\), and \(b \notin H\). If no such two elements exist, then \(\forall a \in H, a \in K \) or \(\forall b \in K, b \in H\), which means that one of these groups is a subgroup of the other, and their union must not be equal to \(G\) as it is equal to either \(H\) or \(K\) (the union of a set and its subset is equal to the set). Because of the existence of these elements in \(G\), \(ab\) must also be in \(G\). However, since only one \(a\) or \(b\) is in either \(H\) or \(K\) (and, in groups, \(xy \neq xz\) for distinct \(y,z\)), \(ab\) cannot be in either of these sets. Therefore, the union of these two sets does not contain \(ab\), so \(G \neq H \cup K\).

\section*{(2)}
Let \(G\) be a group which contains two distinct elements \(a,b \in G\) which commute (i.e. \(ab = ba\)) and satisfy \(|a| = 2 = |b|\). Prove that \(G\) contains a subgroup \(H \leq G\) of order 4. \\

The subgroup \(H = \{e, a, b, ab\}\) has order four and is a subgroup of \(G\). This subgroup is closed (proof by exhaustion):

\begin{itemize}
	\item Closure of \(e\):
	\begin{itemize}
		\item \(ee = e \in H\)
		\item \(ea = a \in H\)
		\item \(eb = b \in H\)
		\item \(e(ab) = ab \in H\)
	\end{itemize}
	\item Closure of \(a\):
	\begin{itemize}
		\item \(ae = a \in H\)
		\item \(aa = e \in H\)
		\item \(ab = ab \in H\)
		\item \(a(ab) = (aa)b = b \in H\)
	\end{itemize}
	\item Closure of \(b\):
	\begin{itemize}
		\item \(be = b \in H\)
		\item \(ba = ab \in H\)
		\item \(bb = e \in H\)
		\item \(b(ab) = (bb)a = a \in H\)
	\end{itemize}
	\item Closure of \(ab\):
	\begin{itemize}
		\item \((ab)e = ab \in H\)
		\item \((ab)a = b(aa) = b \in H\)
		\item \((ab)b = a(bb) = a \in H\)
		\item \((ab)(ab) = (aa)(bb) = e \in H\)
	\end{itemize}
\end{itemize}

It also contains the identity \(e\), and each element has an inverse (\((e)^{-1} = e,\ (a)^{-1} = a,\ (b)^{-1} = b, (ab)^{-1} = ab\)), so it is certainly a subgroup of \(G\) of order \(4\).

\section*{(3)}
Denote the set of \(2 \times 2\) matrices with entries in \(\mathbb{R}\) by \(M_{2}(\mathbb{R})\). Define matrix multiplication as \[\begin{Bmatrix} a & b \\ c & d \end{Bmatrix} \begin{Bmatrix} w & x \\ y & z \end{Bmatrix} = \begin{Bmatrix} aw + by & ax + bz \\ cw + dy & cx + dz \end{Bmatrix}.\]

\subsection*{(a)}
Let \(GL_{2}(\mathbb{R}) = \left\{ \begin{Bmatrix} a & b \\ c & d \end{Bmatrix} \in M_{2}(\mathbb{R}) : ad - bc \neq 0 \right\}\). Prove that \(GL_{2}(\mathbb{R})\) is a group. \\

Proof that \(GL_{2}(\mathbb{R})\) is closed: \\

For two arbitrary elements of \(GL_{2}(\mathbb{R})\), their multiplication is defined by \[\begin{Bmatrix} a & b \\ c & d \end{Bmatrix} \begin{Bmatrix} w & x \\ y & z \end{Bmatrix} = \begin{Bmatrix} aw + by & ax + bz \\ cw + dy & cx + dz \end{Bmatrix}.\] For this resultant matrix to be in the group, it must satisfy \[(aw+by)(cx+dz) - (ax + bz )(cw + dy) \neq 0.\] Additionally, it is provided by definition of an element of this group that \(ad - bc \neq 0\) and \(wz - xy \neq 0 \). We will assume the negation to prove this:  \[(aw+by)(cx+dz) - (ax + bz)(cw+dy) = 0\] \[\implies acwx + adwz + bcxy + bdyz -  acwx - adxy - bcwz - bdyz = 0\] \[\implies adwz + bcxy - adxy - bcwz = 0\] \[\implies ad(wz - xy) - bc(wz - xy) = 0\] \[\implies ad - bc = 0 \text{ or } wz - xy = 0,\] both of which contradict our assumptions, so this equation cannot be true and this group must be closed. 

Proof that \(GL_{2}(\mathbb{R})\) is associative: \\
Let \(A = \begin{Bmatrix} a_{1} & a_{2} \\ a_{3} & a_{4} \end{Bmatrix}\), \(B = \begin{Bmatrix} b_{1} & b_{2} \\ b_{3} & b_{4} \end{Bmatrix}\), and \(C = \begin{Bmatrix} c_{1} & c_{2} \\ c_{3} & c_{4} \end{Bmatrix}\). We will now show that \((AB)C = A(BC)\). \[AB = \begin{Bmatrix} a_{1} & a_{2} \\ a_{3} & a_{4} \end{Bmatrix} \begin{Bmatrix} b_{1} & b_{2} \\ b_{3} & b_{4} \end{Bmatrix} = \begin{Bmatrix} a_{1}b_{1} + a_{2}b_{3} & a_{1}b_{2} + a_{2}b_{4} \\ a_{3}b_{1} + a_{4}b_{3} & a_{3}b_{2} + a_{4}b_{4} \end{Bmatrix},\] \[(AB)C = \begin{Bmatrix} a_{1}b_{1} + a_{2}b_{3} & a_{1}b_{2} + a_{2}b_{4} \\ a_{3}b_{1} + a_{4}b_{3} & a_{3}b_{2} + a_{4}b_{4} \end{Bmatrix} \begin{Bmatrix} c_{1} & c_{2} \\ c_{3} & c_{4} \end{Bmatrix} \] \[ = \begin{Bmatrix} a_{1}b_{1}c_{1} + a_{2}b_{3}c_{1} + a_{1}b_{2}c_{3} + a_{2}b_{4}c_{3} & a_{1}b_{1}c_{2} + a_{2}b_{3}c_{2} + a_{1}b_{1}c_{4} + a_{2}b_{4}c_{4} \\  a_{3}b_{1}c_{1} + a_{4}b_{3}c_{1} + a_{3}b_{2}c_{3} + a_{4}b_{4}c_{3} & a_{3}b_{1}c_{2} + a_{4}b_{3}c_{2} + a_{3}b_{2}c_{4} + a_{4}b_{4}c_{4} \end{Bmatrix}.\] \[BC = \begin{Bmatrix} b_{1} & b_{2} \\ b_{3} & b_{4} \end{Bmatrix} \begin{Bmatrix} c_{1} & c_{2} \\ c_{3} & c_{4} \end{Bmatrix} = \begin{Bmatrix} b_{1}c_{1} + b_{2}c_{3} & b_{1}c_{2} + b_{2}c_{4} \\ b_{3}c_{1} + b_{4}c_{3} & b_{3}c_{2} + b_{4}c_{4} \end{Bmatrix},\] \[A(BC) = \begin{Bmatrix} a_{1} & a_{2} \\ a_{3} & a_{4} \end{Bmatrix} \begin{Bmatrix} b_{1}c_{1} + b_{2}c_{3} & b_{1}c_{2} + b_{2}c_{4} \\ b_{3}c_{1} + b_{4}c_{3} & b_{3}c_{2} + b_{4}c_{4} \end{Bmatrix}\] \[ = \begin{Bmatrix} a_{1}b_{1}c_{1} + a_{2}b_{3}c_{1} + a_{1}b_{2}c_{3} + a_{2}b_{4}c_{3} & a_{1}b_{1}c_{2} + a_{2}b_{3}c_{2} + a_{1}b_{1}c_{4} + a_{2}b_{4}c_{4} \\  a_{3}b_{1}c_{1} + a_{4}b_{3}c_{1} + a_{3}b_{2}c_{3} + a_{4}b_{4}c_{3} & a_{3}b_{1}c_{2} + a_{4}b_{3}c_{2} + a_{3}b_{2}c_{4} + a_{4}b_{4}c_{4} \end{Bmatrix},\] so \((AB)C = A(BC)\). \\

Proof that there is an identity element in \(GL_{2}(\mathbb{R})\): \\
Let \(A = \begin{Bmatrix} a & b \\ c & d \end{Bmatrix} \in GL_{2}(\mathbb{R})\) and \(I = \begin{Bmatrix} 1 & 0 \\ 0 & 1 \end{Bmatrix} \in GL_{2}(\mathbb{R})\). Then, we have \[AI = \begin{Bmatrix} a & b \\ c & d \end{Bmatrix}\begin{Bmatrix} 1 & 0 \\ 0 & 1 \end{Bmatrix} = \begin{Bmatrix} a & b \\ c & d \end{Bmatrix} = A\] and \[IA = \begin{Bmatrix} 1 & 0 \\ 0 & 1 \end{Bmatrix} \begin{Bmatrix} a & b \\ c & d \end{Bmatrix} = \begin{Bmatrix} a & b \\ c & d \end{Bmatrix} = A,\] so this element of \(GL_{2}(\mathbb{R})\) is the identity. \\

Proof that each \(A \in GL_{2}(\mathbb{R})\) has an inverse: \\
Let \(A = \begin{Bmatrix} a & b \\ c & d \end{Bmatrix}\). The inverse of \(A\) is equal to \(\frac{1}{ad-bc} \begin{Bmatrix} d & -b \\ -c & a \end{Bmatrix}\), as \[AA^{-1} = \frac{1}{ad-bc} \begin{Bmatrix} a & b \\ c & d \end{Bmatrix} \begin{Bmatrix} d & -b \\ -c & a \end{Bmatrix} = \frac{1}{ad-bc} \begin{Bmatrix} ad - bc & 0 \\ 0 & ad-bc \end{Bmatrix} = \begin{Bmatrix} 1 & 0 \\ 0 & 1 \end{Bmatrix} = I \] and \[A^{-1}A = \frac{1}{ad-bc} \begin{Bmatrix} d & -b \\ -c & a \end{Bmatrix} \begin{Bmatrix} a & b \\ c & d \end{Bmatrix} = \frac{1}{ad-bc} \begin{Bmatrix} ad - bc & 0 \\ 0 & ad-bc \end{Bmatrix} = \begin{Bmatrix} 1 & 0 \\ 0 & 1 \end{Bmatrix} = I, \] so each matrix in \(GL_{2}(\mathbb{R})\) has an inverse. \\

Since \(GL_{2}(\mathbb{R})\) satisfies all of the group axioms, it is a group.

\subsection*{(b)}
Is \(GL_{2}(\mathbb{R})\) abelian? If so, prove it, and if not, give an example indicating why it isn't. \\

Let \(A = \begin{Bmatrix} a_{1} & a_{2} \\ a_{3} & a_{4} \end{Bmatrix}\) and \(B = \begin{Bmatrix} b_{1} & b_{2} \\ b_{3} & b_{4} \end{Bmatrix}\). Then, we have \[AB = \begin{Bmatrix} a_{1}b_{1} + a_{2}b_{3} & a_{1}b_{2} + a_{2}b_{4} \\ a_{3}b_{1} + a_{4}b_{3} & a_{3}b_{2} + a_{4}b_{4} \end{Bmatrix}\] and \[BA = \begin{Bmatrix} a_{1}b_{1} + a_{3}b_{2} & a_{2}b_{1} + a_{4}b_{2} \\ a_{1}b_{3} + a_{3}b_{4} & a_{2}b_{3} + a_{4}b_{4} \end{Bmatrix}. \] It is evident that these are not always the same, so this group is not abelian. As an example, let \(A = \begin{Bmatrix}  1 & 1 \\ 0 & 1 \end{Bmatrix}\) and \(B = \begin{Bmatrix} 1 & 0 \\ 1 & 1 \end{Bmatrix}\).\(AB\) is then equal to \(\begin{Bmatrix} 2 & 1 \\ 1 & 1 \end{Bmatrix}\) while \(BA\) is equal to \(\begin{Bmatrix} 1 & 1 \\ 1 & 2 \end{Bmatrix}\).

\subsection*{(c)}
Prove that \(SL_{2} = \left\{ \begin{Bmatrix} a & b \\ c & d \end{Bmatrix} \in M_{2}(\mathbb{R}) : ad - bc = 1 \right\}\) is a subgroup of \(GL_{2}(\mathbb{R})\). \\

All elements in \(SL_{2}\) necessarily must be in \(GL_{2}(\mathbb{R}))\) as \(ad - bc = 1 \implies ad -bc \neq 0\).

Proof that \(SL_{2}\) is closed: \\

Let \(A = \begin{Bmatrix} a & b \\ c & d \end{Bmatrix}\) and \(B = \begin{Bmatrix} w & x \\ y & z \end{Bmatrix}\) Their product is then given by \[\begin{Bmatrix} a & b \\ c & d \end{Bmatrix} \begin{Bmatrix} w & x \\ y & z \end{Bmatrix} = \begin{Bmatrix} aw + by & ax + bz \\ cw + dy & cx + dz \end{Bmatrix}.\] For this resultant matrix to be in the group, it must satisfy \[(aw+by)(cx+dz) - (ax + bz )(cw + dy) = 1.\] Additionally, it is provided by definition of an element of this group that \(ad - bc = 1\) and \(wz - xy = 1\). We will compute out the requisite equation to check if it is satisfied: \[(aw+by)(cx+dz) - (ax + bz)(cw+dy) = 1\] \[\implies acwx + adwz + bcxy + bdyz -  acwx - adxy - bcwz - bdyz = 1\] \[\implies adwz + bcxy - adxy - bcwz = 1\] \[\implies ad(wz - xy) - bc(wz - xy) = 0\] \[\implies (ad-bc)(wx-yz) = 1.\] Because both \(ad - bc\) and \(wx - yz\) are equal to one, their product is also equal to one, and so this equation and the inclusion of \(AB \in SL_{2}\) holds. \\

Proof that \(SL_{2}\) has an identity: \\

\(\begin{Bmatrix} 1 & 0 \\ 0 & 1 \end{Bmatrix}\) is the identity and is in \(SL_{2}\), as \((1)(1) - (0)(0) = 1\). \\

Proof that \(SL_{2}\) has inverses: \\

Let \(A = \begin{Bmatrix} a & b \\ c & d \end{Bmatrix} \in SL_{2}\). \(A^{-1} = \begin{Bmatrix} d & -b \\ -c & a \end{Bmatrix}\). This is also in \(SL_{2}\) as \((da) - (bc) = 1\) holds because \(ad- bc = 1\) due to the inclusion of \(A\) in \(SL_{2}\). \\

Therefore, \(SL_{2}\) is a subgroup of \(GL_{2}(\mathbb{R})\).

\subsection*{(d)}
Compute the orders of \(\begin{Bmatrix} 0 & -1 \\ 1 & 1 \end{Bmatrix}\) and \(\begin{Bmatrix} 1 & -1 \\ 0 & 1 \end{Bmatrix}\). \\

Let \(I\) be the identity matrix.

\[\begin{Bmatrix} 0 & -1 \\ 1 & 1 \end{Bmatrix} \neq I\] \[\begin{Bmatrix} 0 & -1 \\ 1 & 1 \end{Bmatrix} \begin{Bmatrix} 0 & -1 \\ 1 & 1 \end{Bmatrix} = \begin{Bmatrix} -1 & -1 \\ 1 & 0 \end{Bmatrix} \neq I\] \[\begin{Bmatrix} 0 & -1 \\ 1 & 1 \end{Bmatrix} \begin{Bmatrix} 0 & -1 \\ 1 & 1 \end{Bmatrix} \begin{Bmatrix} 0 & -1 \\ 1 & 1 \end{Bmatrix} = \begin{Bmatrix} -1 & 0 \\ 0 & -1 \end{Bmatrix} \neq I\] \[\begin{Bmatrix} 0 & -1 \\ 1 & 1 \end{Bmatrix} \begin{Bmatrix} 0 & -1 \\ 1 & 1 \end{Bmatrix} \begin{Bmatrix} 0 & -1 \\ 1 & 1 \end{Bmatrix} \begin{Bmatrix} 0 & -1 \\ 1 & 1 \end{Bmatrix} = \begin{Bmatrix} 0 & 1 \\ -1 & -1 \end{Bmatrix} \neq I\] \[\begin{Bmatrix} 0 & -1 \\ 1 & 1 \end{Bmatrix} \begin{Bmatrix} 0 & -1 \\ 1 & 1 \end{Bmatrix} \begin{Bmatrix} 0 & -1 \\ 1 & 1 \end{Bmatrix} \begin{Bmatrix} 0 & -1 \\ 1 & 1 \end{Bmatrix} \begin{Bmatrix} 0 & -1 \\ 1 & 1 \end{Bmatrix} = \begin{Bmatrix} 1 & 1 \\ -1 & 0 \end{Bmatrix} \neq I\] \[\begin{Bmatrix} 0 & -1 \\ 1 & 1 \end{Bmatrix} \begin{Bmatrix} 0 & -1 \\ 1 & 1 \end{Bmatrix} \begin{Bmatrix} 0 & -1 \\ 1 & 1 \end{Bmatrix} \begin{Bmatrix} 0 & -1 \\ 1 & 1 \end{Bmatrix} \begin{Bmatrix} 0 & -1 \\ 1 & 1 \end{Bmatrix} \begin{Bmatrix} 0 & -1 \\ 1 & 1 \end{Bmatrix} = \begin{Bmatrix} 1 & 0 \\ 0 & 1 \end{Bmatrix} = I,\] so \[\left| \begin{Bmatrix} 0 & -1 \\ 1 & 1 \end{Bmatrix} \right| = 6.\] \\

Note that for positive integer \(n\), \[\begin{Bmatrix} 1 & -1 \\ 0 & 1 \end{Bmatrix} \begin{Bmatrix} 1 & -n \\ 0 & 1 \end{Bmatrix} = \begin{Bmatrix} 1 & -n - 1 \\ 0 & 1 \end{Bmatrix},\] so this matrix has infinite order as no positive amount of combinations of itself will lead to the identity matrix.

\section*{(4)}
Show that a group of order 3 must be cyclic. \\

Proof by contradiction: \\

Let \(G\) be a non-cyclic group of order 3. We can write the group as \(G = \{e, a, b\}\). As \(G\) is non-cyclic, we can say that \(b\) is not a power of \(a\). Therefore, \(aa\) must equal \(e\), as it cannot equal \(b\) or \(e\). Since \(ae = a\), we must have \(ab = b\), as it cannot be any other element in \(G\). However, this is a contradiction, as \(a\) is not the identity element but is simultaneously acting as the identity for \(b\), so this group must be cyclic (and \(b = a^{2}\)).

\section*{(5)}
Let \(G\) be a group and let \(g \in G\) have infinite order, i.e. \(|g| = \infty\). For \(k,l \in \mathbb{Z}\), show that the cyclic groups generated by \(g^{k}\) and \(g^{l}\) are equal if and only if \(k = \pm l\). \\

\((\implies)\) Proof by contradiction: \\

Since \(g\) has infinite order, \(\forall n,m \in Z\), \(g^{n} = g^{m}\) holds true if and only if \(n = m\), as \(g^{n-m} = e\) implies that \(n - m = 0 \implies n = m\). Assume that \(\left<g^{k}\right> = \left<g^{l}\right>\). For sake of contradiction, assume that \(k \neq \pm l\). This implies that for all elements \(g^{nk} \in \left<g^{k}\right>\), there is an equivalent element \(g^{ml} \in \left<g^{l}\right>\) such that \(g^{nk} = g^{ml} \implies g^{nk-ml} = g^{0} = e \implies nk = ml\).  We now have the result that \(\forall n\in \mathbb{Z}\), \(\exists m \in \mathbb{Z}\) such that \(nk = ml\) while \(k \neq \pm l\). This is not possible, so \(m = pm l\). \\

\((\impliedby)\) Direct proof: \\

Let \(m = \pm l\). The group generated by \(g^{k}\) is then all \(g\) to integer multiples of \(k \text{ or } \pm l\), which certainly holds true. We can also see this when listing the group elements: \[\left<g^{k}\right> = \{..., g^{-2k}, g^{-k}, e, g^{k}, g^{2k}, ...\} = \{..., g^{\pm 2l}, g^{\pm l}, e, g^{\mp l}, g^{\mp 2l}\} = \left< g^{l} \right>.\]

\end{document}