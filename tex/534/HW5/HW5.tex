\documentclass[12pt,letterpaper]{article}
\usepackage[utf8]{inputenc}
\usepackage{amsmath}
\usepackage{amsfonts}
\usepackage{amssymb}
\usepackage{graphicx}
\usepackage{lastpage}
\usepackage[left=1.00in, right=1.00in, top=1.00in, bottom=1.00in]{geometry}
\usepackage{fancyhdr}
\title{Math 534 HW 5}
\author{Morgan Gribbins}
\date{}
\pagestyle{fancy}
\fancyhf{}
\lhead{Page \thepage\ of \pageref{LastPage}}
\begin{document}
	
\maketitle

\textbf{(1)} Let \(\alpha = \begin{pmatrix} 1 & 2 & 3 & 4 & 5 & 6 \\ 2 & 1 & 3 & 5 & 4 & 6 \end{pmatrix}\) and \(\beta = \begin{pmatrix} 1 & 2 & 3 & 4 & 5 & 6 \\ 6 & 1 & 2 & 4 & 3 & 5 \end{pmatrix}\).\\



\textbf{(1a)} Compute the following: \(\alpha^{-1}, \beta^{-1}, \alpha\beta,\) and \(\beta\alpha\).\\

\[\alpha^{-1} = \begin{pmatrix} 1 & 2 & 3 & 4 & 5 & 6 \\ 2 & 1 & 3 & 5 & 4 & 6 \end{pmatrix}.\] \[\beta^{-1} = \begin{pmatrix} 1 & 2 & 3 & 4 & 5 & 6 \\ 2 & 3 & 5 & 4 & 6 & 1 \end{pmatrix}.\] \[\alpha\beta = \begin{pmatrix} 1 & 2 & 3 & 4 & 5 & 6 \\ 6 & 2 & 1 & 5 & 3 & 4 \end{pmatrix}.\] \[\beta\alpha = \begin{pmatrix} 1 & 2 & 3 & 4 & 5 & 6 \\ 1 & 6 & 2 & 3 & 4 & 5 \end{pmatrix}.\] \\

\textbf{(1b)} Write \(\alpha\) and \(\beta\) as products of disjoint cycles.\\

\[\alpha = (12)(45) \text{ and } \beta = (16532).\] \\

\textbf{(1c)} Compute \(\beta^{-1}\alpha\beta\) and \(\alpha^{-1}\beta\alpha\) and write them of products of disjoint cycles. How do their ``cycle structures" compare to those of \(\alpha\) and \(\beta\) (respectively)?\\

\[\beta^{-1}\alpha\beta = \begin{pmatrix} 1 & 2 & 3 & 4 & 5 & 6 \\ 1 & 3 & 2 & 6 & 5 & 4 \end{pmatrix} = (23)(46).\] \[\alpha^{-1}\beta\alpha = \begin{pmatrix} 1 & 2 & 3 & 4 & 5 & 6 \\ 2 & 6 & 1 & 3 & 5 & 4 \end{pmatrix} = (12643).\] Both \(\beta^{-1}\alpha\beta\) and \(\alpha\) are comprised of two disjoint 2-cycles and both \(\alpha^{-1}\beta\alpha\) and \(\beta\) are comprised of one 5-cycle. \\

\textbf{(2)} How many elements of order 4 are there in \(\mathfrak{S}_{6}\)? How many of order 2? Justify your answers.\\

All elements of order \(4\) in \(\mathfrak{S}_{6}\) have the disjoint cyclic forms \((abcd)\) or \((abcd)(ef)\). \\

The first case, \((abcd)\) has \(6\times5\times4\times3 = 360\) possible elements, without disregarding identical cycles with different letter placement. For each actual unique cycle \((abcd)\), can be written as \((abcd)\), \((bcda)\), \((cdab)\), or \((dabc)\), so the amount of one 4-cycle elements of \(\mathfrak{S}_{6}\) are \(360/4 = 90.\) \\

The second case, \((abcd)(ef)\) also has \(90\) cases, as there is no ``choice" for the final 2-cycle, as there are only two elements left to choose from.\\

This means that there are a total of \(180\) order 4 elements in \(\mathfrak{S}_{6}\). \\
 
Elements of order 2 in \(\mathfrak{S}_{6}\) are those able to be expressed in the forms \((ab)\), \((ab)(cd)\), or \((ab)(cd)(ef)\). For the first case, there are \(6 \times 5 = 30\) combinations, with half of them being duplicates, so \(15\) total of the form \((ab)\). For the second case, there are \(6 \times 5\times 4 \times 3 = 360\) combinations, with \(3/4\) of them being duplicates (and half of them being in different cycle order), so \(45\) of the form \((ab)(cd)\). For the the third case, there are \(6! = 720\) combinations before removing duplicates. \(7/8)\) are identical up to rearrangement, and \(5/6\) are identical up to cycle placement, so we have \(15\) total of this form. \\

This means that there are a total of 75 elements of order 2 in \(\mathfrak{S}_{6}\). \\

\textbf{(3)} A \textit{perfect shuffle} is performed on a deck by splitting the deck into two halves (the top and bottom half, assuming an even number of cards), and then interweaving the decks so that every other card comes from the same half. There are two ways to do this: an \textit{out shuffle}, which preserves the first and last cards of the deck, and an \textbf{in shuffle}, which does not.\\

\textbf{(3a)} Show that after 8 perfect \textbf{out shuffles}, a deck of 52 cards is returned to its original position, but no fewer number of perfect out shuffles will do this.\\

A perfect out shuffle on 52 cards begins takes the 1st card to the 1st place, then the 27th card to the 2nd place, 2nd to 3rd, 28th to 4th, etc, so the perfect out shuffle permutation can be defined by \[\sigma  = \begin{pmatrix} 1 & 2 & 3 & 4 & ... & 49 & 50 & 51 & 52\\ 1 & 3 & 5 & 7 & ... & 46 & 48 & 50 & 52 \end{pmatrix},\] which maps \(n \in \{x \in \mathbb{N} : 0 < x \leq 52\}\) to \(2n-1\) if \(n \leq 26\), and to \(2n \mod 52\) for \(n > 26\). The disjoint cyclic representation of \(\sigma\) is then \[\sigma = (1)(2\ 3\ 5\ 9\ 17\ 33\ 14\ 27)(4\ 7\ 13\ 25\ 49\ 46\ 40\ 28)(6\ 11\ 21\ 41\ 30\ 8\ 15\ 29)\]\[(10\ 19\ 37\ 22\ 43\ 34\ 16\ 31)(12\ 23\ 45\ 38\ 24\ 47\ 42\ 32)(18\ 35)(20\ 39\ 26\ 51\ 50\ 48\ 44\ 36)(52),\] which has cycles of length 1, 8, 8, 8, 8, 8, 2, 8, and 1. The order of this element is then equal to the least common multiple of all of these, which is \(8\). Therefore, \(8\) perfect out shuffles are required to return a deck of cards to its original position. \\

\textbf{(3b)} How many perfect \textbf{in shuffles} does it take to return a deck of 10 cards back to its original position?\\

The in shuffle on a deck of ten cards takes the 6th card to the 1st position, 1st to the 2nd, 7th to 3rd, etc can be described by the permutation \[\sigma = \begin{pmatrix} 1 & 2 & 3 & 4 & 5 & 6 & 7 & 8 & 9 & 10 \\ 2 & 4 & 6 & 8 & 10 & 1 & 3 & 5 & 7 & 9 \end{pmatrix} = (1\ 2\ 4\ 8\ 5\ 10\ 9\ 7\ 3\ 6).\] Therefore, this permutation has order 10, so it takes 10 perfect in shuffles to return a deck of 10 cards to its original position. \\

\textbf{(4)} Let \((ab)\) and \((cd)\) be distinct 2-cycles in \(\mathfrak{S}_{n}\). Show that these elements commute if and only if they are disjoint. Using this, show that \(\mathfrak{S}_{n}\) is not abelian if \(n \geq 3\).\\

Proof of \((\implies)\) by contrapositive. Assume that \((ab)\) and \((cd)\) are not disjoint. Without loss of generality, this implies that we can set \(b = c\), so \((ab) = (ac)\). These two cycles commute if \((ab)(cd) = (cd)(ab)\), which is the same as \((ac)(cd) = (cd)(ac)\). We can show that this cannot be true by showing what these two cycles do to the \(c\) element. We have \((ac)(cd)[c] = d\) and \((cd)(ac)[c] = a\), so these do not commute, and the contrapositive is true. \\

Proof of \((\impliedby)\). Assume that \((ab)\) and \((cd)\) are disjoint. To prove that these cycles commute, we will show that \((ab)(cd)\) and \((cd)(ab)\) provide the same results for a, b, c, and d. Note that \((ab),(cd)\) disjoint implies \((cd)[a] = a\), \((cd)[b] = b\), \((ab)[c] = c\), \((ab)[d] = d\). \[(ab)(cd)[a] = b,\ (cd)(ab)[a] = b,\]\[(ab)(cd)[b] = a,\ (cd)(ab)[b] = a,\]\[(ab)(cd)[c] = d,\ (cd)(ab)[c] = d,\]\[(ab)(cd)[d] = c,\ (cd)(ab)[d] = c,\] and all other letters that are permuted upon remain fixed. Therefore, these cycles commute. \\

Proof that \(\mathfrak{S}_{n}\) is not abelian if \(n \geq 3.\) The elements \((12)\) and \((23)\) are both in \(\mathfrak{S}_{n}\) if \(n \geq 3\), as 1, 2, and 3 are both less than or equal to \(3 \leq n\). By the prior proof, we know that because these are non disjoint, they do not commute. Therefore, these groups are not abelian. \\

\textbf{(5)} Let \(\alpha = (123)(145) \in \mathfrak{S}_{5}.\) Compute \(\alpha^{99}\).\\

\[\alpha = \begin{pmatrix} 1 & 2 & 3 & 4 & 5 \\ 2 & 3 & 1 & 4 & 5 \end{pmatrix} \begin{pmatrix} 1 & 2 & 3 & 4 & 5 \\ 4 & 2 & 3 & 5 & 1\end{pmatrix} = \begin{pmatrix} 1 & 2 & 3 & 4 & 5 \\ 4 & 3 & 1 & 5 & 2 \end{pmatrix} = (1\ 4\ 5\ 2\ 3).\] Therefore, the order of \(\alpha\) is 5. This means that any \(\alpha^{5n}=e\), \(n \in \mathbb{N}\). \[\alpha^{99} = \alpha^{95}\alpha^{4} \implies \alpha^{99} = \alpha^{4} = (1\ 3\ 2\ 5\ 4).\]

\textbf{(6)} Show that we cannot find an element \(\sigma \in \mathfrak{S}_{7}\) so that \(\sigma^{2} = (1234).\) \textit{Hint: what would be the possible orders of such an element?}\\

Because \(\sigma^{2} = (1234)\), the order of \(\sigma^{2}\) is 4. This implies that \(\sigma^{8} = e\), so \(|\sigma| \div 8\). \(|\sigma| = 1,2,4,\) or \(8\). It cannot be \(1\) or \(2\) as \(\sigma^{2} \neq e \implies \sigma \neq e\). It also cannot be \(4\) as \((1234)(1234) = (13)(24) \neq e\), so the order of \(\sigma\) must equal 8. This cannot be true, however, as the order of an element in \(\mathfrak{S}_{7}\) must divide \(7!\). Therefore, there can be no \(\sigma \in \mathfrak{S}_{7}\) such that \(\sigma^{2} = (1234)\). 



\end{document}