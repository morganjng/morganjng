\documentclass[12pt,letterpaper]{article}
\usepackage[utf8]{inputenc}
\usepackage{amsmath}
\usepackage{amsfonts}
\usepackage{amssymb}
\usepackage{graphicx}
\usepackage{lastpage}
\usepackage[left=1.00in, right=1.00in, top=1.00in, bottom=1.00in]{geometry}
\usepackage{fancyhdr}
\title{Math 534 HW 8}
\author{Morgan Gribbins}
\date{}
\pagestyle{fancy}
\fancyhf{}
\lhead{Page \thepage\ of \pageref{LastPage}}
\begin{document}
	
\maketitle

\textbf{(1)} Let \(H, K \leq G\) be two subgroups of a given group \(G\). Show that for \(a \in G\) the coset \(a(H \cap K)\) is equal to \((aH)\cap (aK)\), i.e. is equal to the intersection of the cosets \(aH\) and \(aK\). \\

Let \(ax \in a(H \cap K)\). By definition of cosets, \(x \in H \cap K\). This implies that \(ax \in aH\) and \(ax \in aK\), so \(ax \in (aH) \cap (aK)\). Now, let \(ax \in (aH) \cap (aK)\), which means that \(ax \in aH \implies x \in H\) and \(ax \in aK \implies x \in K\). Therefore, \(x \in H \cap K \implies ax \in a(H \cap K)\). Because \(a(H \cap K) \subseteq (aH) \cap (aK) \text{ and } (aH) \cap (aK) \subseteq a(H \cap K)\), these two sets are equal. \\

\textbf{(2)} Use the Theorem of Lagrange (and its consequences) to show that \(|(\mathbb{Z}/n)^{\times}|\) is always even when \(n > 2\). \\

Let \(n > 2, n \in \mathbb{N}\). The group \(G = (\mathbb{Z}/n)^{\times}\) then consists of the numbers relatively prime to \(n\). Consider the group generated by the element \(n-1 \in G\) (this element must be in \(G\) because \(n-1\) is relatively prime to \(n\) for \(n > 2\)!). This group \(H = \left<n-1\right> = \{1, n-1\}\), as \((n-1)^{2} = n^{2} - 2n + 1 \equiv 1 \mod n\). As this group has order \(2\), by Lagrange's Theorem, \(|G|\) must be even as the order of a subgroup of \(G\) must divide the order of \(G\). \\

\textbf{(3)} Suppose \(G\) is a finite abelian group and \(|G|\) is odd. Show that the product of all the elements in \(G\) is the identity. Is the same true if \(|G|\) is even? \\

Let \(a \in G\) such that \(|a| = k+1\), and let \(|G| = n\). As \(|G|\) is odd, there are no elements with even orders. Consider the cyclic group generated by \(a\), which is \(\{e, a, a^{2}, ..., a^{k}\}\). Now, consider its \(n/(k+1)\) cosets, which partitions the entirety of \(G\), so the product of all of their elements is the product of all elements of \(G\).\\

First, consider the powers of \(a\) in this product. The product over \(\left<a\right>\) is equal to \(a^{k(k+1)/2}\), and there are \(n/(k+1)\) cosets, so the power of \(a\) after multiplying these cosets is \(a^{nk/2} = e\), as an element of \(G\) to the power of a multiple of \(|G|\) is equal to the identity, by the Theorem of Lagrange. \\

For any element \(b \in G\), if \(b\left<a\right>\) is a coset of \(\left<a\right>\), then \(b^{-1}\left<a\right>\) must also be a coset of \(\left<a\right>\), as \(b\notin \left<a\right>\implies b^{-1} \notin \left<a\right>\). The coset \(b\left<a\right> = \{b, ba, ba^{2}, ..., ba^{k}\}\), and the power of \(b\) in the product over this coset is \(b^{k+1}\). As \(b^{-1}\left<a\right>\) is also a coset, \(b^{-(k+1)}\) is also present in the product, so the total product over this set must be equal to the identity of \(G\). \\

This does not hold true if \(|G|\) is even. Consider the set \(\{1,2\}\) under multiplication modulo 3, which has the product of \(2 \neq 1 = e\). \\ 

\textbf{(4)} Let \(|G| = 8\). Prove that \(G\) has an element of order \(2\). \\

Let \(g \in G\). If \(|g| = 8\), there must be an element of order \(2\) in \(G\), namely \(g^{4}\). If \(|g| = 4\), there must be an element of order \(2\) in \(G\), namely \(g^{2}\). If \(|g| = 1\), then \(g = e\) and there must be some other element in \(G\) of order \(2, 4,\) or \(8\), which all imply that there is some element with order \(2\) in \(G\), so there must be an element of order \(2\) in \(G\). \\

\textbf{(5)} Let \(G\) be a group and let \(H, K \leq G\) be subgroups satisfying \(|H| = 20\) and \(|K| = 28\). Prove that \(H \cap K\) is abelian. (Hint: Start by computing the order of \(H \cap K\). We've seen that \(H \cap K\) is a subgroup of \(G\), but it can also be viewed as a subgroup of...). \\

The group \(H\cap K\) can be viewed as a subgroup of either \(H\) or \(K\), so the order of \(H \cap K\) must divide the order of both \(H\) and \(K\). The only common factors that the orders of \(H\) and \(K\) share are \(2\) and \(4\). If \(|H \cap K| = 2\), then this group can be represented by the set \(\{e,a\}\) which is clearly abelian as \(ea = ae = a\). If \(|H \cap K| = 4\), then all elements of said group must be of order 2 or 4. If they are of order 4, then the group is cyclic and necessarily abelian. If the elements are of order 2, then this group can be described by the set \(\{e,a,b,ab\}\), where \(a^{2} = b^{2} = (ab)^{2} = e\), and it must be abelian. Therefore, \(H \cap K\) must be abelian. \\

\end{document}