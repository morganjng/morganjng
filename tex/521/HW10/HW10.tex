\documentclass[12pt,letterpaper]{article}
\usepackage[utf8]{inputenc}
\usepackage{amsmath}
\usepackage{amsfonts}
\usepackage{amssymb}
\usepackage{graphicx}
\usepackage{lastpage}
\usepackage[left=1.00in, right=1.00in, top=1.00in, bottom=1.00in]{geometry}
\usepackage{fancyhdr}
\title{Math 521 HW 10}
\author{Morgan Gribbins}
\date{}
\pagestyle{fancy}
\fancyhf{}
\lhead{Page \thepage\ of \pageref{LastPage}}
\begin{document}
	
\maketitle

\textbf{Exercise 3.2.5.} Prove Theorem 3.2.8. \\

\textit{A set} \(F \subseteq \mathbb{R}\) \textit{is closed if and only if every Cauchy sequence contained in} \(F\) \textit{has a limit that is also an element of} \(F\). \\

Proof of (\(\implies\) closed) by contrapositive. Assume that some set \(F \subseteq \mathbb{R}\) is not closed. This implies that there is some \(x\in \mathbb{R}\) that is a limit point of \(F\), not contained in \(F\). As \(x\) is a limit point of \(F\), there is some Cauchy (implied by convergence) sequence entirely in \(F\), that has a limit outside of \(F\). Therefore, \(F\) not closed implies that not every Cauchy sequence contained in \(F\) has a limit that is also contained in \(F\). \\

Proof of (\(\impliedby\) closed). Assume that  some set \(F \subseteq \mathbb{R}\) is closed. Let \((a_{n})\) be a Cauchy sequence fully contained in \(F\). This implies that \((a_{n})\) is convergent, by theorem. Therefore, by the definition of a closed set, \((a_{n})\) converges to some \(x \in F\), which proves that \(F\) being closed implies that all Cauchy sequences contained in \(F\) has a limit contained in \(F\). \\

\textbf{Exercise 3.2.8.} Assume \(A\) is an open set and \(B\) is a closed set. Determine if the following sets are definitely open, definitely closed, both, or neither. \\

\textbf{3.2.8.a.} \(\overline{A\cup B}\) \\

This is definitely closed, as the closure of any set is closed. \\

\textbf{3.2.8.b.} \(A\setminus B = \{x \in A \text{ and } x \notin B\}\). \\

This is definitely open, as it is the intersection of two open sets (\(A\) and \(B^{c}\)). \\

\textbf{3.2.8.c.} \((A^{c} \cup B)^{c}\) \\

This is definitely open, as it is the intersection of two open sets: \((A^{c} \cup B)^{c} = A \cap B^{c}\). \\

\textbf{3.2.8.d.} \((A \cap B) \cup (A^{c} \cap B)\) \\

This set is the union of a closed set \((A^{c} \cap B)\) and the intersection of a closed and open set \((A \cap B)\), which is neither, so this set is neither closed nor open. \\

\textbf{3.2.8.e.} \(\overline{A}^{c} \cap \overline{A^{c}}\) \\

This is the complement of the closure of \(A\), so it is definitely open. \\

\textbf{Exercise 3.2.10.} Only one of the following three descriptions can be realized. Provide an example that illustrates the viable description, and explain why the other two cannot exist. \\

\textbf{(i)} A countable set contained in \([0,1]\) with no limit points. \\

\textbf{(ii)} A countable set contained in \([0,1]\) with no isolated points. \\

\textbf{(iii)} A set with an uncountable number of isolated points. \\

The only one of these that is possible is \textbf{(ii)}. \textbf{(i)} is impossible, as Bolzano-Weierstrass demands that there be a convergent subsequence within any bounded set, and \textbf{(iii)} can't be true due to the density of the real numbers. An example of \textbf{(ii)} is given by the set \(S = \{ 1/n: n \in \mathbb{N}\}\). \\

\textbf{Exercise 3.3.4.} Assume \(K\) is compact and \(F\) is closed. Decide if the following sets are definitely compact, definitely closed, both, or neither. \\ 

\textbf{3.3.4.a.} \(K \cap F\) \\

This is definitely compact, as it is bounded (as all elements in \(K\) are bounded) and closed (as it is a intersection of two closed sets). \\

\textbf{3.3.4.b.} \(\overline{F^{c} \cup K^{c}}\) \\

This is definitely closed, but not necessarily compact. \\

\textbf{3.3.4.c.} \(K \setminus F = \{x \in K \text{ and } x \notin F\}\) \\

This is not necessarily closed or compact. \\

\textbf{3.3.4.d.} \(\overline{K \cap F^{c}}\) \\

This is definitely closed, and definitely compact, as \(K\) is compact. \\

\textbf{Exercise 3.3.5.} Decide whether the following propositions are true or false. If the claim is valid, supply a short proof, and if the claim is false, provide a counterexample. \\

\textbf{3.3.5.a.} The arbitrary intersection of compact sets is compact. \\

This is true, as this intersection is bounded by the lowest bound of all the sets, and an arbitrary intersection of closed sets is closed. \\

\textbf{3.3.5.b.} The arbitrary union of compact sets is compact. \\

This is not true. Consider the sets \(A_{n} = [0,n]\), which are all compact individually; the union of these sets \(\cup_{n=1}^{\infty} A_{n}\) is the set of all positive (and zero) real numbers, which is not compact. \\

\textbf{3.3.5.c.} Let \(A\) be arbitrary, and let \(K\) be compact, then, the intersection \(A \cap K\) is compact. \\

This is not true, as if \(A\) is open, then \(A \cap K\) is not necessarily closed, and as such not necessarily compact. Take the sets \(K = [0,1]\) and \(A = (1/2, 1)\), whose intersection is \((1/2, 1)\), which is open and as such, not compact. \\

\textbf{3.3.5.d.} If \(... \subseteq F_{4} \subseteq F_{3} \subseteq F_{2} \subseteq F_{1}\) is a nested sequence of nonempty closed sets, then the intersection \(\cap_{n=1}^{\infty} F_{n} \neq \emptyset\). \\

This is not necessarily true. Take the sets \(F_{n} = \{x \in \mathbb{R} : x \geq n\}\), which have an empty infinite intersection. \\

\end{document}