\documentclass[12pt,letterpaper]{article}
\usepackage[utf8]{inputenc}
\usepackage{amsmath}
\usepackage{amsfonts}
\usepackage{amssymb}
\usepackage{graphicx}
\usepackage[left=1.00in, right=1.00in, top=1.00in, bottom=1.00in]{geometry}
\usepackage{fancyhdr}
\pagestyle{fancy}
\title{Math 521 HW 3}
\author{Morgan Gribbins}
\fancyhf{}
\lhead{Page \thepage\ of 4}
\begin{document}
	
\maketitle

\section*{1.4.2.}
Let \(A \subseteq \mathbb{R}\) be nonempty and bounded above, and let \(s \in \mathbb{R}\) have the property that for all \(n \in \mathbb{N}\), \(s + \frac{1}{n}\) is an upper bound for \(A\) and \(s - \frac{1}{n}\) is not an upper bound for \(A\). Show \(s=supA\). \\

For \(s=supA\) to be true, then we must first show that \(s\) is an upper bound on \(A\). To do this, we must show that \(s + \frac{1}{n}\) being an upper bound on \(A\) implies that \(s\) is an upper bound on \(A\). If \(s\) were not an upper bound on \(A\), then there must be some element \(a \in A\) that satisfies \(\forall n \in \mathbb{N}\), \(s < a\) and \(a \leq s + \frac{1}{n}\). Note that for all \(\epsilon > 0\), \(\exists n \in \mathbb{N}\) such that \(\frac{1}{n} < \epsilon\), by Archimedes' principle. Therefore, \(\forall n \in \mathbb{N}\), \(\exists \epsilon > 0\) such that \(s + \frac{1}{n} < s + \epsilon\), so \(a < s + \epsilon\) and \(a - s < \epsilon\). From this, \(s = a\) by theorem, which contradicts the assumption that \(s < a\) so \(s\) must be an upper bound of \(A\).

Now, we must show that \(s = supA\) with the knowledge that \(s\) is an upper bound on \(A\) and that \(\forall n \in \mathbb{N}\), \(s - \frac{1}{n}\) is not an upper bound on \(A\). The statement that \(s - \frac{1}{n}\) is not an upper bound for \(A\) means that \(\forall n \in \mathbb{N}\), \(\exists \alpha \in A\) such that \(s - \frac{1}{n} < \alpha\). From Archimedes' Principle, we have the fact that \(\forall n \in \mathbb{N}\), \(\exists \epsilon > 0\) such that \(\frac{1}{n} < \epsilon\), so without loss of generality we have \(\forall \epsilon > 0\), \(\exists \alpha \in A\) such that \(s - \epsilon < \alpha\). By Lemma 1.3.8, this implies that \(s = supA\).

\section*{1.4.3.}
Prove that \(\bigcap_{n=1}^{\infty} (0,1/n)=\emptyset\). Notice that this demonstrates that the intervals in the Nested Interval Property must be closed for the conclusion of the theorem to hold. \\

Proof by contradiction. Let \(A = \bigcap_{n=1}^{\infty} (0,1/n) = \emptyset\), and let \(a \in A\) (note that \(a > 0\) must hold). This means that \(\forall n \in \mathbb{N}\), \(a \in (0,1/n)\), i.e. there is some \(a > 0\) which is smaller than all \(1/n\). By Archimedes' Principle, \(\forall \epsilon > 0\), \(\exists n \in \mathbb{N}\) such that \(1/n < \epsilon\), which indicates that this property cannot hold for \(a\), so a contradiction is met and \(A\) must be empty.

\section*{1.5.1.}
Finish the following proof for Theorem 1.5.7. \\

Assume \(B\) is a countable set. Thus, there exists \(f: \mathbb{N} \rightarrow B\), which is \(1-1\) and onto. Let \(A \subseteq B\) be an infinite subset of \(B\). We must show that \(A\) is countable. \\

Let \(n_{1} = min\{n \in \mathbb{N} : f(n) \in A\}\). As a start to a definition of \(g : \mathbb{N} \rightarrow A\), set \(g(1) = f(n_{1})\). Show how to inductively continue this process to produce a \(1-1\) function \(g\) from \(\mathbb{N}\) onto \(A\). \\

Let \(A_{k}\) be the set \(A\), removing the \(k\) least elements from \(A\)---so \(A_{1} = A \setminus \{n_{1}\}\). Then, let \(n_{k} = min\{n \in \mathbb{N} : f(n) \in A_{k-1}\}\). Now, map \(g(k) = f(n_{k})\). This is a \(1-1\) function from \(\mathbb{N}\) onto \(A\) because all elements of \(A\) are mapped to uniquely by elements of \(\mathbb{N}\) by the well-ordering principle.

\section*{1.5.3.}
Use the following outline to supply proofs for the statements in Theorem 1.5.8.

\subsection*{(a) First, prove statement (i) for two countable sets, \(A_{1}\) and \(A_{2}\). Then, explain how the more general statement in (i) follows.}

First, we make the substitution \(B_{2} = A_{2} \setminus A_{1} = \{x \in A_{2} : x \notin A_{1}\}\) (which is provided in the question). We will then look at the set \(A_{1} \cup B_{2} = A_{1} \cup A_{2}\). As \(A_{1}\) and \(B_{2}\) are countable sets, we can enumerate the sets as \(A_{1} = \{a_{1}, a_{2}, a_{3},...\}\) and \(B_{2} = \{b_{1}, b_{2}, b_{3}, ...\}\). We can then enumerate \(A_{1} \cup B_{2} = A_{1} \cup A_{2} = \{a_{1}, b_{1}, a_{2}, b_{2}, a_{3}, b_{3}, ...\}\), so \(A_{1} \cup A_{2}\) must be countable. \\

The broader statement that \(\bigcup_{n = 1}^{m} A_{n}\) is countable when \(A_{n}\) is countable follows from this result as each individual union is countable, so the totality of the unions must be countable. Alternatively, we can use the enumeration of \(\bigcup_{n = 1}^{m} A_{n} = \{a_{11}, a_{12}, ..., a_{1m}, a_{21}, a_{22}, ..., a_{2m}, ...\}\), with the first element of the \(m\) sets, then the second, and so on.

\subsection*{(b) Explain why induction cannot be used to prove part (ii) of Theorem 1.5.8 from part (i).}

This induction cannot be used to prove part (ii) of Theorem 1.5.8 from part (i) because with an infinite amount of sets, the same enumeration cannot hold as there would be infinite first, second, etc., elements of the sets.

\subsection*{(c) Show how arranging \( \mathbb{N} \) into the two dimensional array provided by the book leads to a proof of Theorem 1.5.8 (ii).}

By arranging \(\mathbb{N}\) into this two dimensional array and laying out the infinite sets, a natural enumeration rises: 

\begin{center}
	\begin{tabular}{c c c c c}
		1 & 3 & 6 & 10 & ... \\
		2 & 5 & 9 & ... \\
		4 & 8 & ... \\
		7 & ... \\
		... \\
	\end{tabular}
	\[\downarrow\]
	\begin{tabular}{c c c c c}
		\(a_{11}\) & \(a_{12}\) & \(a_{13}\) & \(a_{14}\) & ... \\
		\(a_{21}\) & \(a_{22}\) & \(a_{23}\) & ... \\
		\(a_{31}\) & \(a_{32}\) & ... \\
		\(a_{41}\) & ... \\
		... \\
	\end{tabular}
\end{center}

So a countably infinite union of countably infinite sets can be enumerated by \[\bigcup_{n = 1}^{\infty} A_{n} = \{a_{11}, a_{21}, a_{12}, a_{31}, a_{22}, ...\},\] and as such, is countable.

\section*{HW3.1:}
Proof that there exists some \(\alpha\) that satisfies \(\alpha^{2} = 3\).

Let \(T = \{x \in \mathbb{R} : x \geq 1\ and\ x^{2} < 3\}\), and let \(\alpha = supT\). For the sake of contradiction, assume \(\alpha^{2} < 3\), i.e. \(\alpha^{2} + \delta = 3\). This \(\alpha\) cannot be an upper bound for \(T\) because there must be some \(\alpha + \epsilon \in T,\ \epsilon > 0\) which is greater than \(\alpha\). \[(\alpha + \epsilon)^{2} = \alpha^{2} + 2\alpha\epsilon + \epsilon^{2} \implies \alpha^{2} + 2\alpha\epsilon + \epsilon^{2} < 2,\] so this \(\epsilon\) and as such \(\alpha + \epsilon\) exists when \(\alpha\) and \(\epsilon\) satisfies this inequality (which is guaranteed by the density of the rationals in \(\mathbb{R}\)). \\

Now, let \(\alpha^{2} > 3\). This \(\alpha\) is a greater bound on \(T\), because it is greater than all elements in \(T\). This \(\alpha \neq supA\) as \(\exists \epsilon > 0,\ \forall a \in T,\ \alpha - \epsilon \geq a\) (negation of Lemma 1.3.8), by letting \(\alpha - \epsilon > 3\), which must exist by the density of the reals. Therefore, \(\alpha^{2}\) cannot be less than or greater than \(3\), so \(\alpha^{2} = 3\).

\section*{HW3.2:}
Proof that there exists some \(\alpha\) that satisfies \(a^{3} = 2\). 

Let \(T = \{x \in \mathbb{R} : x \geq 1\ and\ x^{2} < 3\}\), and let \(\alpha = supT\). For the sake of contradiction, assume that \(\alpha^{3} < 2\). Then, consider \(\alpha + 1/n\), for positive n. \[(\alpha + 1/n)^{3} = \alpha^{3} + 3\alpha^{2}/n + 3\alpha/n^{2} + 1/n^{3} \leq \alpha^{3} + (3\alpha^{2} + 3\alpha + 1)/n.\] Therefore, \(\alpha + 1/n \in T\) when \(1/n < (3-\alpha^{3})/(3\alpha^{2} + 3\alpha + 1)\), so \(\alpha \neq supT\) if \(\alpha^{3} < 2\). \\

Now, let \(\alpha^{3} > 2\). Then consider \(\alpha - 1/n\)---we will show that this may be an upper bound on \(T\), so \(\alpha\) cannot be \(supT\). \[(a - 1/n)^{3} = \alpha^{3} - 3\alpha^{2}/n + 3\alpha/n^{2} - 1/n^{3} \geq \alpha^{3} - (3\alpha^{2} + 1)/n.\] So, for \(n\) that satisfies \(-(3-\alpha^{3})/(3\alpha^{2} + 1) < 1/n\), we have \(\alpha - 1/n\) is an upper bound on \(T\), so \(\alpha \neq supT\) in this case. Therefore, \(\alpha^{3} = 2\).
 
 

\end{document}