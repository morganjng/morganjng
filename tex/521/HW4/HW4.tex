\documentclass[12pt,letterpaper]{article}
\usepackage[utf8]{inputenc}
\usepackage{amsmath}
\usepackage{amsfonts}
\usepackage{amssymb}
\usepackage{graphicx}
\usepackage[left=1.00in, right=1.00in, top=1.00in, bottom=1.00in]{geometry}
\usepackage{fancyhdr}
\pagestyle{fancy}
\title{Math 521 HW 4}
\author{Morgan Gribbins}
\date{}
\fancyhf{}
\lhead{Page \thepage\ of 4}
\begin{document}
	
\maketitle

\section*{2.2.2.}
Verify, using the definition of convergence of a sequence, that the following sequences converge to the proposed limit.

\subsection*{(a) \(\text{lim } \frac{2n+1}{5n+4} = \frac{2}{5}.\)} 

Given arbitrary \(\epsilon > 0\), let \(N_{\epsilon} \in \mathbb{N}\) satisfy \(N_{\epsilon} > \frac{20\epsilon + 3}{25\epsilon}\). This sequence converges if all \(n \in \mathbb{N}\) greater than or equal to \(N_{\epsilon}\) satisfy the inequality \[\left|\frac{2n-1}{5n-4} - \frac{2}{5}\right| = \left| \frac{3}{25n-20} \right| < \epsilon.\] This inequality that holds for \(n \geq N_{\epsilon}\) can be modified to get \[n > \frac{20\epsilon + 3}{25\epsilon} \implies 25n - 20 > \frac{3}{\epsilon} \implies \frac{3}{25n - 20} < \epsilon \implies \left|\frac{3}{25n - 20}\right| < \epsilon \implies \left| \frac{2n-1}{5n-4} - \frac{2}{5} \right| < \epsilon,\] so the requisite inequality holds if \(n \geq N_{\epsilon}\), so the sequence converges to \(\frac{2}{5}\).

\subsection*{(b) \(\text{lim } \frac{2n^{2}}{n^{3} + 3} = 0.\)}

Given arbitrary \(\epsilon > 0\), let \(N_{\epsilon} \in \mathbb{N}\) satisfy \(N_{\epsilon} > \frac{2}{\epsilon}\). For this sequence to converge, whenever \(n \in \mathbb{N}\) is greater than or equal to \(N_{\epsilon}\) the inequality \[\left|\frac{2n^{2}}{n^{3}+3}\right| < \epsilon\] must hold. Note that \[\left|\frac{2n^{2}}{n^{3}+3}\right| < \left|\frac{2n^{2}}{n^{3}}\right|\] holds for all \(n \in \mathbb{N}\), so if \(\left|\frac{2n^{2}}{n^{3}}\right| = \left|\frac{2}{n}\right| < \epsilon\), then \(\left|\frac{2n^{2}}{n^{3} + 3}\right| < \epsilon\), and the sequence must converge. With a little arithmetic, we have \[n > \frac{2}{\epsilon} \implies \frac{1}{n} < \frac{\epsilon}{2} \implies \frac{2}{n} < \epsilon \implies \left|\frac{2n^{2}}{n^{3}}\right| < \epsilon \implies \left|\frac{2n^{2}}{n^{3} + 3}\right| < \epsilon,\] so this sequence converges to \(0\).

\subsection*{(c) \(\text{lim } \frac{\text{sin }(n^{2})}{\sqrt[3]{n}} = 0.\)}

Note that the range of the sine function is the interval \([-1,1]\), so for all \(n\), the inequality \[\left|\frac{\text{sin }(n^{2})}{\sqrt[3]{n}}\right| \leq \left|\frac{1}{\sqrt[3]{n}}\right|\] holds. Given arbitrary \(\epsilon > 0\), let \(N_{\epsilon} \in \mathbb{N}\) satisfy \(N_{\epsilon} > \frac{1}{\epsilon^{3}}\). For this sequence to converge, whenever \(n \in \mathbb{N}\) greater than or equal to  \(N_{\epsilon}\) must satisfy the inequality \[\left|\frac{\text{sin }(n^{2})}{\sqrt[3]{n}}\right| < \epsilon.\] From the inequality set on \(n\) and \(N_{\epsilon}\), we have \[n > \frac{1}{\epsilon^{3}} \implies \frac{1}{n} < \epsilon^{3} \implies \frac{1}{\sqrt[3]{n}} < \epsilon \implies \left|\frac{1}{\sqrt[3]{n}}\right| < \epsilon \implies \left|\frac{\text{sin }(n^{2})}{\sqrt[3]{n}}\right| < \epsilon,\] so this sequence converges to \(0\).

\section*{2.2.4.}
Give an example of each or state that the request is impossible. For any that are impossible, give a compelling argument for why that is the case.

\subsection*{(a) A sequence with an infinite number of ones that does not converge to one.}

The sequence \((a_{n} = \{0,1,0,1,0,1,...\})\) i.e. the sequence where \(a_{n}\) is \(0\) when \(n\) is odd and \(1\) when \(n\) is even has an infinite number of ones but does not converge to one.

\subsection*{(b) A sequence with an infinite number of ones that converges to a limit that is not equal to one.}

There is no such sequence. To argue against this case, let \((a_{n})\) be a sequence with infinite ones and a proposed limit \(a \neq 1\). Let the difference between this \(a\) and \(1\) be equal to \(\delta\). Therefore, there are an infinite number of elements in the sequence such that \(\left|a_{n} - a\right| = \delta\). Selecting \(\epsilon = \delta/2\) provides us with the fact that an infinite number of elements in this sequence satisfy \(\left|a_{n} - a\right| \geq \epsilon\). As there is no way to insure that the infinite ones stop in this sequence, there can be no \(N_{\epsilon}\) such that all \(n \geq N_{\epsilon}\) satisfy \(\left|a_{n} - a\right| < \epsilon\) and as such, this cannot converge to \(1\).

\subsection*{(c) A divergent sequence such that for every \(n \in \mathbb{N}\) it is possible to find \(n\) consecutive ones somewhere in the sequence.}

The sequence \((a_{n}) = \{1, 0, 1, 1, 0, 1, 1, 1, 0, ...\}\) (the sequence of subsequent natural number amount of \(1\)s separated by a \(0\)) is divergent (as it doesn't converge to \(0\) or \(1\)) and has an \(n\)-length string of consecutive \(1\)s for all \(n\).

\section*{2.2.5.}
Let \([[x]]\) be the greatest integer less than or equal to \(x\). For each sequence, find lim \(a_{n}\) and verify it with the definition of convergence.

\subsection*{(a) \(a_{n} = [[5/n]]\).}

For this sequence, lim \(a_{n}\) is equal to \(0\). To prove this, take an arbitrary \(\epsilon > 0\), and take \(N_{\epsilon} > 5\). For all \(n \geq N_{\epsilon}\), we have \(\left|[[5/n]]\right| < \epsilon\), as \([[5/n]]\) for \(n \geq 6\) is equal to \(0\) and less than \(\epsilon\).


\subsection*{(b) \(a_{n} = [[(12 + 4n)/3n]]\).}

For this sequence, lim \(a_{n}\) is equal to \(1\). To prove this, take arbitrary \(\epsilon > 0\) and \(N_{\epsilon} > 6\). For all \(n \geq N_{\epsilon}\), we have \(\left|[[(12+4n)/3n]] - 1\right| = 0 < \epsilon\), as \([[(12+4n)/3n]] = 1\) for all \(n > 6\). \\

The statement following Definition 2.2.3 that the ``smaller the \(\epsilon\)-neighborhood, the larger \(N\) may have to be" does not hold in all cases, as the \(N\) in this case is independent of the \(\epsilon\).

\section*{2.2.6.}
Prove Theorem 2.2.7. To get started, assume \((a_{n}) \to a\) and also that \((a_{n}) \to b\). Now argue that \(a = b\). \\

Assume that for some arbitrary sequence \((a_{n})\), both \((a_{n}) \to a\) and \((a_{n}) \to b\) hold true. This means that for both \(a\) and \(b\), the statements \[\forall \epsilon > 0,\ \exists N_{\epsilon} \in \mathbb{N} \text{ such that } n \geq N_{\epsilon} \implies \left| a_{n} - a \right| < \epsilon \text{ and }\] \[\forall \epsilon > 0,\ \exists N_{\epsilon} \in \mathbb{N} \text{ such that } n \geq N_{\epsilon} \implies \left| a_{n} - b \right| < \epsilon.\] Fixing arbitrary \(\epsilon\), we get separate \(N_{\epsilon,b}\) and \(N_{\epsilon,a}\) that satisfy these statements. The larger of these two numbers guarantees that for all \(n\) greater than or equal to it satisfy these inequalities, so \(n \geq N_{\epsilon,a} \text{ and } n \geq N_{\epsilon,b} \implies \left|a_{n} - a\right| < \epsilon \text{ and } \left| a_{n} - b \right| < \epsilon\). This means that \((a_{n})\) converges to equality at some point to both \(a\) and \(b\) by the definition of equality, and by the transitivity of equality, this implies that \(a = b\).

\section*{HW4.1:}
For each \(n \in \mathbb{N}\), let \(x_{n} = 2 + \frac{(-1)^{n}}{n}\). Prove \(x_{n} \to 2\). \\

Given arbitrary \(\epsilon > 0\), let \(N_{\epsilon} > \frac{1}{n}\). For this sequence to converge, whenever \(n \in \mathbb{N}\) is greater or equal to \(N_{\epsilon}\), the following inequality must hold \[\left| 2 - \frac{(-1)^{n}}{n} - 2 \right| < \epsilon \implies \left| \frac{(-1)^{n}}{n} \right| < \epsilon \implies \frac{1}{n} < \epsilon.\] Since our \(n\) is greater than \(\frac{1}{\epsilon}\), this inequality must hold for desired \(n\), so this sequence does converge to \(2\).

\section*{HW4.2:}
For each \(n \in \mathbb{N}\), let \(x_{n} = (-1)^{n}\). Prove that \((x_{n})\) does not converge to any real number. \\

For the sake of contradiction, let \(x_{n} \to a\). This means that for all \(\epsilon > 0\), \(\exists N_{\epsilon}\) such that \(n \geq N_{\epsilon} \implies \left| (-1)^{n} - a \right| < \epsilon\). Let the value \(\left| 1 - a \right| = \delta\). If \(\delta > 0\), then setting \(\epsilon = \delta/2\) provides an \(\epsilon > 0\) for which there is no satisfactory \(N_{\epsilon}\). If \(\delta = 0\), then \(\left| -1 - a \right| = 2\), and setting \(\epsilon = 1\) provides a counterexample. Therefore, this sequence can not converge to any real number.

\end{document}