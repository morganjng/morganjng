\documentclass[12pt,letterpaper]{article}
\usepackage[utf8]{inputenc}
\usepackage{amsmath}
\usepackage{amsfonts}
\usepackage{amssymb}
\usepackage{graphicx}
\usepackage{lastpage}
\usepackage[left=1.00in, right=1.00in, top=1.00in, bottom=1.00in]{geometry}
\usepackage{fancyhdr}
\title{Math 521 HW 9}
\author{Morgan Gribbins}
\date{}
\pagestyle{fancy}
\fancyhf{}
\lhead{Page \thepage\ of \pageref{LastPage}}
\begin{document}
	
\maketitle

\textbf{Exercise 2.7.1.} Proving the Alternating Series Test amounts to showing that the sequence of partial sums \[s_{n} = a_{1} - a_{2} + ... \pm a_{n}\] converges. \\

\textbf{2.7.1.a.} Prove the Alternating Series Test by showing that \((s_{n})\) is a Cauchy sequence. \\

Given \(\epsilon > 0\), let \(N \in \mathbb{N}\) large enough such that \(a_{n}\) with \(n \geq N\) satisfies \(|a_{n+1}| < \epsilon\). Given \(m \geq n \geq N\), we then examine the expression \[|s_{m} - s_{n}| = |a_{n+1} - a_{n+2} + ... \pm a_{m}|.\] As \(a_{n+1} \leq a_{n}\) for all \(n\), this absolute value is less than \(|a_{n+1}|\), so we have \[|s_{m} - s_{n}| \leq |a_{n+1}| < \epsilon,\] so this series is Cauchy. \\

\textbf{2.7.1.b.} Supply another proof for this result using the Nested Interval Property. \\



\textbf{2.7.1.c.} Consider the subsequences \((s_{2n})\) and \((s_{2n+1})\), and show how the Monotone Convergence Theorem leads to a third proof for the Alternating Series Test. \\

The subsequence \((s_{2n}) = a_{1} - a_{2} + ... - a_{2n}\), and the subsequence \((s_{2n+1}) = a_{1} - a_{2} + ... - a_{2n} + a_{2n+1}\) and as \(a_{n} \geq 0\) for all \(n\), \(s_{2n+1} \geq s_{2n}\). Note that, because \(a_{n}\) is decreasing, all \(a_{2} \leq s_{n} \leq a_{1}\), and so all subsequences are bounded above and below. \(s_{2n}\) and \(s_{2n+1}\) both converge, as they are both monotone, and their difference converges to \(0\), so they converge to the same limit, so this sequence is convergent. \\

\textbf{Exercise 2.7.4.} Give an example to show that it is possible for both \(\sum x_{n}\) and \(\sum y_{n}\) to diverge but for \(\sum x_{n}y_{n}\) to converge. \\

The sum \(\sum 1/n\) diverges, but \(\sum 1/n^{2}\) converges. \\

\textbf{Exercise 3.2.3.} Decide whether the following sets are open, closed, or neither. If a set is not open, find a point in the set for which there is no \(\epsilon\)-neighborhood contained in the set. If a set is not closed, find a limit point that is not contained in the set. \\

\textbf{3.2.3.a.} \(\mathbb{Q}\) \\

This set is not open, as \(0\) has no \(\epsilon\)-neighborhoods entirely contained in \(\mathbb{Q}\) centered at it. It is also not closed, as the sequence \(1, 1.4, 1.41, ...\) (of rational numbers increasingly approaching without surpassing \(\sqrt{2}\)) converges to \(\sqrt{2}\), which is not contained in \(\mathbb{Q}\). \\

\textbf{3.2.3.b.} \(\mathbb{N}\) \\

\(\mathbb{N}\) is not open, as \(1\) has no \(\epsilon\)-neighborhoods entirely contained in \(\mathbb{N}\) centered at it. It is, however, closed, as its complement is open. \\

\textbf{3.2.3.c.} \(\{x \in \mathbb{R} : x > 0\}\) \\

This set is open (as its complement is closed), but not closed. The sequence \(1, 1/2, 1/3,...\) is entirely contained in this set, but converges outside of this set; therefore, \(0\) is a limit point not contained in the set. \\

\textbf{3.2.3.d.} \((0, 1] = \{x \in \mathbb{R} : 0 < x \leq 1\}\) \\

This set is neither open nor closed. \(1\) has no \(\epsilon\)-neighborhoods entirely contained in this set, and the sequence \(1, 1/2, 1/3, ...\) is entirely in this set, yet converges outside of it, so \(0\) is a limit point not in this set.

\textbf{3.2.3.e.} \(\{1 + 1/4 + 1/9 + ... + 1/n^{2} : n \in \mathbb{N}\}\) \\

This set is neither open nor closed, as there are no \(\epsilon\)-neighborhoods centered at \(1\) entirely contained in the set, and the number \(\frac{\pi^{2}}{6}\) is the limit of the sequence \(1, 1+1/4, 1+1/4+1/9, ...\), but is not contained in the set. \\



\end{document}