\documentclass[12pt,letterpaper]{article}
\usepackage[utf8]{inputenc}
\usepackage{amsmath}
\usepackage{amsfonts}
\usepackage{amssymb}
\usepackage{graphicx}
\usepackage{lastpage}
\usepackage[left=1.00in, right=1.00in, top=1.00in, bottom=1.00in]{geometry}
\usepackage{fancyhdr}
\title{Math 521 HW 9}
\author{Morgan Gribbins}
\date{}
\pagestyle{fancy}
\fancyhf{}
\lhead{Page \thepage\ of \pageref{LastPage}}
\begin{document}
	
\maketitle

\textbf{Exercise 2.7.1.} Proving the Alternating Series Test amounts to showing that the sequence of partial sums \[s_{n} = a_{1} - a_{2} + ... \pm a_{n}\] converges. \\

\textbf{2.7.1.a.} Prove the Alternating Series Test by showing that \((s_{n})\) is a Cauchy sequence. \\



\textbf{2.7.1.b.} Supply another proof for this result using the Nested Interval Property. \\



\textbf{2.7.1.c.} Consider the subsequences \((s_{2n})\) and \((s_{2n+1})\), and show how the Monotone Convergence Theorem leads to a third proof for the Alternating Series Test. \\



\textbf{Exercise 2.7.4.} Give an example to show that it is possible for both \(\sum x_{n}\) and \(\sum y_{n}\) to diverge but for \(\sum x_{n}y_{n}\) to converge. \\



\textbf{Exercise 3.2.3.} Decide whether the following sets are open, closed, or neither. If a set is not open, find a point in the set for which there is no \(\epsilon\)-neighborhood contained in the set. If a set is not closed, find a limit point that is not contained in the set. \\

\textbf{3.2.3.a.} \(\mathbb{Q}\) \\



\textbf{3.2.3.b.} \(\mathbb{N}\) \\



\textbf{3.2.3.c.} \(\{x \in \mathbb{R} : x > 0\}\) \\



\textbf{3.2.3.d.} \((0, 1] = \{x \in \mathbb{R} : 0 < x \leq 1\}\) \\



\textbf{3.2.3.e.} \(\{1 + 1/4 + 1/9 + ... + 1/n^{2} : n \in \mathbb{N}\}\) \\





\end{document}