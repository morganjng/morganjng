\documentclass[12pt,letterpaper]{article}
\usepackage[utf8]{inputenc}
\usepackage{amsmath}
\usepackage{amsfonts}
\usepackage{amssymb}
\usepackage{graphicx}
\usepackage{lastpage}
\usepackage[left=1.00in, right=1.00in, top=1.00in, bottom=1.00in]{geometry}
\usepackage{fancyhdr}
\title{Math 521 HW 11}
\author{Morgan Gribbins}
\date{}
\pagestyle{fancy}
\fancyhf{}
\lhead{Page \thepage\ of \pageref{LastPage}}
\begin{document}
	
\maketitle

\textbf{Exercise 3.3.9.} Follow these steps to prove the final implication in Theorem 3.3.8. \\

Assume \(K\) satisfies (i) and (ii), and let \(\{O_{\lambda} : \lambda \in \Lambda\}\) be an open cover for \(K\). For contradiction, let's assume that no finite subcover exists. Let \(I_{0}\) be a closed interval containing \(K\). \\

\textbf{3.3.9.a.} Show that there exists a nested sequence of closed intervals \(... \subseteq I_{2} \subseteq I_{1} \subseteq I_{0}\) with the property that, for each \(n\), \(I_{n} \cap K\) cannot be finitely covered and \(\lim |I_{n}| = 0.\) \\



\textbf{3.3.9.b.} Argue that there exists an \(x \in K\) such that \(x \in I_{n}\) for all \(n\). \\



\textbf{3.3.9.c.} Because \(x\in K\), there must exist an open set \(O_{\lambda_{0
}}\) from the original collection that contains \(x\) as an element. Explain how this leads to the desired contradiction. \\





\end{document}