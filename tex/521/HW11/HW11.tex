\documentclass[12pt,letterpaper]{article}
\usepackage[utf8]{inputenc}
\usepackage{amsmath}
\usepackage{amsfonts}
\usepackage{amssymb}
\usepackage{graphicx}
\usepackage{lastpage}
\usepackage[left=1.00in, right=1.00in, top=1.00in, bottom=1.00in]{geometry}
\usepackage{fancyhdr}
\title{Math 521 HW 11}
\author{Morgan Gribbins}
\date{}
\pagestyle{fancy}
\fancyhf{}
\lhead{Page \thepage\ of \pageref{LastPage}}
\begin{document}
	
\maketitle

\textbf{Exercise 3.3.9.} Follow these steps to prove the final implication in Theorem 3.3.8. \\

Assume \(K\) satisfies (i) and (ii), and let \(\{O_{\lambda} : \lambda \in \Lambda\}\) be an open cover for \(K\). For contradiction, let's assume that no finite subcover exists. Let \(I_{0}\) be a closed interval containing \(K\). \\

\textbf{3.3.9.a.} Show that there exists a nested sequence of closed intervals \(... \subseteq I_{2} \subseteq I_{1} \subseteq I_{0}\) with the property that, for each \(n\), \(I_{n} \cap K\) cannot be finitely covered and \(\lim |I_{n}| = 0.\) \\



\textbf{3.3.9.b.} Argue that there exists an \(x \in K\) such that \(x \in I_{n}\) for all \(n\). \\

By the Nested Interval Property, there must exists one \(x \in K\) such that \(x \in \bigcap_{n \in \mathbb{N}} I_{n}\), because of the established definitions of each \(I_{n}\).\\

\textbf{3.3.9.c.} Because \(x\in K\), there must exist an open set \(O_{\lambda_{0
}}\) from the original collection that contains \(x\) as an element. Explain how this leads to the desired contradiction. \\



\textbf{Exercise 4.2.2.} For each stated limit, find the largest possible \(\delta\)-neighborhood that is a proper response to the given \(\epsilon\) challenge. \\

\textbf{4.2.2.a.} \(\lim_{x \to 3} (5x - 6) = 9,\) where \(\epsilon = 1\). \\



\textbf{4.2.2.b.} \(\lim_{x \to 4} \sqrt{x} = 2,\) where \(\epsilon = 1\). \\



\textbf{4.2.2.c.} \(\lim_{x \to \pi} [[x]] = 3,\) where \(\epsilon = 1\). \\



\textbf{4.2.2.d.} \(\lim_{x \to \pi} [[x]] = 3,\) where \(\epsilon = .01\). \\



\textbf{Exercise 4.2.5.} Use Definition 4.2.1 to supply a proper proof for the following limit statements. \\

\textbf{4.2.5.a.} \(\lim_{x \to 2} (3x + 4) = 10\). \\



\textbf{4.2.5.b.} \(\lim_{x \to 0} x^{3} = 0\). \\



\textbf{4.2.5.c.} \(\lim_{x \to 2}(x^{2} + x - 1) = 5\). \\



\textbf{4.2.5.d.} \(\lim_{x \to 3} 1/x = 1/3\). \\



\textbf{Exercise 4.3.5.} Show using Definition 4.3.1 that if \(c\) is an isolated point of \(A \subseteq \mathbb{R}\), then \(f : A \to \mathbb{R}\) is continuous at \(c\). \\



\textbf{Exercise 4.3.6.} Provide an example of each or explain why the request is impossible. \\

\textbf{4.3.6.a.} Two functions \(f\) and \(g\), neither of which is continuous at \(0\) but such that \(f(x)g(x)\) and \(f(x)+g(x)\) are continuous at \(0\). \\



\textbf{4.3.6.b.} A function \(f(x)\) continuous at \(0\) and \(g(x)\) not continuous at \(0\) such that \(f(x) + g(x)\) is continuous at \(0\). \\



\textbf{4.3.6.c.} A function \(f(x)\) continuous at \(0\) and \(g(x)\) not continuous at \(0\) such that \(f(x)g(x)\) is continuous at \(0\). \\



\textbf{4.3.6.d.} A function \(f(x)\) not continuous at \(0\) such that \(f(x) + 1/f(x)\) is continuous at \(0\). \\



\textbf{4.3.6.e.} A function \(f(x)\) not continuous at \(0\) such that \([f(x)]^{3}\) is continuous at \(0\). \\



\textbf{Exercise 4.3.9.} Assume \(h : \mathbb{R} \to \mathbb{R}\) is continuous on \(\mathbb{R}\) and let \(K = \{x : h(x) = 0\}\). Show that \(K\) is a closed set. \\







\end{document}