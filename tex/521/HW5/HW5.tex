\documentclass[12pt,letterpaper]{article}
\usepackage[utf8]{inputenc}
\usepackage{amsmath}
\usepackage{amsfonts}
\usepackage{amssymb}
\usepackage{graphicx}
\usepackage[left=1.00in, right=1.00in, top=1.00in, bottom=1.00in]{geometry}
\usepackage{fancyhdr}
\title{Math 521 HW 5}
\author{Morgan Gribbins}
\date{}
\pagestyle{fancy}
\fancyhf{}
\lhead{Page \thepage\ of TOTAL}
\begin{document}
	
\maketitle

\textbf{Exercise 2.3.1.} Let \(x_{n} \geq 0\) for all \(n \in \mathbb{N}\). \\

\textbf{2.3.1.a.} If \((x_{n}) \to 0\), show that \( ( \sqrt{x_{n}} ) \to 0\). \\



\textbf{2.3.1.b.} If \((x_{n}) \to x\), show that \((\sqrt{x_{n}}) \to \sqrt{x}\). \\



\textbf{Exercise 2.3.2.} Using only Definition 2.2.3 (no Algebraic Limit Theorem), prove that if \((x_{n}) \to 2, \) then \\

\textbf{2.3.2.a.} \((\frac{2x_{n}-1}{3}) \to 1\); \\



\textbf{2.3.2.b.} \((1/x_{n}) \to 1/2\). \\



\textbf{Exercise 2.3.3 (Squeeze Theorem).} Show that if \(x_{n} \leq y_{n} \leq z_{n} \) for all \(n \in \mathbb{N}\), and if \(\text{lim } x_{n} = \text{lim }z_{n} = l\), then \(\text{lim } y_{n} = l\) as well. \\



\textbf{2.3.7.} Give an example of each of the following, or state that such a request is impossible by referencing the proper theorem(s): \\

\textbf{2.3.7.a.} sequences \((x_{n})\) and \((y_{n})\), which both diverge, but whose sum \((x_{n} + y_{n})\) converges; \\



\textbf{2.3.7.b.} sequences \((x_{n})\) and \((y_{n})\), where \((x_{n})\) converges, \((y_{n})\) diverges, and \((x_{n} + y_{n})\) converges; \\



\textbf{2.3.7.c.} a convergent sequence \((b_{n})\) with \(b_{n} \neq 0\) for all \(n\) such that \((1/b_{n})\) diverges; \\



\textbf{2.3.7.d.} an unbounded sequence \((a_{n})\) and a convergent sequence \((b_{n})\) with \((a_{n} - b_{n})\) bounded; \\



\textbf{2.3.7.e.} two sequences \((a_{n})\) and \((b_{n})\), where \((a_{n}b_{n})\) and \((a_{n})\) converge but \((b_{n})\) does not. \\



\textbf{Exercise 2.3.9.} \\

\textbf{2.3.9.a.} Let \((a_{n})\) be a bounded (not necessarily convergent) sequence, and assume \(\text{lim } b_{n} = 0\). Show that \(\text{lim }(a_{n}b_{n}) = 0.\) Why are we not allowed to use the Algebraic Limit Theorem to prove this? \\



\textbf{2.3.9.b.} Can we conclude anything about the convergence of \((a_{n}b_{n})\) if we assume that \((b_{n})\) converges to some nonzero limit \(b\)? \\



\textbf{2.3.9.c.} Use (a) to prove Theorem 2.3.3, part (iii), for the case when \(a = 0\). \\





\end{document}