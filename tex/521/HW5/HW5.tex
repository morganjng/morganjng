\documentclass[12pt,letterpaper]{article}
\usepackage[utf8]{inputenc}
\usepackage{amsmath}
\usepackage{amsfonts}
\usepackage{amssymb}
\usepackage{graphicx}
\usepackage[left=1.00in, right=1.00in, top=1.00in, bottom=1.00in]{geometry}
\usepackage{fancyhdr}
\title{Math 521 HW 5}
\author{Morgan Gribbins}
\date{}
\pagestyle{fancy}
\fancyhf{}
\lhead{Page \thepage\ of TOTAL}
\begin{document}
	
\maketitle

\textbf{Exercise 2.3.1.} Let \(x_{n} \geq 0\) for all \(n \in \mathbb{N}\). \\

\textbf{2.3.1.a.} If \((x_{n}) \to 0\), show that \( ( \sqrt{x_{n}} ) \to 0\). \\

Given arbitrary \(\epsilon > 0\), there exists some \(N_{\epsilon} \in \mathbb{N}\) such that all \(n \geq N_{\epsilon}\) satisfy \[|x_{n}| < \epsilon.\] Using \(\epsilon^{2}\) instead of \(\epsilon\), we have \[|x_{n} | < \epsilon^{2}.\] This implies \[|\sqrt{x_{n}}||\sqrt{x_{n}}| < \epsilon^{2} \implies |\sqrt{x_{n}}|^{2} < \epsilon^{2} \implies  |\sqrt{x_{n}}| < \epsilon.\] Therefore \((\sqrt{x_{n}})\to 0\). \\

\textbf{2.3.1.b.} If \((x_{n}) \to x\), show that \((\sqrt{x_{n}}) \to \sqrt{x}\). \\

Given arbitrary \(\epsilon > 0\), there exists some \(N_{\epsilon} \in \mathbb{N}\) such that all \(n \geq N_{\epsilon}\) satisfy \[|x_{n} - x| < \epsilon.\] Choosing \(\epsilon^{2}\) instead of \(\epsilon\), we have \[|x_{n} - x| < \epsilon^{2} \implies |\sqrt{x_{n}}-\sqrt{x}||\sqrt{x_{n}} + \sqrt{x}| < \epsilon^{2},\] and since \[|\sqrt{x_{n}} - \sqrt{x}| \leq |\sqrt{x_{n}} + \sqrt{x}|, \text{ then } |\sqrt{x_{n}} - \sqrt{x}|^{2} \leq |x_{n} - x| < \epsilon^{2}\] \[\implies |\sqrt{x_{n}} - \sqrt{x}|^{2} < \epsilon^{2} \implies |\sqrt{x_{n}} - \sqrt{x}| < \epsilon,\] so this sequence converges to \(\sqrt{x}\). \\

\textbf{Exercise 2.3.2.} Using only Definition 2.2.3 (no Algebraic Limit Theorem), prove that if \((x_{n}) \to 2, \) then \\

\textbf{2.3.2.a.} \((\frac{2x_{n}-1}{3}) \to 1\); \\

Due to the convergence of \((x_{n})\), we have for any arbitrary \(\epsilon > 0\), there exists \(N \in \mathbb{N}\) such that all \(n \geq N\) implies that \[|x_{n} - 2| < \epsilon.\] Choosing \(3\epsilon/2\) instead of \(\epsilon\), we have \[|x_{n} - 2| < 3\epsilon/2 \implies \left|\frac{2}{3}\right||x_{n} - 2| < \epsilon \implies \left|\frac{2}{3}x_{n} - \frac{4}{3}\right| < \epsilon \implies \left|\frac{2x_{n}-4}{3}\right| < \epsilon\] \[\implies \left|\frac{2x_{n}-1}{3} -1\right| < \epsilon,\] so \(\left(\frac{2x_{n}-1}{3} \right)\) converges to 1. \\

\textbf{2.3.2.b.} \((1/x_{n}) \to 1/2\). \\

Due to the convergence of \((x_{n})\), we have for any arbitrary \(\epsilon > 0\), there exists \(N \in \mathbb{N}\) such that all \(n \geq N\) implies that \[|x_{n} - 2| < \epsilon.\]

\textbf{Exercise 2.3.3 (Squeeze Theorem).} Show that if \(x_{n} \leq y_{n} \leq z_{n} \) for all \(n \in \mathbb{N}\), and if \(\text{lim } x_{n} = \text{lim }z_{n} = l\), then \(\text{lim } y_{n} = l\) as well. \\



\textbf{2.3.7.} Give an example of each of the following, or state that such a request is impossible by referencing the proper theorem(s): \\

\textbf{2.3.7.a.} sequences \((x_{n})\) and \((y_{n})\), which both diverge, but whose sum \((x_{n} + y_{n})\) converges; \\

The sequences \((x_{n}) = ((-1)^{n})\) and \((y_{n}) = ((-1)^{n+1})\) both diverge, yet their sum \(((-1)^{n} + (-1)^{n+1}) = (0) \to 0\). \\

\textbf{2.3.7.b.} sequences \((x_{n})\) and \((y_{n})\), where \((x_{n})\) converges, \((y_{n})\) diverges, and \((x_{n} + y_{n})\) converges; \\



\textbf{2.3.7.c.} a convergent sequence \((b_{n})\) with \(b_{n} \neq 0\) for all \(n\) such that \((1/b_{n})\) diverges; \\

This is impossible by the Algebraic Limit Theorem---the uniform sequence of 1 converges to 1 and the sequence \((b_{n})\) converges to some \(b \in \mathbb{R}\), so \((1/b_{n})\) must converge to \(1/b\), and as such cannot diverge. \\

\textbf{2.3.7.d.} an unbounded sequence \((a_{n})\) and a convergent sequence \((b_{n})\) with \((a_{n} - b_{n})\) bounded; \\



\textbf{2.3.7.e.} two sequences \((a_{n})\) and \((b_{n})\), where \((a_{n}b_{n})\) and \((a_{n})\) converge but \((b_{n})\) does not. \\

If \((a_{n})\) is uniformly \(0\) and \((b_{n})\) any divergent series, we have \((a_{n}b_{n})\) uniformly zero, which converges. \\

\textbf{Exercise 2.3.9.} \\

\textbf{2.3.9.a.} Let \((a_{n})\) be a bounded (not necessarily convergent) sequence, and assume \(\text{lim } b_{n} = 0\). Show that \(\text{lim }(a_{n}b_{n}) = 0.\) Why are we not allowed to use the Algebraic Limit Theorem to prove this? \\



\textbf{2.3.9.b.} Can we conclude anything about the convergence of \((a_{n}b_{n})\) if we assume that \((b_{n})\) converges to some nonzero limit \(b\)? \\



\textbf{2.3.9.c.} Use (a) to prove Theorem 2.3.3, part (iii), for the case when \(a = 0\). \\





\end{document}