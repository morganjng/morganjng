\documentclass[12pt,letterpaper]{article}
\usepackage[utf8]{inputenc}
\usepackage{amsmath}
\usepackage{amsfonts}
\usepackage{amssymb}
\usepackage{graphicx}
\usepackage[left=1.00in, right=1.00in, top=1.00in, bottom=1.00in]{geometry}
\usepackage{fancyhdr}
\usepackage{lastpage}
\title{Math 521 HW 5}
\author{Morgan Gribbins}
\date{}
\pagestyle{fancy}
\fancyhf{}
\lhead{Page \thepage\ of \pageref{LastPage}}
\begin{document}
	
\maketitle

\textbf{Exercise 2.3.1.} Let \(x_{n} \geq 0\) for all \(n \in \mathbb{N}\). \\

\textbf{2.3.1.a.} If \((x_{n}) \to 0\), show that \( ( \sqrt{x_{n}} ) \to 0\). \\

Given arbitrary \(\epsilon > 0\), let \(N_{\epsilon} \in \mathbb{N}\) such that all \(n \geq N_{\epsilon}\) satisfy \[|x_{n}| < \epsilon^{2}.\] This implies \[|\sqrt{x_{n}}||\sqrt{x_{n}}| < \epsilon^{2} \implies |\sqrt{x_{n}}|^{2} < \epsilon^{2} \implies  |\sqrt{x_{n}}| < \epsilon.\] Therefore \((\sqrt{x_{n}})\to 0\). \\

\textbf{2.3.1.b.} If \((x_{n}) \to x\), show that \((\sqrt{x_{n}}) \to \sqrt{x}\). \\

Given arbitrary \(\epsilon > 0\), let \(N_{\epsilon} \in \mathbb{N}\) such that all \(n \geq N_{\epsilon}\) satisfy \[|x_{n} - x| < \epsilon^{2} \implies |\sqrt{x_{n}}-\sqrt{x}||\sqrt{x_{n}} + \sqrt{x}| < \epsilon^{2},\] and since \[|\sqrt{x_{n}} - \sqrt{x}| \leq |\sqrt{x_{n}} + \sqrt{x}|, \text{ then } |\sqrt{x_{n}} - \sqrt{x}|^{2} \leq |x_{n} - x| < \epsilon^{2}\] \[\implies |\sqrt{x_{n}} - \sqrt{x}|^{2} < \epsilon^{2} \implies |\sqrt{x_{n}} - \sqrt{x}| < \epsilon,\] so this sequence converges to \(\sqrt{x}\). \\

\textbf{Exercise 2.3.2.} Using only Definition 2.2.3 (no Algebraic Limit Theorem), prove that if \((x_{n}) \to 2, \) then \\

\textbf{2.3.2.a.} \((\frac{2x_{n}-1}{3}) \to 1\); \\

Given \(\epsilon > 0\), let \(N \in \mathbb{N}\) such that all \(n \geq N\) implies that \[|x_{n} - 2| < 3\epsilon/2.\] Therefore, we have \[  \left|\frac{2}{3}\right||x_{n} - 2| < \epsilon \implies \left|\frac{2}{3}x_{n} - \frac{4}{3}\right| < \epsilon \implies \left|\frac{2x_{n}-4}{3}\right| < \epsilon\] \[\implies \left|\frac{2x_{n}-1}{3} -1\right| < \epsilon,\] so \(\left(\frac{2x_{n}-1}{3} \right)\) converges to 1. \\

\textbf{2.3.2.b.} \((1/x_{n}) \to 1/2\). \\

Since \((x_{n})\) is a convergent sequence, it is bounded---we can set \(M\) equal to the lowest real number such that \(x_{n} \leq M\), \(\forall n \in \mathbb{N}\). Given \(\epsilon > 0\), let \(N \in \mathbb{N}\) such that all \(n \geq N\) implies \[\left| x_{n} - 2\right| < 2|M|\epsilon.\] We then have \[\left|2x_{n}\right|\left| \frac{1}{x_{n}} - \frac{1}{n} \right| < 2|M|\epsilon,\] and because \(x_{n} \leq M\)  we have \[\left|2M\right| \left| \frac{1}{x_{n}} - \frac{1}{2} \right| < 2|M|\epsilon \implies \left| \frac{1}{x_{n}} - \frac{1}{2} \right| < \epsilon,\] so this sequence converges to 1/2. \\

\textbf{Exercise 2.3.3 (Squeeze Theorem).} Show that if \(x_{n} \leq y_{n} \leq z_{n} \) for all \(n \in \mathbb{N}\), and if \(\text{lim } x_{n} = \text{lim }z_{n} = l\), then \(\text{lim } y_{n} = l\) as well. \\

We will prove this by contradiction in two cases. Take \(x_{n} \leq y_{n} \leq z_{n}\) for all \(n \in \mathbb{N}\), \(\lim x_{n} = \lim z_{n} = l\), and let \(\lim y_{n} = k \neq l\). \\

First, we will prove that \(k\) cannot be less than \(l\), by contradiction. Assume that \(k > l\). Then, (by the Order Theorem) because both \(z_{n}\) and \(y_{n}\) are convergent sequences with \(y_{n} \leq z_{n}\) for all \(n\), \(k \leq l\), which contradicts our assumptions, so \(k\) cannot be less than \(l\). \\

Similarly, assume \(k < l\). This is contradicted by \(x_{n}\) and \(y_{n}\) converging and \(x_{n} \leq y_{n}\) for all \(n\), which implies \(l \leq k\). This is a contradiction, so \(k\) cannot be less than \(l\) either. \\

Therefore, \(k = l\). \\

\textbf{2.3.7.} Give an example of each of the following, or state that such a request is impossible by referencing the proper theorem(s): \\

\textbf{2.3.7.a.} sequences \((x_{n})\) and \((y_{n})\), which both diverge, but whose sum \((x_{n} + y_{n})\) converges; \\

The sequences \((x_{n}) = ((-1)^{n})\) and \((y_{n}) = ((-1)^{n+1})\) both diverge, yet their sum \(((-1)^{n} + (-1)^{n+1}) = (0) \to 0\). \\

\textbf{2.3.7.b.} sequences \((x_{n})\) and \((y_{n})\), where \((x_{n})\) converges, \((y_{n})\) diverges, and \((x_{n} + y_{n})\) converges; \\

This cannot converge by the Algebraic Limit Theorem. If \((x_{n} + y_{n})\) and \((x_{n})\) converge, then \((y_{n})\) must converge because \((x_{n} + y_{n} - x_{n}) = (y_{n})\) would be the sum of two convergent sequences, yet \((y_{n})\) is assumed to be divergent. \\ 

\textbf{2.3.7.c.} a convergent sequence \((b_{n})\) with \(b_{n} \neq 0\) for all \(n\) such that \((1/b_{n})\) diverges; \\

This is impossible by the Algebraic Limit Theorem---the uniform sequence of 1 converges to 1 and the sequence \((b_{n})\) converges to some \(b \in \mathbb{R}\), so \((1/b_{n})\) must converge to \(1/b\), and as such cannot diverge. \\

\textbf{2.3.7.d.} an unbounded sequence \((a_{n})\) and a convergent sequence \((b_{n})\) with \((a_{n} - b_{n})\) bounded; \\

This cannot be true because we have the difference between an unbounded and bounded set, which cannot be bounded. An unbounded set is naturally non-convergent, so this cannot converge. We have \(\forall M_{a} \in \mathbb{R}\), there is some \(a_{n}\) such that \(|a_{n}| > M_{a}\) and some \(M_{b}\) such that all \(b_{n}\) satisfy \(|b_{n}| \leq M_{b}\). We then have \(M_{a} - M_{b} < a_{n} - b_{n}\) because all real numbers have some \(a_{n}\) larger, and \(b_{n}\) is some finite number (so this difference is unbounded and as such divergent). \\

\textbf{2.3.7.e.} two sequences \((a_{n})\) and \((b_{n})\), where \((a_{n}b_{n})\) and \((a_{n})\) converge but \((b_{n})\) does not. \\

If \((a_{n})\) is uniformly \(0\) and \((b_{n})\) any divergent series, we have \((a_{n}b_{n})\) uniformly zero, which converges. \\

\textbf{Exercise 2.3.9.} \\

\textbf{2.3.9.a.} Let \((a_{n})\) be a bounded (not necessarily convergent) sequence, and assume \(\text{lim } b_{n} = 0\). Show that \(\text{lim }(a_{n}b_{n}) = 0.\) Why are we not allowed to use the Algebraic Limit Theorem to prove this? \\

Let \(M\) be the bound of \((a_{n})\), i.e., for all \(n\in \mathbb{N}\), \(|a_{n}| \leq M\). Given \(\epsilon > 0\), let \(N \in \mathbb{N}\) such that all \(n \geq N\) satisfies \[|b_{n}| < \epsilon/M.\] This implies \[M|b_{n}| < \epsilon \implies |b_{n}||a_{n}| < \epsilon \implies |a_{n}b_{n} - 0| < \epsilon,\] so this converges to \(0\). \\

We are not allowed to use the Algebraic Limit Theorem in this case because the Algebraic Limit Theorem requires convergent sequences for its conclusions, and \((a_{n})\) is not necessarily convergent. \\

\textbf{2.3.9.b.} Can we conclude anything about the convergence of \((a_{n}b_{n})\) if we assume that \((b_{n})\) converges to some nonzero limit \(b\)? \\

We can not conclude anything about the convergence of \((a_{n}b_{n})\) assuming the convergence of \((b_{n})\) to some \(b \neq 0\) if we do not know the convergence of \((a_{n})\). For instance, take \((a_{n}) = ((-1)^{n})\). For this, we have non-convergent \((a_{n}b_{n})\), but for some convergent \((a_{n})\), we have convergent \((a_{n}b_{n})\). \\

\textbf{2.3.9.c.} Use (a) to prove Theorem 2.3.3, part (iii), for the case when \(a = 0\). \\

Letting \((b_{n}) \to a = 0\), and \((a_{n})\) convergent (to b) and bounded by \(M\) (to satisfy the two convergent sequences required by hypothesis). Then, given \(\epsilon > 0\), let \(N \in \mathbb{N}\) such that all \(n \geq N\) satisfies \[|b_{n}| < \epsilon/M.\] This implies \[M|b_{n}| < \epsilon \implies |b_{n}||a_{n}| < \epsilon \implies |a_{n}b_{n} - 0| < \epsilon,\] so we have \(\lim (a_{n}b_{n}) = 0 = ab.\) \\



\end{document}