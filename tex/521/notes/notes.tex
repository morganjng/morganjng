\documentclass[12pt,letterpaper]{article}
\usepackage[utf8]{inputenc}
\usepackage{amsmath}
\usepackage{amsfonts}
\usepackage{amssymb}
\usepackage{graphicx}
\usepackage[left=1.00in, right=1.00in, top=1.00in, bottom=1.00in]{geometry}
\usepackage{fancyhdr}
\usepackage{lastpage}
\pagestyle{fancy}
\title{Analysis I}
\author{Morgan Gribbins}
\date{}
\fancyhf{}
\lhead{Page \thepage\ of \pageref{LastPage}}
\begin{document}
	
\maketitle
\tableofcontents
\pagebreak

\section{The Real Numbers}

\subsection{The Axiom of Completeness}

The \textbf{axiom of completeness} states that \textit{every nonempty set of real numbers that is bounded above has a least upper bound}. \\

A set \(A \subseteq \mathbb{R}\) is \textbf{bounded above} if there exists a number \(b \in \mathbb{R}\) such that \(\forall a \in A,\ a \leq b\). The number \(b\) is called an \textbf{upper bound} for \(A\). Similarly, the set \(A\) is \textbf{bounded below} if there exists a \textbf{lower bound} \(l \in \mathbb{R}\) satisfying \(\forall a \in A,\ l \leq a\). \\

A real number \(s\) is the \textbf{least upper bound} for a set \(A \subseteq \mathbb{R}\) if it meets the following two criteria:

\begin{enumerate}
	\item \(s\) is an upper bound for \(A\);
	\item if \(b\) is any upper bound for \(A\), then \(s \leq b\).
\end{enumerate}

The least upper bound is also called the \textbf{supremum} of the set \(A\), and is called \(s = \sup A\). The \textbf{greatest lower bound} is defined similarly, and is called the infimum, with \(l = \inf A\). Both suprema and infima are unique. A real number \(a_{0}\) is a \textbf{maximum} on a set \(A\) if \(a_{0}\) is an element of \(A\) and \(a_{0} = \sup A\). A \textbf{minimum}  is an element of \(A\) that is also the infimum of \(A\).  \\

Assuming \(s \in \mathbb{R}\) is an upper bound for a set \(A \subseteq \mathbb{R}\), then \(s = \sup A \iff \forall \epsilon > 0,\ \exists a \in A, \text{ such that } s-\epsilon < a\).

\subsection{Consequences of Completeness}

The \textbf{Archimedean property} states that:

\begin{enumerate}
	\item Given any number \(x \in \mathbb{R}\), there exists an \(n \in \mathbb{N}\) such that \(n > x\).
	\item Given any real number \(y > 0\), there exists an \(n \in \mathbb{N}\) such that \(1/n < y\).
\end{enumerate}

Additionally, for every two real number \(a\) and \(b\) with \(a < b\), there exists some \(q \in b\mathbb{Q}\) (or in the set of irrational numbers) such that \(a < r < b\).

\subsection{Cardinality}

A function \(f : A \to B\) is \textbf{one-to-one} (injective) if \(a_{1} \neq a_{2}\) in \(A\) implies that \(f(a_{1}) \neq f(a_{2})\) in \(B\). The function \(f\) is \textbf{onto} (surjective) if, given any \(b \in B\), there exist an \(a \in A\) such that \(f(a) = b\). If \(f\) is both onto and one-to-one, then it is called \textbf{bijective}. \\

The set \(A\) \textbf{has the same cardinality} as \(B\) if there exists a bijection \(f : A \to B\). In this case, we write \(A \sim B\). A set \(A\) is \textbf{countable} if \(\mathbb{N} \sim A\). An infinite set that is not countable is called an \textbf{uncountable} set. \(\mathbb{Q}\) is countable and \(\mathbb{R}\) is uncountable. Additionally, subsets of countable sets are countable and countable unions of countable sets are countable.

\pagebreak

\section{Sequences and Series}

\subsection{The Limit of a Sequence}

A \textbf{sequence} is a function whose domain is \(\mathbb{N}\). A sequence \((a_{n})\) \textbf{converges} to a real number \(a\) if, for every positive number \(\epsilon\), there exists an \(N \in \mathbb{N}\) such that whenever \(n \geq N\) it follows that \(|a_{n}-a| < \epsilon.\) This is notated as \(\lim a_{n} = a\). \\

Given a real number \(a \in \mathbb{R}\) and a positive number \(\epsilon > 0\), the set \[V_{\epsilon}(a) = \{x \in \mathbb{R} : |x-a| < \epsilon\}\] is called the \textbf{\(\epsilon\)-neighborhood of \(a\)}. \\

The limit of a sequence, when it exists, must be unique. \\

A sequence that does not converge is said to \textbf{diverge}. 

\subsection{The Algebraic and Order Limit Theorems}

A sequence \((x_{n})\) is \textbf{bounded} if there exists a number \(M > 0\) such that \(|x_{n}| \leq M\) for all \(n \in \mathbb{N}\). Every convergent sequence is bounded. \\

The Algebraic Limit Theorem states that for two convergent sequences with \(\lim a_{n} = a\) and \(\lim b_{n} = b\), 

\begin{enumerate}
	\item \(\lim ca_{n} = ca, \text{ for all } c \in \mathbb{R};\)
	\item \(\lim a_{n} + b_{n} = a + b;\)
	\item \(\lim a_{n}b_{n} = ab;\)
	\item \(\lim a_{n}/b_{n} = a/b, \text{ provided } b \neq 0.\)
\end{enumerate}

The Order Limit Theorem states that for two convergent sequences with \(\lim a_{n} = a\) and \(\lim b_{n} = b\),

\begin{enumerate}
	\item If \(a_{n} \geq 0\) for all \(n \in \mathbb{N}\), then \(a \geq 0\).
	\item If \(a_{n} \leq b_{n}\) for all \(n \in \mathbb{N}\), then \(a \leq b\).
	\item If there exists \(c \in \mathbb{R}\) for which \(c \leq b_{n}\) for all \(n \in \mathbb{N}\), then \(c \leq b\). Similarly, if \(a_{n}\leq c\) for all \(n \in \mathbb{N}\), then \(a \leq c\).
\end{enumerate}

\end{document}