\documentclass[12pt,letterpaper]{article}
\usepackage[utf8]{inputenc}
\usepackage{amsmath}
\usepackage{amsfonts}
\usepackage{amssymb}
\usepackage{graphicx}
\usepackage[left=1.00in, right=1.00in, top=1.00in, bottom=1.00in]{geometry}
\usepackage{fancyhdr}
\pagestyle{fancy}
\fancyhf{}
\title{Math 521 Homework 2}
\author{Morgan Gribbins}
\lhead{Page \thepage\ of 3}
\begin{document}
	
\maketitle

\section{1.3.1.}
	
\subsection{(a) Write a formal definition in the Style of Definition 1.3.2 for the infimum or greatest lower bound of a set.}
	
	A real number s is the \textit{greatest lower bound} for a set \(A \subseteq \mathbb{R}\) if it meets the following two criteria:
	
	\begin{enumerate}
		
		\item s is a lower bound for A
		
		\item if b is any lower bound for a, then \(b \leq s\).
		
	\end{enumerate}

\subsection{(b) Now, state and prove a version of Lemma 1.3.8 for greatest lower bounds}

	Lemma 1.3.8 states that for some \(s \in \mathbb{R}\), with s as an upper bound for some set \(A \subseteq \mathbb{R}\), \[s = sup\ A \iff \forall \epsilon > 0, \exists a \in A,\ s - \epsilon < a.\] A version of 1.3.8 for greatest lower bounds would state that for some lower bound for \(A \subseteq \mathbb{R}\), \(c \in \mathbb{R}\), \[c = inf\ A \iff \forall \epsilon > 0, \exists a \in A,\ c + \epsilon > a.\]
	
	Direct proof of \((\implies)\): \\
	
	Assume that \(c = inf\ A\), for some \(A \subseteq \mathbb{R}\). Therefore, c is the greatest lower bound of A, so any other lower bound of A is either lesser or equal to c. Now, for any \(\epsilon\), \(c + \epsilon\) must not be a lower bound of A, as \(c + \epsilon\) is strictly greater than c. From this, there must be some \(a \in A,\ c + \epsilon > a\), from the definition of a lower bound. \\
	
	Proof by contrapositive and contradiction of \((\impliedby)\): \\
	
	To begin this proof, we assume that for some \(c \in \mathbb{R}\) and for some fixed \(A \subseteq \mathbb{R}\), \[\forall \epsilon > 0, \exists a \in A,\ c + \epsilon > a.\]  We are given that c is a lower bound of A by hypothesis, so we must show that any other lower bounds of A are less than or equal to c. Any lower bound greater than c, s, can be expressed as \(s = c + \epsilon\), for some \(\epsilon > 0\), and by assumption, there must be some \(a \in A\) that is less than this other, greater, lower bound. Therefore, c must be the greatest lowest bound of A.
	
\section{1.3.2. Give an example of each of the following, or state that the request is impossible.}

\subsection{(a) A set B with inf B \(\geq\) sup B.}

	B = \(\{0\}\) has sup B = inf B = 0.
	
\subsection{(b) A finite set that contains its infimum but not its supremum.}

	This request is impossible, as a finite set must contain its supremum.
	
\subsection{(c) A bounded subset of \(\mathbb{Q}\) that contains its supremum, but not its infimum.}

	Let A = \(\{x \in \mathbb{Q} : -\sqrt{2} > x \geq 0\}\). This set is bounded above and below, and sup A = 0 \(\in\) A, while inf A = \(\sqrt{2}\ \notin\) A.
	
\section{1.3.5. As in Example 1.3.7, let \(A \subseteq \mathbb{R}\) be nonempty and bounded above, and let \(c \in \mathbb{R}\). This time define the set \(cA = \{ca : a \in A\}\).}

\subsection{(a) If \(c \geq 0\), show that sup(cA) = c sup(A).}

	First, let c = 0. Therefore, cA = {0}, and sup(cA) = 0 = 0 sup(A), for all A (bounded and nonempty). Now, we look at the case when c $>$ 0. Let sup(A) = a, which is the lowest number which is greater than or equal to all elements of A. Formally, \(\forall b \in A,\ b \leq a\). When this inequality is multiplied by c (as is allowed for positive c), we get \(\forall b \in A,\ cb \leq ca\), which implies \(\forall d \in cA,\ d \leq ca\), by definition of cA, implicating that sup(cA) = c sup(A).

\subsection{(b) Postulate a similar type of statement for sup(cA) for the case c \(<\) 0.}

	Postulate: for c \(<\) 0, sup(cA) = c inf(-A). Observe that sup(-A) = -inf(A), so by the proof in (a), sup(cA) = c inf(-A) from some working and substitutions.
	
\section{1.3.8. Compute, without proofs, the suprema and infima (if they exist) of the following sets:}
 
\subsection{(a) \(\{m/n : m, n \in \mathbb{N}, m < n\}\).}

	The supremum of this set is 1, and the infimum of this set is 0.
	
\subsection{(b) \(\{(-1)^{m}/n : m,n \in \mathbb{N}\}\).}

	The supremum of this set is 1, and the infimum of this set is -1.
	
\subsection{\(\{n/(3n+1) : n \in \mathbb{N}\}\).}

	The supremum of this set is \(1/3\), and the infimum of this set is \(1/4\).
	
\subsection{\(\{m/(m+n) : m,n \in \mathbb{N}\}\).}

	The supremum of this set is 1, and the infimum of the set is 0.
	
\section{1.3.9.}

\subsection{(a) If sup A \(<\) sup B, show that there exists an element \(b \in B\) that is an upper bound for A.}

	Let x = sup A, and let y = sup B. By definition of supremum, we have x is greater or equal to all elements of A, and y is greater of equal to all elements of B. The assertion that there is an upper bound for A in B (called b) states that said element is greater than or equal to all elements in A. Let us assume that there is no element \(b \in B\), which is an upper bound for A. This means that \(\forall b \in B, \exists a \in A,\ b < a\). By Lemma 1.3.8, we have \(\forall \epsilon > 0, \exists c \in A,\ x - \epsilon < c\) and \(\forall \epsilon > 0, \exists d \in B,\ y - \epsilon < d\). Choosing the same \(\epsilon\) for both of these we have existing elements c in A and d in B that satisfy \(x - \epsilon > c\) and \(y - \epsilon > d\), which implies \(x > c + \epsilon\) and \(y > d + \epsilon\), and because \(x > y\), we have \(c > d\), which shows that the prior assumptions lead to the existence of an upper bound b \(\in\) A.
	
\subsection{(b) Give an example to show that this is not almays the case if we only assume sup A \(\leq\) sub B.}

	Let A = B = \(\{x \in \mathbb{R} : 0 \leq x < 1\}\) have sup A \(\leq\) sup B, but there is no upper bound for A contained in B.


\end{document}