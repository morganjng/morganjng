\documentclass[12pt,letterpaper]{article}
\usepackage[utf8]{inputenc}
\usepackage{amsmath}
\usepackage{amsfonts}
\usepackage{amssymb}
\usepackage{graphicx}
\usepackage{lastpage}
\usepackage[left=1.00in, right=1.00in, top=1.00in, bottom=1.00in]{geometry}
\usepackage{fancyhdr}
\title{Math 521 HW 6}
\author{Morgan Gribbins}
\date{}
\pagestyle{fancy}
\fancyhf{}
\lhead{Page \thepage\ of \pageref{LastPage}}
\begin{document}
	
\maketitle

\textbf{Exercise 2.4.1} \\

\textbf{2.4.1.a.} Prove that the sequence defined by \(x_{1} = 3\) and \[x_{n+1} = \frac{1}{4-x_{n}}\] converges. \\

By the Monotonous Convergence Theorem, we must prove that this sequence is monotone and bounded, in order to show that it converges. \\

Proof that \((x_{n})\) is bounded above. \(x_{1} \leq 3\), and we want to show that \(x_{n} \leq 3 \implies x_{n+1} \leq 3\). \[x_{n} \leq 3 \implies -x_{n} \geq -3 \implies 4-x_{n} \geq 1 \implies \frac{1}{4-x_{n}} =x_{n+1} \leq 1.\] This proves that \(x_{n}\) is bounded above by \(3\). \\

Proof that \((x_{n})\) is bounded below. \(x_{1} \geq 0\). We want to show that \(x_{n} \geq 0 \implies x_{n+1} \geq 0\). \[x_{n} \geq 0 \implies -x_{n} \leq 0 \implies 4-x_{n} \leq 4 \implies \frac{1}{4-x_{n}} = x_{n+1} \geq 1/4 \geq 0.\] This proves that \(x_{n}\) is bounded below. \\

Proof that \((x_{n})\) is monotone. Note that \(x_{1} = 3 \geq 1 = x_{2}\). To inductively prove this sequence is monotonous for all \(n \in \mathbb{N}\), we must now show that \(x_{n-1} \geq x_{n} \implies x_{n} \geq x_{n+1}\). Assuming \(x_{n-1} \geq x_{n}\), we want to show \(x_{n} - x_{n+1} \geq 0\), which is the same as \(\frac{1}{4-x_{n-1}} - \frac{1}{4-x_{n}} \geq 0\). Combining these fractions gives us \[\frac{x_{n-1}-x_{n}}{(4-x_{n-1})(4-x_{n})}.\] As \(x_{n-1} - x_{n} \geq 0\) by hypothesis, and all \(x_{n}\) are between (inclusive) 3 and 0, the numerator and denominator are both positive, so this holds and \((x_{n})\) is necessarily monotone. \\

Because \((x_{n})\) is monotone (decreasing) and bounded, it converges to its greatest lower bound. \\

\textbf{2.4.1.b.} Now that we know that \(\lim x_{n}\) exists, explain why \(\lim x_{n+1}\) must also exist and equal the same value. \\

\(\lim x_{n}\) and \(\lim x_{n+1}\) must exists and be equal by the definition of convergence. If \((x_{n})\) converges, then given \(\epsilon > 0\), there is some \(N \in \mathbb{N}\) such that all \(n \geq N\) have \[|x_{n} - x| < \epsilon.\] Because for all \(n \geq N\), we have \(n+1 \geq N\), then \[|x_{n+1} - x| < \epsilon,\] so they converge to the same value. \\

\textbf{2.4.1.c.} Take the limit of each side of the recursive equation in part (a) to explicitly compute \(\lim x_{n}\). \\

\[x_{n+1} = \frac{1}{4-x_{n}} \implies \lim x_{n+1} = \frac{1}{4-\lim x_{n}} \implies x = \frac{1}{4-x}\]\[\implies x(4-x) = 1 \implies x^{2} - 4x + 1 = 0 \implies x = 2 \pm \sqrt{3},\] and because this sequence converges to its lower bound, \(\lim x_{n} = x = 2 - \sqrt{3}\). \\

\textbf{Exercise 2.4.2.} \\

\textbf{2.4.2.a.} Consider the recursively defined equation \(y_{1} = 1,\) \[y_{n+1} = 3-y_{n},\] and set \(y = \lim y_{n}\). Because \((y_{n})\) and \((y_{n+1})\) have the same limit, taking the limit across the recursive equation gives \(y = 3 - y.\) Solving for \(y\), we conclude \(\lim y_{n} = 3/2\). \\
What is wrong with this argument? \\

This argument is wrong because \((y_{n})\) is not convergent, as it is unbounded; therefore, we cannot manipulate equations involving the sequence like this. \\

\textbf{2.4.2.b.} This time set \(y_{1} = 1\) and \(y_{n+1} = 3 - 1/y_{n}\). Can the strategy in (a) be applied to compute the limit of this sequence? \\

The strategy in (a) can be applied to compute the limit of this sequence, as this sequence is bounded an monotone, and therefore convergent. This means that we can use algebraic methods on \(y_{n}\) in this case. \\

\textbf{Exercise 2.4.5. (Calculating Square Roots)} Let \(x_{1} = 2\) and define \[x_{n+1} = \frac{1}{2}\left(x_{n} + \frac{2}{x_{n}}\right).\] \\

\textbf{2.4.5.a.} Show that \(x_{n}^{2}\) is always greater than or equal to \(2\), and then use this to prove that \(x_{n} - x_{n+1} \geq 0\). Conclude that \(\lim x_{n} = \sqrt{2}\). \\

Proof that \(x_{n}^{2}\geq 2\). Note that \(x_{1}^{2} = 4 \geq 2\). We would like to prove this inductively by showing that \(x_{n}^{2} \geq 2 \implies x_{n+1}^{2} \geq 2\). Because \(x_{n}^{2} \geq 2\), \(4/x_{n}^{2} \leq 2\). Therefore, \[x_{n}^{2} \geq 2 \implies x_{n}^{2} + 4/x_{n}^{2} \geq 4 \implies x_{n}^{2} + 4 + 4/x_{n}^{2} \geq 8 \] \[\implies \frac{1}{4}\left(x_{n}^{2} + 4 + 4/x_{n}^{2}\right) = x_{n+1}^{2} \geq 2.\] 

Proof that \(x_{n} - x_{n+1} \geq 0\). We have \(x_{n}^{2} \geq 2 \implies \frac{3}{4}x_{n}^{2} \geq 3/2 \implies \frac{3}{4}x_{n}^{2} - 1 \geq 1/2 \implies \frac{3}{4}x_{n}^{2} - 1 - 1/x_{n}^{2} \geq 0\). The right side of this final inequality is equal to \(x_{n}^{2} - x_{n+1}^{2} \geq 0 \implies (x_{n}-x_{n+1})(x_{n}+x_{n+1}) \geq 0 \implies x_{n}-x_{n+1} \geq 0.\) \\

Therefore, \(\lim x_{n}^{2} = 4 \implies \lim x_{n} = \sqrt{2}\), as this sequence converges to its greatest lower bound as it is bounded and monotonously decreasing.\\

\textbf{2.4.5.b.} Modify the sequence \((x_{n})\) so that it converges to \(\sqrt{c}\). \\

Let \(x_{1} = c\) and define \[x_{n+1} = \frac{1}{2}\left(x_{n}  + \frac{c}{x_{n}}\right).\] \\

Proof that \(x_{n}^{2}\geq c\). Note that \(x_{1}^{2} = c^{2} \geq c\). We would like to prove this inductively by showing that \(x_{n}^{2} \geq c \implies x_{n+1}^{2} \geq c\). Because \(x_{n}^{2} \geq c\), \(c^{2}/x_{n}^{2} \leq c\). Therefore, \[x_{n}^{2} \geq c \implies x_{n}^{2} + c^{2}/x_{n}^{2} \geq 2c \implies x_{n}^{2} + 2c + c^{2}/x_{n}^{2} \geq 4c \] \[\implies \frac{1}{4}\left(x_{n}^{2} + 2c + c^{2}/x_{n}^{2}\right) = x_{n+1}^{2} \geq c.\] 

Proof that \(x_{n} - x_{n+1} \geq 0\). We have \(x_{n}^{2} \geq c \implies \frac{3}{4}x_{n}^{2} \geq 3c/4 \implies \frac{3}{4}x_{n}^{2} - c/2 \geq c/4 \implies \frac{3}{4}x_{n}^{2} - c/2 - c^{2}/4x_{n}^{2} \geq 0\). The right side of this final inequality is equal to \(x_{n}^{2} - x_{n+1}^{2} \geq 0 \implies (x_{n}-x_{n+1})(x_{n}+x_{n+1}) \geq 0 \implies x_{n}-x_{n+1} \geq 0.\) \\

Therefore, \(\lim x_{n}^{2} = c^{2} \implies \lim x_{n} = \sqrt{c}\), as this sequence converges to its greatest lower bound as it is bounded and monotonously decreasing. \\

\end{document}