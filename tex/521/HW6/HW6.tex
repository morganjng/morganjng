\documentclass[12pt,letterpaper]{article}
\usepackage[utf8]{inputenc}
\usepackage{amsmath}
\usepackage{amsfonts}
\usepackage{amssymb}
\usepackage{graphicx}
\usepackage{lastpage}
\usepackage[left=1.00in, right=1.00in, top=1.00in, bottom=1.00in]{geometry}
\usepackage{fancyhdr}
\title{Math 521 HW 6}
\author{Morgan Gribbins}
\date{}
\pagestyle{fancy}
\fancyhf{}
\lhead{Page \thepage\ of \pageref{LastPage}}
\begin{document}
	
\maketitle

\textbf{Exercise 2.4.1} \\

\textbf{2.4.1.a.} Prove that the sequence defined by \(x_{1} = 3\) and \[x_{n+1} = \frac{1}{4-x_{n}}\] converges. \\



\textbf{2.4.1.b.} Now that we know that \(\lim x_{n}\) exists, explain why \(\lim x_{n+1}\) must also exist and equal the same value. \\



\textbf{2.4.1.c.} Take the limit of each side of the recursive equation in part (a) to explicitly compute \(\lim x_{n}\). \\



\textbf{Exercise 2.4.2.} \\

\textbf{2.4.2.a.} Consider the recursively defined equation \(y_{1} = 1,\) \[y_{n+1} = 3-y_{n},\] and set \(y = \lim y_{n}\). Because \((y_{n})\) and \((y_{n+1})\) have the same limit, taking the limit across the recursive equation gives \(y = 3 - y.\) Solving for \(y\), we conclude \(\lim y_{n} = 3/2\). \\
What is wrong with this argument? \\



\textbf{2.4.2.b.} This time set \(y_{1} = 1\) and \(y_{n+1} = 3 - 1/y_{n}\). Can the strategy in (a) be applied to compute the limit of this sequence? \\



\textbf{Exercise 2.4.5. (Calculating Square Roots)} Let \(x_{1} = 2\) and define \[x_{n+1} = \frac{1}{2}\left(x_{n} + \frac{2}{x_{n}}\right).\] \\

\textbf{2.4.5.a.} Show that \(x_{n}^{2}\) is always greater than or equal to \(2\), and then use this to prove that \(x_{n} - x_{n+1} \geq 0\). Conclude that \(\lim x_{n} = \sqrt{2}\). \\



\textbf{2.4.5.b.} Modify the sequence \((x_{n})\) so that it converges to \(\sqrt{c}\). \\






\end{document}