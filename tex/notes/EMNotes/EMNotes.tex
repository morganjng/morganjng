\documentclass[12pt,letterpaper]{article}
\usepackage[utf8]{inputenc}
\usepackage{amsmath}
\usepackage{amssymb}
\usepackage{amsfonts}
\usepackage{amsthm}
\usepackage[hidelinks]{hyperref}
\usepackage{xcolor}
\usepackage{amssymb}
\usepackage{graphicx}
\newtheorem{theorem}{Theorem}
\newcommand{\ihat}{\hat{\i}}
\newcommand{\jhat}{\hat{\j}}
\newcommand{\khat}{\hat{k}}
\usepackage[left=2cm,right=2cm,top=2cm,bottom=2cm]{geometry}
\author{Jonathan Gribbins}
\title{Electricity \& Magnetism Notes}
\date{}

\begin{document}

\maketitle

\tableofcontents

\pagebreak

\section{Electric Charges and Forces}

\subsection{Charge}

The modern day terms used to describe charges are \textbf{positive} and \textbf{negative} charges. These charges were originally discovered and described by Benjamin Franklin---he described a positively charged object as objects repelled by a glass rod that has been rubbed with silk, and he described negatively charged objects as objects attracted by a glass rod rubbed with silk. Now we know that these negative charged objects have an \textbf{increased amount of electrons}, and positive charged objects have a \textbf{decreased amount of electrons}. \\

The fundamental unit of charge is denoted by $e$, and this is the magnitude of the charge of a proton or electron. An electron has a charge of $-e$, and a proton has a charge of $+e$. The charge of an object is generally denoted by the letter $q$, and it is $$q = N_{p}e - N_{e}e = \left(N_{p} - N_{e}\right)e,$$ where $N_{p}$ is the number of protons in the object and $N_{e}$ is the number of electrons. A number with an equal amount of protons and electrons is \textit{electrically neutral} (this means there is no \textit{net charge} in the object, not that there is no charge). \\

The charge of an object does not change by losing or gaining protons, but \textbf{by losing or gaining electrons.} This is because---comparatively to electrons---protons are incredibly difficult to remove from the nucleus of an atom. This process of removing an electron from an atom is called \textbf{ionization}, and an atom that has more or less electrons than it has protons is called an \textit{ion}. \\

The \textbf{law of conservation of charge} states that charge is \textit{neither created nor destroyed,} only transferred. In a transfer of charge, the charge lost by one object is gained by another, and vice versa. 

\subsection{Insulators and Conductors}

There are two classes of objects based on whether they interact with charged objects or not: \textbf{conductors} are affected by the charge of charged objects, and they \textit{``conduct" this charge}, while \textbf{insulators} remain unaffected and \textit{do not ``conduct" charge}. Objects that are conductive are said to have a \textbf{sea of electrons} surrounding positively charged \textbf{ion cores}, which allows electrons to move freely between the atoms of the object. The motion of charges are called \textbf{current}, and the particles that move are called \textbf{charge carriers}. \\

If charges are at rest in an object, they are said to be in \textbf{electrostatic equilibrium}. As a consequence of this, \textit{an isolated charged object has all of its excess charge on its surface.} \\

When a charged object's surface is touched by an uncharged object, the object \textbf{discharges} to the uncharged object. An object is considered \textbf{grounded} if it is connected via a conductor to the Earth or another large body---when grounded, an object's charge is absorbed entirely by the large body. \\

\textbf{Charge polarization} is the slight separation of charges in a neutral object, generally as a result of a charged object approaching the neutral object and exerting a force on the charge. Because of how electrostatic force decreases with distance, the force repelling the far side of the polarized object is less than the force pulling the near side; therefore a polarized object has a net force in the direction of the polarizing object. \\

An \textbf{electric dipole} occurs when a slight separation forms between two differently charged particles is formed. The net force between these two particles is such that they are drawn together. 

\subsection{Coulomb's Law}

The force law that describes the attraction and repulsion between charges is called \textbf{Coulomb's Law}. The force between two objects $a$ and $b$ with charge $q_{a}$ and $q_{b}$ is given by $$F_{a \rightarrow b} = F_{b \rightarrow a} = \frac{k |q_{a}| |q_{b}|}{r^{2}},$$ where r is the distance between the two particles and $k$ is the \textbf{electrostatic constant} (more on that later). This force (like all forces) is a vector, and it is pointed in the direction of the other object if the charges are different signs, and opposite of the other object if the charges are the same sign. \\

When doing his research of electrostatics, Coulomb went without a unit of measurement of charge---so he created his own, the \textbf{coulomb (C)}. An electron's charge $e$ is about $$e = 1.60 \times 10^{-19} C,$$ and the value of $k$ (in this formula) is $$k = 8.99 \times 10^{9} N m^{2} C^{-2} \approx 9.0 \times 10^{9} N m^{2} C^{-2}.$$ This law is not used too often in electrostatics, as things like \textit{fields and potentials} are more important to the study. The \textbf{permittivity constant}, $\epsilon_{0}$ (pronounced ``epsilon zero") is given by $$\epsilon_{0} = \frac{1}{4\pi k} = 8.85 \times 10^{-12} C^{2} N^{-1} m^{-2},$$ and with this constant we can redefine electrostatic force between charged objects $a$ and $b$ as $$F = \frac{1}{4\pi\epsilon_{0}} \frac{|q_{a}| |q_{b}|}{r^{2}}.$$

\subsection{The Electric Field}

The idea for fields to describe long-range forces like gravity and electromagnetism came from physicists wondering about how objects interacted with these long-range forces based on time---should a particle instantly react to a shift in a particle light years away? To describe this, Michael Faraday proposed the field. A \textbf{field}, in physics, is a function that assigns a vector to every point in space. So, an object's field affects every point at space, and other objects react to how the field affects their position. The alteration of space because of mass is called a \textit{gravitational field}, and the alteration because of electricity and charge is called an \textbf{electric field}. \\

\textbf{The electric field is the agent that exerts an electric force on and from a charged particle}. \textit{Charged particles interact via the electrical field}. The electric field at a point $(x, y, z)$ 

\end{document}