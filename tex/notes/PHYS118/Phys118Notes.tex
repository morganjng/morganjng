\documentclass[12pt,letterpaper]{article}
\usepackage[utf8]{inputenc}
\usepackage{amsmath}
\usepackage{amsfonts}
\usepackage[hidelinks]{hyperref}
\usepackage{xcolor}
\usepackage{amssymb}
\usepackage{graphicx}
\newcommand{\ihat}{\hat{\i}}
\newcommand{\jhat}{\hat{\j}}
\newcommand{\khat}{\hat{k}}
\usepackage[left=2cm,right=2cm,top=2cm,bottom=2cm]{geometry}
\author{Jonathan Gribbins}
\title{PHYS 118 Notes}
\date{}

\begin{document}

\maketitle

\tableofcontents

\pagebreak

\section[Concepts of Motion]{Concepts of Motion}

\subsection[Motion Diagrams]{Motion Diagrams}

\textbf{Motion} is the change of an object's position with time. There are many different types of motion, but two of the most important ones (in this book) are \textbf{translational motion}, where an object moves through space along a \textbf{trajectory}, and \textbf{rotational motion}, where an object changes it's orientation in space. \\

To visualize the motion of an object, it is helpful to draw a motion diagram. A motion diagram is a drawing of the object through time---for example, a motion diagram for an object moving along a straight line may show the object's position at every second or $\frac{1}{2}$ second. The amount of change in the distances between the object with time can be used to tell how it is moving---it's speed, acceleration, etc.

\subsection[Models and Modeling]{Models and Modeling}

To simplify some of the complicated situations that are presented in physics, it is important to build \textbf{models}. Models are simplified pictures of reality that can be used to understand the ``bigger picture". There are two different types of models---\textbf{descriptive models}, which display the essentials of a phenomenon in the simplest form, and \textbf{explanatory models}, which predict how certain minutiae of the universe can effect how things act. \\

A very important type of model is a \textbf{particle model}, which simplifies the shapes and physical characteristics of an object down to that of a particle (these characteristics are generally insignificant enough to disregard). The particle which represents an object should be located at it's real center of mass. \\

However, some things require these minutiae, so particle models are not always useful. For example, a gear cannot be represented by a particle, so it's model must be more in-depth.

\subsection[Position, Time, and Displacement]{Position, Time, and Displacement}

For a motion diagram, you need to know an object's \textbf{position} (where it is) at certain \textbf{times}. To describe position, the Cartesian (xy) plane is generally used to describe how different an objects position is from a set origin (0, 0) point. To incorporate time into this plane, we can assign various points time, or t, values by marking the times (relative to a chosen starting or 0 time) when a particle is at a certain position. \\

Another way to represent position is through \textbf{vectors}. Vectors are arrows pointing from the origin of a plane to the position of an object at a certain time. A vector (notated $\vec{r}$) can be in the form of an (x, y) coordinate or in the form of a (r, $\theta$) coordinate, where r is the distance from the origin and $\theta$ is the angle between the origin and the vector ``drawing". \\

If a value has a magnitude (size) and a direction, then it may be described by a vector. For instance, position, velocity, and acceleration can all be described with vectors. However, if a value has no direction, it must be described by a \textbf{scalar}, which is a magnitude without a direction. A good example of a scalar quantity is mass, as it has no direction. \\

The \textbf{displacement} of an object in motion is the change in its position between two different points in time. Displacement is a vector, and if the two  position vectors which provide this change are $\vec{r_{1}}$ and $\vec{r_{2}}$, then this displacement vector is notated $\Delta\vec{r}$, and it is the vector that solves the equation $\vec{r_{2}} = \vec{r_{1}} + \Delta\vec{r}$. \\

It is also important to define a \textbf{time interval}, or a change in time with the motion of an object. This time interval is notated as $\Delta t$, and given a final time ($t_{f}$) and an initial time ($t_{i}$), $\Delta t=t_{f}-t_{i}$.

\subsection[Velocity]{Velocity}

The \textbf{average speed} (fastness) over a time period is given by the total distance over a period of time, divided by that given time interval. Mathematically, 

\begin{center}
$$average\ speed=\frac{distance\ travelled}{time\ interval\ spent\ travelling}=\frac{d}{\Delta t}.$$\linebreak
\end{center}

This quantity is a scalar, and as such, has no direction. In order to get a vector quantity (which is necessary for more advanced physics stuff), we use the displacement vector, $\Delta\vec{r}$ and its corresponding time interval to calculate the average $velocity$ of an object over a time interval. The formula for this is

\begin{center}
$$v_{avg}=\frac{\Delta\vec{r}}{\Delta t}.$$\linebreak
\end{center}

This vector is in the same direction as the displacement vector, and this direction shows how the object is moving over that time interval (on average). The length of the velocity vector is the speed at which the object is moving over the given time interval (longer $\rightarrow$ faster).

\subsection[Linear Acceleration]{Linear Acceleration}

\textbf{Acceleration} is the vector quantity that describes the change in the velocity of an object. As the velocity vector has two components (magnitude and direction), an acceleration vector can describe the change of one or both of these components. The \textbf{average acceleration} for a time interval is given by the ratio between the change in velocity for this time interval and the time interval. Symbolically,

\begin{center}
$$\vec{a_{avg}}=\frac{\vec{\Delta v}}{\Delta t}.$$\linebreak
\end{center}

This vector points with the vector $\vec{\Delta v}$, as on average, the velocity changes in that direction over its time interval. \\

A complete motion diagram utilizes all three of these vector quantities---position, velocity, and acceleration.

\subsection[The Rest of Chapter 1]{The Rest of Chapter 1}

The rest of chapter 1 just details motion in one dimension, solving problems with physics, units, and significant figures. These are generally self-explanatory, and any further explanation will be left up to the reader reading the book.

\pagebreak

\section[Kinematics in One Dimension]{Kinematics in One Dimension}

\textbf{Kinematics} is the mathematical description of motion. Kinematics in \textbf{one dimension} is the description of motion along a straight line.

\subsection[Uniform Motion]{Uniform Motion}

\textbf{Uniform motion} is motion with constant speed. An object has uniform motion if and only if its position-versus-time graph is a straight line. The average velocity of a particle is equal to the slope of its position-versus-time graph. Or, if the y-axis is position and the x-axis is time,

\begin{center}
$$v_{avg}=\frac{\Delta x}{\Delta t}.$$\linebreak
\end{center}

If an object, $s$, is moving with uniform motion with starting point $s_{i}$, final point $s_{f}$, and time interval $\Delta t$, then 

\begin{center}
$$v_{s}=\frac{\Delta s}{\Delta t}=\frac{s_{f}-s_{i}}{\Delta t},$$ which can be rearranged to get $$s_{f}=s_{i}+v_{s}\Delta t.$$\linebreak
\end{center}

So, in words, the final position of a uniformly moving object is equal to its starting position plus the product of the time it travels and its velocity. \\

\textbf{Speed} is always positive (because your fastness can't be less than 0), so the speed of an object is the absolute value of the velocity of the object.

\subsection[Instantaneous Velocity]{Instantaneous Velocity}

The \textbf{instantaneous velocity} of an object is its velocity at a certain instant of time---that is, with an infinitesimally small time interval. As the velocity of an object is the slope of a secant line connecting two different points of time, the instantaneous velocity of an object is the tangent line to a curve at an instant. If the position of an object is given by the vector $s$ (and $s$ varies directly with $t$), then this velocity is

\begin{center}
$$\lim_{t\to0}\frac{\Delta s}{\Delta t} = \frac{ds}{dt},$$\linebreak
\end{center}

or in other words,\textbf{the velocity of a particle is the derivative of its position with respect to time.} The exact way to calculate this derivative will be left up to any calculus class.

\subsection[Finding Position from Velocity]{Finding Position from Velocity}

To find the position of an object based on the function of its velocity, we will use one of the facts of calculus that states that the original function of a derivative is the integral of its derivative. That isn't really a succinct way of stating that, so I'll state it symbolically:

\begin{center}
$$\int_{}^{}(df/dt)dt  = f(t)$$$$\int_{}^{}f'(t)dt = f(t)$$ or $$\int v(t)dt = s(t).$$\linebreak
\end{center}

Just like with taking the derivative, integration will be left up to a calculus class (or a different set of notes) to explain. An important thing to note with this is that \textbf{the definite integral from times $t_{i}$ to $t_{f}$ is not equal to the position at $t_{f}$, but the change in position between the two times}. To find the position at $t_{f}$ instead of the change between $t_{i}$ and $t_{f}$, we would add the initial position:

\begin{center}
$$s_{f} = s_{i} + \int_{t_{i}}^{t_{f}}v(t)dt$$\linebreak
\end{center}

\subsection[Motion with Constant Acceleration]{Motion with Constant Acceleration}

If an object is moving with constant acceleration (as objects often do), the average slope of the velocity graph is equal to the acceleration at all points. The velocity graph will be a straight line with:

\begin{center}
$$a_{avg} = a = \frac{v_{f}-v_{i}}{\Delta t} = \frac{\Delta v}{\Delta t} = \frac{dv}{dt},$$\linebreak
\end{center}

pointed in the direction of vector $\Delta v$. From this equation, we can derive a couple other equations---

\begin{center}
$$v_{f} = v_{i} + a\Delta t,$$ $$v_{f}^{2} = v_{i}^{2} + 2a\Delta s$$ and $$s_{f} = \frac{1}{2}a(\Delta t)^{2} + v\Delta t + s_{i}.$$\linebreak
\end{center}

These functions follow directly from the fact that acceleration is constant and that velocity is the derivative with respect to time of position and acceleration is velocity's derivative with respect to time. I wouldn't recommend memorizing these because with a little bit of calculus, these all boil down to the fact that $a(t)$ doesn't vary with time.\\

\textbf{A very important fact is that the instantaneous acceleration of an object is ALWAYS the derivative with respect to time of velocity. Constant acceleration is NOT a special case!}

\subsection[Free Fall]{Free Fall}

When an object is moving under the influence of only one force, gravity, it is said to be in \textbf{free fall}. In reality, free fall doesn't happen too often, but we will assume for now that air resistance is not a thing that happens (which is sort of true for heavy things). A \textit{very} important fact of free fall is that any two objects being effected by the same source of gravity will accelerate towards that source at the same rate. For Earth (where we live), gravity is a constant rate of $9.81$ $m/s^{2}$ towards the center of the Earth. We denote the \textbf{gravitational acceleration} for objects drawn to a body by the letter $g$. It is important to know that $g$ is not a vector, but $a_{free\ fall}$ is, and most of the time in free fall, $a_{free\ fall}$ has magnitude $g$.

\subsection[Motion on an Inclined Plane]{Motion on an Inclined Plane}

We haven't even mentioned friction yet, so this chapter will assume an object is sliding (or rolling) down a friction-free inclined plane of angle $\theta$ (the angle connecting the ground and the inclined plane). The acceleration parallel to the surface of the inclined plane is equal to 

\begin{center}
$$a_{s} = gsin\theta.$$ \linebreak
\end{center}

There is also a vector with magnitude equal to $g - a_{s}$ that runs perpendicular to the surface of the inclined plane, but this is unimportant as of right now. \\

From this equation for the acceleration down a friction-free inclined plane and some elementary calculus, it is simple to find any other kinematic-y stuff dealing with inclined planes.

\pagebreak

\section[Vectors and Coordinate Systems]{Vectors and Coordinate Systems}

This chapter is a short and more in-depth overview of vectors, scalars, and coordinate systems. It isn't really too important to read because it's stuff you generally know from existing in science classes for a while. However it is important to know what values are vectors and what values are scalars, so that will be included in this chapter.

\subsection[Scalars and Vectors]{Scalars and Vectors}

A \textbf{scalar} is a quantity that can be described by only one number. To denote a scalar quantity, we use symbols without little arrows above them, like $m$ for mass, $T$ for temperature, or $\rho$ for density. \\

A \textbf{vector} is a quantity that has to have a magnitude and a direction for it to be describable. Vectors are usually depicted with arrows in the direction of their direction and with length equal to their magnitude. This magnitude is always a positive scalar, and can be used in scalar calculations because of this fact. \\

A vector is depicted as a letter with a little arrow above it (like $\vec{r}$), and its magnitude is depicted as the same letter without the little arrow (like $r$). \\

\subsection[Using Vectors]{Using Vectors}

\textbf{Two vectors are equal if they have the same magnitude and direction.} That's kind of self-explanatory. \\

Vectors can be added and subtracted to other vectors, or they can be multiplied or divided by scalars. To add together two vectors graphically, you would place the tail end of one vector on the tip of the other vector and draw a line connecting the tip of the connected vector to the origin (to subtract you would add but with the subtracted vector in the opposite direction). However, this isn't really an exact method. To be more precise, trigonometry is used to describe the \textbf{resultant vector.} \\

For two vectors $\vec{A}$ and $\vec{B}$, the resultant vector $\vec{C} = \vec{A} + \vec{B}$ has a magnitude of $C = \sqrt{A^{2} + B^{2}}$. The direction of the resultant vector will vary based on the other vectors, but the angle between $\vec{A}$ and $\vec{B}$ is given by $\theta = tan^{-1}(\frac{B}{A})$. \\

Scalar multiplication and division is very simple---if a vector $\vec{A}$ is multiplied by scalar $k$, then the resultant vector has magnitude $kA$ and the same direction as $\vec{A}$.

\subsection[Coordinate Systems and Vector Components]{Coordinate Systems and Vector Components}

Coordinate systems are not required to use vectors, but they come in handy to visually grasp a situation or graphically solve a problem. The most common coordinate systems used in physics are the xy (or xyz)-plane and the polar plane. The xy(z)-plane is the Cartesian plane, and the polar plane has coordinates defined by their radius and the angle between the x-axis and the line connecting the point to the origin. \\

Vectors can be broken down into their \textbf{components}, which are generally scalar multiples of the unit vectors parallel to the axes of the Cartesian plane. In simpler terms, the x, y, or z coordinates of the vector's endpoint (if it is placed with its end on $(0,\ 0,\ 0)$) in vector form. The original vector is the sum of these components:

\begin{center}
	$$\vec{A} = \vec{A_{x}} + \vec{A_{y}} + \vec{A_{z}}$$ \linebreak
\end{center}

Operation upon vectors when you have their components is usually much simpler than operations upon the original vector. \\

Decomposition of polar vectors is possible with some simple trig, which the reader should be able to do by now.

\subsection[Unit Vector and Vector Algebra]{Unit Vector and Vector Algebra}

A \textbf{unit vector} is the vector that runs parallel or along one axis and has magnitude of 1. These unit vectors are useful in notation for the components of a vector---a vector's components can be given by its magnitude in each direction times the unit vector for that direction. Therefore, if a vector $\vec{A}$ has a magnitude of $A_{x}$ is the x direction (the unit vector of which is usually denoted $\ihat$), a magnitude in the y direction (denoted $\jhat$) of $A_{y}$, and maybe a magnitude in the z direction ($\khat$) of $A_{z}$, then

\begin{center}
	$$\vec{A_{x}} = A_{x}\ihat,$$
	$$\vec{A_{y}} = A_{y}\jhat,\ and$$
	$$\vec{A_{z}} = A_{z}\khat.\ Therefore,$$
	$$\vec{A} = A_{x}\ihat + A_{y}\jhat + A_{z}\khat.$$. \linebreak
\end{center}

Breaking the vectors you encounter in physics down into their component parts simplifies many of the issues that may arise in calculations, while also making vector quantities and operations feel more familiar to a student aware of the Cartesian plane.

\pagebreak

\section[Kinematics in Two Dimensions]{Kinematics in Two Dimensions}

\subsection[Motion in Two Dimensions]{Motion in Two Dimensions}

To define the motion of an object in the xy-plane (which can be done for most 2D motion), we use a vector $\vec{r}$ and its parameterization ($\vec{r} = x\ihat + y\jhat$) ot define its position, and a vector parameterization of the function for its position as $\vec{r}(t) = x(t)\ihat + y(t)\jhat$. \\

As with motion in one dimension, the velocity of an object is given by the derivative with respect to time of the position, and with the basic rule of differentiation that states that the derivative of the sum of functions is equal to the sum of their derivatives. So we arrive at

\begin{center}
	$$\vec{v} = \frac{dx}{dt}\ihat + \frac{dy}{dt}\jhat = v_{x}\ihat + v_{y}\jhat.$$ \linebreak
\end{center}

Also, the total instantaneous speed at a given time is given by

\begin{center}
	$$v = \sqrt{v_{x}^{2} + v_{y}^{2}},$$ \linebreak
\end{center}

and this is tangent to the y-versus-x graph at the given time. Similarly, acceleration is given by the derivative of the velocity vector function with respect to time:

\begin{center}
	$$\vec{a} = \frac{dv_{x}}{dt}\ihat + \frac{dv_{y}}{dt}\jhat = a_{x}\ihat + a_{y}\jhat.$$ The total acceleration (tangent to the velocity graph) is
	$$a = \sqrt{a_{x}^{2} + a_{y}^{2}}.$$ \linebreak
\end{center}

The other axioms of 1D motion holds with 2D motion, but with extra components.

\subsection[Projectile Motion]{Projectile Motion}

A \textbf{projectile} is an object whose motion is only affected by the influence of gravity. A projectile is said to be launched at the start of its motion with a particular vector $\vec{v_{0}}$ comprised of \textbf{launch velocity} and \textbf{launch angle}. From this vector, its x and y components can be derived, which are

\begin{center}
	$$v_{0y} = v_{0}sin\theta$$ and
	$$v_{0x} = v_{0}cos\theta.$$ \linebreak
\end{center}

Assuming there is no air resistance (because we don't know how to do that yet) and that the only force acting upon the particle is acceleration, the x and y components come out to be $0$ and $-g$, respectively. It is important to note that \textbf{the x and y components of this motion are independent of each other} (and so are many other things!!!). \\

There are a whole lot of equations in this section that talk more about projectile motion and how it behaves, but there really is no reason to memorize them when their derivations are possible with this small amount of information (and some calculus). \\

\subsection[Relative Motion]{Relative Motion}

In this chapter, the motion of an object relative to another (moving or non-moving) object is introduced. We call these objects which measure the motion of others relative to themselves \textbf{reference frames}, and we usually denote these by capital letters (like reference frame $A$, or $B$, or $C$), and we denote a value of some reference frame $A$ relative to another frame $B$ by a subscript $AB$. For instance, the velocity of $A$ relative to $B$ is denoted $v_{AB}$. \\

In one dimensional motion, the values for reference frames add together and ``cancel out" inside subscripts. So, if you want to find the velocity of $C$ relative to $B$ and you had the velocity of $C$ relative to $A$ and $A$ relative to $B$, you would add these values and their inner subscripts would ``cancel":

\begin{center}
	$$v_{CB} = v_{CA} + v_{AB}.$$\linebreak
\end{center}

For an example that everyone would be familiar with, if a train was moving at a velocity of $100\ km/hour$ and someone inside this train was walking forward (relative to the train) at $6\ km/hour$, then this passenger's total motion to a stationary outside observer would be $100\ km/hour + 6\ km/hour = 106\ km/hour$. \\

In two dimensions, the vectors for position for different frames can be added in the same way. So (with the same reference frames):

\begin{center}
	$$\vec{r_{CB}} = \vec{r_{CA}} + \vec{r_{AB}},$$
	and with some calculus,
	$$\frac{d\vec{r_{CB}}}{dt} = \frac{d\vec{r_{CA}}}{dt} + \frac{d\vec{r_{AB}}}{dt} = \vec{v_{CB}} = \vec{v_{CA}} + \vec{v_{AB}};$$
	$$\frac{d^{2}\vec{r_{CB}}}{dt^{2}} = \frac{d^{2}\vec{r_{CA}}}{dt^{2}} + \frac{d^{2}\vec{r_{AB}}}{dt^{2}} = \vec{a_{CB}} = \vec{a_{CA}} + \vec{a_{AB}}.$$\linebreak
\end{center}

This is called the \textbf{Galilean transform of motion}, and can be used to \textbf{transform} the velocity or other motion measurements of one reference frame into another reference frame (which are basically coordinate planes originated at an object, moving or stationary).

\subsection[Uniform Circular Motion]{Uniform Circular Motion}

Uniform circular motion is used to describe the movement of a particle around a circle of radius $r$. Because this motion is \textbf{uniform}, the velocity of the particle at all times will be of equal magnitude, but different direction (but always tangent to the circle). The magnitude of this particle is equal to the circumference of the circle divided by the time ($T$) that it takes to go around this circle. That is---

\begin{center}
	$$v = \frac{2\pi r}{T}.$$\linebreak
\end{center}

To describe the position of a particle moving in uniform circular motion, we use the particle's distance from the center of the circle---the radius---and the angle between the position of the particle and the center of the circle. The distance spanned by a particle over a certain angle of the circle is given by

\begin{center}
	$$s = r\theta,\ or\ \theta = \frac{s}{r}.$$\linebreak
\end{center}

The \textbf{angular velocity} of a particle in uniform circular motion is the rate at which the angular displacement of a particle is changing with time---instantaneous angular velocity is given by the derivative of angular movement with respect to time:

\begin{center}
	$$\omega = \lim_{\Delta t\to0}\frac{\Delta \theta}{\Delta t} = \frac{d\theta}{dt},$$\linebreak
\end{center}

and a particle is under uniform circular motion if and only if $\omega$ is a constant value. \\

The velocity of uniform circular motion is always tangent to the circle, and as such can be called the \textit{tangential velocity} of a particle---this velocity is directly related to the radius and the angular velocity of a particle moving uniformly:

\begin{center}
	$$v_{t} = \omega r;$$\linebreak
\end{center}

also, the sign of this velocity is always positive if the motion is counter-clockwise, and always negative if the motion is clockwise. 

\subsection[Centripetal Acceleration]{Centripetal Acceleration}

When an object is under uniform circular motion, their velocity may not be changing in magnitude, but it is changing in direction---therefore it has non-zero acceleration. This \textbf{centripetal acceleration} pulls the velocity of the particle \textit{towards the center of its circle}. This vector has a magnitude of 

\begin{center}
	$$a_{c} = \frac{v^{2}}{r} = \omega^{2} r.$$\linebreak
\end{center}

The uniform circular motion model is especially important not only because it describes this type of notion, but also because it can be applied to describe other motion such as the rotation of an object.

\subsection[Nonuniform Circular Motion]{Nonuniform Circular Motion}

Circular motion with changing speed is \textbf{nonuniform circular motion}. This motion is defined by the changing angular velocity---like linear motion, where $a = dv/dt$, \textbf{angular acceleration} is given by

\begin{center}
	$$\alpha = \frac{d\omega}{dt},$$\linebreak
\end{center}

which is the rate at which the angle of a an object changes in angle units per time unit. When $a$ and $v$ point in the same direction, the object is speeding up (in regular motion)---likewise, when $\alpha$ and $\omega$ are in the same direction, the object is speeding up; when they are in opposite direction, the object is slowing down. \\

The kinematics of circular motion are for all intents and purposes identical to linear motion, but with angles instead of positions. \\

In addition to the \textbf{radial acceleration} which draws a circular motion particle towards the center of its path along the radius, nonuniform motion has a \textbf{tangential acceleration} which accounts for the nonuniform nature of its velocity---this runs tangent to the circle and ``pulls" the velocity to change. Since acceleration is the rate of change of velocity, tangential acceleration is the rate of change of the tangential velocity---

\begin{center}
	$$a_{t} = \frac{dv_{t}}{dt} = \frac{d(\omega r)}{dt},$$
	and with constant radius (always true for circles!):
	$$a_{t} = \frac{d\omega}{dt}r = \alpha r.$$\linebreak
\end{center}

\pagebreak

\section[Force and Motion]{Force and Motion}

\subsection[Force]{Force}

\textbf{Force} is the fundamental concept of \textbf{mechanics}. A force is a \textit{push or pull} that an \textit{agent} enacts upon an object. A force \textbf{is a vector}, and there are two different types of forces---\textbf{contact forces} and \textbf{long range forces}. Contact forces occur when two objects touch, and long range forces occur without this contact. \\

To draw a force acting upon an object, construct the object as a particle and place the force vector with its tail coming from the particle (with length and direction proportional to the actual force's). \\

The \textbf{net force} acting upon an object is the sum of all of the forces acting upon that object. For $N$ forces,

\begin{center}
	$$\vec{F}_{net} = \sum_{i = 1}^{N} \vec{F}_{i} = \vec{F}_{1} + \vec{F}_{2} + ... + \vec{F}_{N}.$$\linebreak
\end{center}

\subsection[A Short Catalog of Forces]{A Short Catalog of Forces}

\subsubsection{Gravity} 

Gravity is a long range force that draws two bodies together---the larger and closer the bodies, the bigger the force. All objects in ``gravitational range" of a body is affected by gravity---whether the bodies affected are at rest or not. The vector for gravity \textbf{points towards the center of the objects exerting the gravitational force}.

\subsubsection{Spring Force} 

When a spring is contacting another object, it exerts spring force of that object. If the spring is compressed, it pushes against the object, and when stretched it pulls the object. The tail of a spring force vector is placed on the diagram particle irregardless of the nature of its force. Also, not all springs are the metal coils everyone is familiar with. Anything that compresses or stretches and ``springs" back into its original shape is a spring.

\subsubsection{Tension Force}

When a rope, spring, or something else pulls on an object, it exerts a contact force called tension ($\vec{T}$). This vector (with its tail on the particle) follows along the direction of the string, spring, etc. The use of the ball-and-spring model to describe molecular forces is an application of the tension force to understand how the universe works at a microscopic level.

\subsubsection{Normal Force} 

Normal force is the force exerted upon an object that is contacting another object, perpendicular to the object being contacted. This normal force is equal to the force being put upon the other object, and it is denoted by the symbol $\vec{n}$. This force is called \textit{normal} force because it is perpendicular to a surface, not because it isn't abnormal.

\subsubsection{Friction} 

Friction is another force exerted by a surface on an object. However, unlike normal force, this force runs parallel to the surface and opposite of the direction of the object's motion (or the direction that the object ``would" move). This force during motion is called \textit{kinetic friction}, or $\vec{f}_{k}$, and at rest is called \textit{static friction}, or $\vec{f}_{s}$. Kinetic friction fights against motion, and static friction prevents motion.

\subsubsection{Drag} 

When an object moves through a fluid (a gas or liquid !I think!), it experiences a \textit{resistive force} called drag. Drag, like friction, moves against the motion of the object, and the magnitude of this force is increased with a greater area covered by the object, less weight, higher speed, and other factors.

\subsubsection{Thrust} 

Thrust is the force that propels a body forward through the expulsion of other objects. This is the force that propels rockets and other ships during their launch. 

\subsubsection{Electricity and Magnetism} 

Electric and magnetic forces are long range forces, like gravity. These forces are what pull different molecules together (not little springs).

\subsection[What Do Forces Do?]{What Do Forces Do?}

\textbf{An object affected by a constant net force moves with a constant acceleration}. By definition, a force is something that causes an object to accelerate, or to accelerate in a different way. \textbf{The acceleration of an object is directly proportional to the force exerted on it.} Additionally, the acceleration is \textbf{inversely proportional to the mass of an object}---therefore,

\begin{center}
	$$a = \frac{F}{m}.$$\linebreak
\end{center}

The basic unit of force is the \textbf{Newton}, which is equal to $1 kg \times 1 \frac{m}{s^{2}} = 1 \frac{kg \times m}{s^{2}}$. \\

Mass is the amount of matter that something contains. \textbf{Inertia} is the tendency of an object to resist change in motion, and the $m$ in the force equation ($F = ma$) is \textbf{inertial mass}, which causes this resistance to change.

\subsection[Newton's Second Law]{Newton's Second Law}

\textbf{Newton's second law} states that the acceleration vector $\vec{a}$ of an object undergoing a force will be pointed in the same direction of the net force vector ($\vec{F}_{net}$) divided by the mass of the object. Symbolically,

\begin{center}
	$$\vec{a} = \frac{\vec{F}_{net}}{m}.$$\linebreak
\end{center}

In any given instance, an object only reacts to the forces acting upon at that time. Additionally, according to \textbf{Newton's third law,} every action (or force) has an equal and opposite reaction. If something pushes or pulls on an object, that object pulls or pushes back with an equal magnitude force in the opposite direction. \\

When solving problems using this law, it is usually helpful to set up the same amount of equations as dimensions of your forces and find the net force for each dimension. With two dimensions, that would look like

\begin{center}
	$$(\vec{F}_{net})_{x} = \sum_{i = 1}^{N}(\vec{F}_{i})_{x} = m\vec{a}_{x}$$
	and
	$$(\vec{F}_{net})_{y} = \sum_{i = 1}^{N}(\vec{F}_{i})_{y} = m\vec{a}_{y}.$$
\end{center}

\subsection[Newton's First Law]{Newton's First Law}

\textbf{Newton's first law} is the proposition of Newton that stated that an object in rest tends to stay at rest, and an object in motion will continue to move as long as there are no forces working upon it. This is also called the \textit{law of inertia}. An object in the state where no forces are acting upon it is said to be in \textbf{mechanical equilibrium}. \\

Based on Newton's first law, \textit{an object's resting state is uniform motion.} A force is the only thing that is responsible for any changes in motion that an object experiences. \\

An \textbf{inertial reference frame} is a reference frame in which the law of inertia holds---that is, if $\vec{F}_{net} = 0$, then $\vec{a} = 0$. A reference frame is not \textit{inertial} if that reference frame is accelerating, and reference frames that are not \textit{inertial} \textbf{are not subject to the Newtonian laws.} The Earth is technically not an inertial reference frame, but its acceleration has so little effect on physical measurements that we can treat it as an inertial reference frame. \\

\pagebreak

\section[Dynamics I: Motion Along a Line]{Dynamics I: Motion Along a Line}








\end{document}