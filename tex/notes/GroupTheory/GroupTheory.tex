\documentclass[12pt,letterpaper]{article}
\usepackage[utf8]{inputenc}
\usepackage{amsmath}
\usepackage{amssymb}
\usepackage{amsfonts}
\usepackage{amsthm}
\usepackage[hidelinks]{hyperref}
\usepackage{xcolor}
\usepackage{amssymb}
\usepackage{graphicx}
\newtheorem{theorem}{Theorem}
\newcommand{\ihat}{\hat{\i}}
\newcommand{\jhat}{\hat{\j}}
\newcommand{\khat}{\hat{k}}
\usepackage[left=2cm,right=2cm,top=2cm,bottom=2cm]{geometry}
\author{Jonathan Gribbins}
\title{Group Theory}
\date{}

\begin{document}

\maketitle

\tableofcontents

\pagebreak

\section{Introduction to Groups}

\subsection{Basic Axioms}

A group is one of the fundamental algebraic objects studied in abstract algebra. Groups are sets coupled with a binary operation in the ordered pair $(G, \star)$, where $\star$ is a \textbf{binary operation}---

\begin{description}
	\item[(1)] A \textit{binary operation} is a function on a set $G$ mapping $G \times G$ to $G$: $\star: G \times G \rightarrow G$. This operation upon an ordered pair in $G$ is denoted $\star (a, b)$ for $a, b \in G$.
	\item[(2)] A binary operation is \textit{associative} if for all $a, b, c \in G$, $a \star (b \star c) = (a \star b) \star c$.
	\item[(3)] Two elements $a, b \in G$ \textit{commute} if $a \star b = b \star a$. A binary operation is \textit{commutative} if $\forall a, b \in G, a \star b = b \star a$.
\end{description}

A \textbf{group} is an ordered pair of a set and a binary operation upon this set $(G, \star)$ such that three axioms are fulfilled:

\begin{description}
	\item[(1)] $G$ is \textbf{associative}, so for all $a, b, c \in G$, $a \star (b \star c) = (a \star b) \star c$.
	\item[(2)] There is some element $e \in G$ such that for all elements $a \in G$, $a \star e = a$.
	\item[(3)] For all $a \in G$, there is some element $a^{-1} \in G$ such that $a \star a^{-1} = e$.
\end{description}

A group $G$ is called \textbf{abelian} or \textbf{commutative} if for all $a, b \in G$, $a \star b = b \star a$. \\

It can be shown that for any group $G$ under binary operation $\star$, 

\begin{description}
	\item[(1)] The identity ($e$) of $G$ is unique.
	\item[(2)] For each $a \in G$, $a^{-1}$ is unique.
	\item[(3)] $(a^{-1})^{-1} = a$ for all $a \in G$.
	\item[(4)] $(a \star b)^{-1} = (b)^{-1} \star (a)^{-1}$.
	\item[(5)] For any $a_{1}, a_{2}, a_{3}, ... a_{n} \in G$, the value of $a_{1} \star a_{2} \star a_{3} \star ... \star a_{n}$ does not vary based on parentheses or brackets.
\end{description}

Because of \textbf{(5)}, for any element $a \in G$, the product of $n \in Z^{+}$ $a$s ($a \star a \star a ... (n times)$) can be denoted $a^{n}$. Additionally, if we let $a$ be the inverse $x^{-1}$ of an element $x \in G$, we would denote the nth product of $x^{-1}$ as $x^{-n}$. The identity of a group $G$ can be denoted $a^{0}$ for all $a \in G$. 

\begin{description}
	\item[\boldmath Order of an element $x \in G$:] The order of an element $x \in G$ is the \textit{smallest positive integer} $n$ such that $x^{n} = 1$ (where is the identity of $G$). This integer is also denoted $|x|$. If there is no integer $n$ such that $x^{n} = 1$, $x$ is said to be of infinite order.
\end{description}

\begin{description}
	\item[\boldmath Cayley table of group $G$] The \textit{Cayley, multiplication, or group table} of a finite group $G = \{g_{1}, g_{2}, g_{3}, ... g_{n}\}$ is an $n \times n$ table where the entry at location $(i, j)$ in the table is equal to $g_{i}g_{j}$.
\end{description}

\subsection{Dihedral Groups}

\textbf{Dihedral groups} are groups that describe the symmetries of simple planar polygons. For all $n \in Z^{+}$ with $n \geq 3$, the \textit{dihedral group} $D_{2n}$ is the group that describes its symmetries. A \textbf{symmetry} of an n-gon is a \textit{rigid motion on the n-gon that leaves the n-gon in the same ``orientation" (non-pointwise) of the original n-gon}. \\

Formally, these symmetries are described as permutations upon the vertices of the n-gon, described by the set $\{1, 2, 3, ..., n\}$. A symmetry $s$ that moves the vertex $i$ from its original position to the position of (arbitrary) vertex $j$, then the \textit{permutation} $\sigma$ sends $i$ to $j$, and moves the rest of the vertices with the same permutation. \\

For instance, if $s$ is the symmetry describing the rotation of an n-gon by $2\pi/n$ radians, then the permutation $\sigma$ sends each element $i \in \{1, 2, 3, ..., n\}; i > n$ to $i + 1$, and sends $n$ to $1$. \\

For $D_{2n}$ and any symmetries $s, t \in D_{2n}$, $st$ is the symmetry resulting after applying $t$ then $s$ to the n-gon. Symmetries on a n-gon are functions on the n-gon, so the combination of these symmetries is just function composition---as a result, they are inherently associative. The inverse of a symmetry in $D_{2n}$ is the symmetry that ``reverses" the actions that the original symmetry wrought upon the n-gon. \\

$$|D_{2n}| = 2n,$$ so the group $D_{2n}$ is generally called the \textit{dihedral group of order 2n}. If $r$ describes a rotation of an n-gon one-$n^{th}$ way around the n-gon, and $s$ describes a flip about a line bisecting the first vertex, the some rules of these elements of $D_{2n}$ are

\begin{description}
	\item[(1)] $1, r, r^{2}, ..., r^{n-1}$ are all distinct and $r^{n} = 1$, so $|r| = n$.
	\item[(2)] $|s| = 2$.  
	\item[(3)] For any $i$, $s \neq r^{i}$.
	\item[(4)] $sr^{i} \neq sr^{j}$ for any $0 \geq i, j \geq n-1; i \neq j$, so $$D_{2n} = \{1, r, r^{2}, ..., r^{n-1}, s, sr, sr^{2}, ..., sr^{n-1}\}.$$ Each element in $D_{2n}$ can be \textit{uniquely expressed by $s^{k}r^{i}$} with $k = 0 or 1$ and $0 \geq i \geq n-1$.
	\item[(5)] $r^{i}s = sr^{-i}$ for all $0 \geq i \geq n$.
\end{description}

Based on these rules, any element of $D_{2n}$ can be expressed in terms of $r$ and $s$ only---because of this fact, we call $r$ and $s$ \textbf{generators} of the group $D_{2n}$. Formally speaking, a subset $S \subseteq G$ with the property that \textit{every element of $G$ can be written as a finite product of elements of $S$ and their inverses} is said to be a \textbf{generator} of $G$, and \textbf{generates} $G$. For instance, the set $\{1\}$ generates the set $Z$ of all integers because all integers can be expressed as a finite sum of $+1$s and $-1$s. \\

The set $S$ that generates a group $G$ can be notated by $G=\langle S\rangle$, which is read \textit{$G$ is the set generated by $S$}. For $D_{2n}$, the set $S = \{r, s\}$ generates $D_{2n}$, so $$D_{2n} = \langle S \rangle = \langle \{r, s\} \rangle.$$ Any equation satisfied by the generators in a group are called \textbf{relations}. For $D_{2n}$, $S = \{r, s\}$, $r^{n} = 1$, $s^{2} = 1$, and $rs = sr^{-1}$ are the relations of $D_{2n}$. \\

If a group is generated by a set $S$ and some relations in that set, $R_{1}, R_{2}, R_{3}, ..., R_{n}$, then the group can be shown as a \textbf{presentation} of $S$ and the relations as $$G = \langle S | R_{1}, R_{2}, R_{3}, ..., R_{n} \rangle.$$ For $D_{2n}$, one presentation is $$D_{2n} = \langle r, s | r^{n} = 1, s^{2} = 1, rs = sr^{-1} \rangle.$$ These presentations are very useful for determining certain properties of a group, as they can usually be applied to find implicit relations that are not outright stated.

\subsection{Symmetry Groups}

Let $\Omega$ be some non-empty set and let $S_{\Omega}$ be the set of bijections mapping from $\Omega \rightarrow \Omega$. $S_{\Omega}$ is a group under function composition, as for some permutations $\sigma, \tau \in S_{\Omega}$, $\sigma: \Omega \rightarrow \Omega$ and $\tau: \Omega \rightarrow \Omega$, so $\sigma \circ \tau: \Omega \rightarrow \Omega$ and  $\tau \circ \sigma: \Omega \rightarrow \Omega$ by the rules of composition of bijections. For any permutation $\sigma \in S_{\Omega}$, there is some $\sigma^{-1} \in S_{\Omega}$ such that $\sigma \circ \sigma^{-1} = \sigma^{-1} \circ \sigma = 1$, where $1$ is the identity permutation where $\forall a \in \Omega, 1(a) = a$. \\

This group $S_{\Omega}$ is called the \textit{symmetric group of set $\Omega$}. When $\Omega = \{1, 2, ..., n\}$, then $S_{\Omega}$ is called $S_{n}$ or \textit{the symmetric group of order n}. The actual order of $S_{n}$ is $n!$, but the behaviors of \textit{any symmetric group is based upon the order of the group it operates on} so we call it order $n$. \\

An order to notate these different permutations, we use \textit{cycle notation}. A \textbf{cycle} is a string of integers $$(a_{1}a_{2}...a_{m})$$ where each $a_{i},  1 \geq i < m$ is sent to $a_{i + 1}$ and $a_{m}$ is sent to $a_{1}$. For each $\sigma \in S_{n}$, the numbers 1 to $n$ are grouped into $k$ cycles of the previous form. These cycles can be used for some $x \in \{1, 2, ..., n\}$ where $\sigma(x)$ takes $x$ from its original position in a cycle to the number to the right of it, or the first number in its cycle if it is the farthest right number. \\

The \textit{length} of a cycle is the amount of elements in $S$ that are in said cycle. A cycle of length $t$ is called a t-cycle. Two cycles are \textit{disjoint} if they have no elements in common. 1-cycles are generally omitted from cycle notation.

\end{document}