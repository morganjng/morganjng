\documentclass[12pt,letterpaper]{article}
\usepackage[utf8]{inputenc}
\usepackage{amsmath}
\usepackage{amssymb}
\usepackage{amsfonts}
\usepackage{amsthm}
\usepackage[hidelinks]{hyperref}
\usepackage{xcolor}
\usepackage{amssymb}
\usepackage{graphicx}
\newtheorem{theorem}{Theorem}
\newcommand{\ihat}{\hat{\i}}
\newcommand{\jhat}{\hat{\j}}
\newcommand{\khat}{\hat{k}}
\usepackage[left=2cm,right=2cm,top=2cm,bottom=2cm]{geometry}
\author{Jonathan Gribbins}
\title{Discrete Mathematics}
\date{}

\begin{document}

\maketitle

\tableofcontents

\pagebreak

\section{Logic \& Proofs \& Circuits}

\subsection{Propositional Logic \& Truth Tables}

A \textbf{proposition} is a declarative sentence that is \textbf{either} true or false, but not both. To represent \textbf{propositional variables} to denote propositions---conventionally these are $p$, $q$, $r$, $s$,... . A proposition has a \textbf{truth value} of either $T$ or $F$, based on whether it is \textit{true} or \textit{false}. \\

\textbf{Propositional logic} or \textbf{propositional calculus} is the area of logic that deals with these propositions. New propositions, called \textbf{compound propositions}, are statements built out of already established statements through the use of \textit{logical operators}.

\begin{theorem}
	Let $p$ be a proposition. The negation of $p$ is $\lnot p$, and this is the statement ``it is not the case that $p$." The statement $\lnot p$ is read ``not p", and the truth value of $\lnot p$ is the opposite of $p$.
\end{theorem}

The logical operators that are used to connect two or more propositions and create a new proposition are called \textbf{connectives}. 

\begin{theorem}
	Let p and q be propositions. The \textbf{conjunction} of p and q is denoted $p\wedge q$, and is said ``p and q." This proposition is true if and only if both p and q are true, and is false otherwise..
\end{theorem}

\begin{theorem}
	Let p and q be propositions. The \textbf{disjunction} of p and q is denoted $p\vee q$, and is said ``p or q." This proposition is false if both p and q are false, and is true otherwise.
\end{theorem}

The \textit{or} in the disjunction connective is the English \textit{inclusive or}, which means ``either or both." On the contrary, the \textit{exclusive or} is the or that mean ``either but not both." The exclusive or is given by an altogether different connective. 

\begin{theorem}
	Let p and q be propositions. The \textbf{exclusive} of p and q is denoted $p \oplus q$, and is said ``p or q, but not both." This proposition is true only if one of p or q is true. It is false if both are true or if both are false.
\end{theorem}

You can also combine propositions with \textbf{conditional statements.} These are statements that connect two different propositions and are true based on the truth of the statements.

\begin{theorem}
	Let p and q be propositions. The \textbf{conditional statement} $p \rightarrow q$  is the propositions ``if p then q," and is only false when p is true and q is false. In this statement, p is the \textbf{hypothesis} and q is the \textbf{conclusion.} This connective is also called an \textbf{implication.}
\end{theorem}

There are various ways to state this implication. Most of them are simple and are easily recognized as $p \rightarrow q$, but there are a couple notably confusing ones. For instance, ``q only if p" and "q unless $\lnot$p" both mean $p\rightarrow q$. \\

In addition to the conditional statement that $p \rightarrow q$, there are three other commonplace conditional statements. These are:

\begin{description}
	\item[Contrapositive:] The contrapositive of $p \rightarrow q$ is the proposition $\lnot q \rightarrow \lnot p$. The truth value of the contrapositive is always the same as the truth value of the original implication.
	\item[Converse:] The converse of $p \rightarrow q$ is the proposition $q \rightarrow p$. The converse has the same truth values of the inverse.
	\item[Inverse:] The inverse of $p \rightarrow q$ is the proposition $\lnot p \rightarrow \lnot q$. The inverse the same truth values of the converse.
\end{description}

When two different statements have the same truth values, they are called \textbf{equivalent}. So the contrapositive and the implication are equivalent, and the inverse and converse are equivalent. \\

Biconditional statements are statements whose truth is based upon the conditions of two propositions. 

\begin{theorem}
	The \textbf{biconditional statement} $p \iff q$ is the proposition ``p if and only if q." This statement is only true if $p \rightarrow q$ and $q \rightarrow p$ have the same truth value, otherwise it is false. Biconditional statements are also called \textbf{bi-implications}, and $p \iff q$ is sometimes also written as ``p iff q."
\end{theorem}

Truth tables are tables that display different values of propositions and how these values effect a certain propositional statement. For instance, a statement like $p \rightarrow q$ would have a truth table like 

\begin{center}
\begin{tabular}{|c|c|c|}
	\hline
	$p$ & $q$ & $p \rightarrow q$ \\
	\hline
	$T$ & $T$ & $T$ \\
	\hline
	$T$ & $F$ & $F$ \\
	\hline
	$F$ & $T$ & $T$ \\
	\hline
	$F$ & $F$ & $T$ \\ 
	\hline 
\end{tabular}
\end{center}

The resultant statement of these truth tables are generally much longer than $p \rightarrow q$, and in this case, the statement may be broken down into component parts and ``built up" from smaller statements. \\

These different connectives can be combined to create longer logical expressions with different truth values. \\

Truth values can also be represented with \textbf{bits}. Bits are values of 1 or 0, where 1 represents true and 0 represents false. When using bits, connectives are generally represented as AND for $\wedge$, OR for $\vee$, and XOR for $\oplus$. A \textbf{bitstring} is a string of bits (like 000101110), and these connectives can be used on bitstrings as \textbf{bitwise connectives}. These bitwise connectives take in two strings of equal length and use the operation on corresponding bits. For instance, $$101110 \oplus\ (XOR)\ 111000 \rightarrow 010110.$$ 

\subsection{Introduction to Proofs}

A \textbf{proof} is an argument that serves to establish the truth of a mathematical or logical statement. A proof utilizes the hypothesis of the proof in conjunction with assumed axioms and other proved statements to complete this argument. In \textit{formal proofs}, every single step used to argue this truth is explicitly stated, but in \textit{informal proofs,} which are generally used in explanations for people, simple steps may be implicit and in-between steps. \\

There are many different terms used in proofs and when dealing with proofs:

\begin{description}
	\item[Theorem] A theorem is a statement that has been proven to be true. In mathematics, this word is generally saved for really important stuff---``theorems" that are not very important are usually called propositions.
	\item[Proof] A proof is an argument that establishes a proposition to be true.
	\item[Axiom] An axiom is a statement that is assumed to be true and is used in proofs. Also called a postulate.
	\item[Lemma] A lemma is a theorem or proposition that is useful in the proofs of other theorems.
	\item[Corollary] A corollary is a theorem that can be established directly from a theorem that has been proved.
	\item[Conjecture] A conjecture is a statement that has been proposed to be true but hasn't been proven yet.
\end{description}

There are two common types of theorems---universal theorems and existence theorems. Universal theorems (with universal qualifier $\forall$) state that for all elements in a domain some proposition holds true. Existence theorems (with existence qualifier $\exists$) state that some element in a domain exists such that a proposition holds true. \\

A universal theorem is usually in the form $\forall x(P(x) \rightarrow Q(x))$. These theorems are usually proved by assuming some element $c$ in the domain in question, and showing that $P(c) \rightarrow Q(c)$ holds true for this arbitrary case. 

\subsubsection{Direct Proofs}

\textbf{Direct proofs} are proofs that begin by assuming that p is true, and then attempting to show that q holds true because of that. These proofs are generally self-evident if they end up working out.

\subsubsection{Proof by Contrapositive}

\textbf{Indirect proofs} are proofs that do not begin with the assumption that p is true. An important type of indirect proof is the \textbf{proof by contrapositive}. This type of proofs begins by assuming that $\lnot q$ is true, and trying to find that $\lnot p$ is always true as a result---this works because the contrapositive of $p \rightarrow q$ is $\lnot q \rightarrow \lnot p$, and these two statements are \textit{logically equivalent}. \\

It is important to note that proving a proposition $p \rightarrow q$ true is trivial if we can prove $p$ is false because when $p$ is false, $p \rightarrow q$ is always true. his type of proof is called \textbf{vacuous.} If a proposition $p \rightarrow q$ is proven by showing that $q$ is true, this proof is called \textbf{trivial}.

\subsubsection{Proof by Contradiction}

If we want to prove that a statement p is true and we find a contradiction q that is proven false that is implied by $\lnot p \rightarrow q$, then $\lnot p$ must be false---therefore p must be true. This type of proof is called \textbf{proof by contradiction,} and is started by assuming that either q and $\lnot p$ or $\lnot q$ and p are both true (but not both assumptions), and using this assumption to lead to a contradiction.

\subsubsection{Proofs of Equivalence}

A \textbf{proof of equivalence} is a proof of a biconditional statement. This type of proof is based on the logical fact that $$(p \iff q) \iff (p \rightarrow q)\wedge(q \rightarrow p).$$ That is---p if and only if q if and only if p implies q and q implies p. If the statement $p \iff q \iff r$ is true, the truth or falsity of one statement (p, q, or r) guarantees identical truth or falsity of the other statements (this works form more than three propositions too!).

\subsubsection{Counterexamples}

To disprove a universal proposition like $\forall x(P(x))$, it takes only one example $c$ for which $P(c)$ is not true. Additionally, for existence proofs like $\exists x(P(x))$, it takes only one example $c$ for which $P(c)$ holds true to prove the statement.

\subsubsection{Mistakes in Proofs}

It is very easy to make mistakes in proofs. However, if in a proof there is some mistake in the axiomatic or logical foundation, the proof becomes incorrect, and the results are no longer correct. 

\pagebreak

\section{Vocabulary}
	
\begin{description}
	\item[Proposition] A declarative statement that is either true or false, but not both.
	\item[Propositional variables] Letters used to represent a proposition
	\item[Truth value] The truth (T) or falsehood (F) of a proposition
	\item[Propositional calculus/logic] Area of logic that deals with propositions
	\item[Compound propositions] Propositions built out of other propositions with logical operators
	\item[Connectives] Logical operators that connect two or more propositions and create a new proposition
	\item[Conjunction] Connective that combines two propositions and is true if and only if both propositions are true
	\item[Disjunction] Connective that combines two propositions and is false if both propositions are false, and true otherwise
	\item[Exclusive] Connective that combines two propositions and is true only if one of the two propositions is true, but not both
	\item[Conditional statements] Statements that connect two different propositions and whose truth value is based on the truth of the statements
	\item[Implication] A conditional statement stating ``if p then q" that is only false if p is true and q is false
	\item[Contrapositive] A conditional statement rearranging $p \rightarrow q$ and stating ``if not p then not q" that is only false if q is false and p is true
	\item[Converse] A conditional statement rearranging $p \rightarrow q$ and stating ``if q then p" that is only false if q is true and p is false
	\item[Inverse] A conditional statement rearranging $p \rightarrow q$ and stating ``if not p then not q" that is only false if q is true and p is false
	\item[Logically equivalent] Two statements are logically equivalent if all of their truth values are the same
	\item[Biconditional statement] A conditional statement stating ``p if and only if q" that is only true if $p \rightarrow q$ and $q \rightarrow p$ have the same truth value
	\item[Bit] A value of 1 or 0, where 1 denotes truth and 0 denotes falsehood
	\item[Bitstring] A string of bits
	\item[Bitwise connectives] Connectives operating upon corresponding bits in two bitstrings
	\item[Proof] An argument establishing the truth of a mathematical or logical statement
\end{description}

\end{document}