\documentclass[12pt,letterpaper]{article}
\usepackage[utf8]{inputenc}
\usepackage{amsmath}
\usepackage{amsfonts}
\usepackage{amssymb}
\usepackage{graphicx}
\usepackage{lastpage}
\usepackage[left=1.00in, right=1.00in, top=1.00in, bottom=1.00in]{geometry}
\usepackage{fancyhdr}
\title{Math 523 HW8}
\author{Morgan Gribbins}
\date{}
\pagestyle{fancy}
\fancyhf{}
\lhead{Page \thepage\ of \pageref{LastPage}}
\begin{document}
	
\maketitle

Use the method of Examples 1 and 2 to compute these integrals. \\

\textbf{1.} \(\int_{-\infty}^{\infty} \frac{x^{4}}{1+x^{8}}dx\) \\

Let \(f= z^{4}/(1+z^{8})\). This function has isolated singularities where \(z^{8} = -1\), i.e. at \(z = e^{i(\pi/8+k\pi/4)}\), with integer \(k\). The singularities that lie in the domain \(U\) are the points \(e^{i\pi/8},\ e^{3i\pi/8},\ e^{5i\pi/8},\ e^{7i\pi/8}\). By the Residue Theorem, this integral is equal to the sum of these residues multiplied by \(2\pi i\). \\

By theorem, the residue at \(z_{0}\) of a rational function \(F/G\) is equal to \(F(z_{0})/G'(z_{0})\), so by letting \(F=z^{4}\) and \(G=1+z^{8}\), we may calculate our residues.\\
\begin{itemize}
	\item Res\((f; e^{i\pi/8}) = 1/8e^{3i\pi/8}\) \\
	\item Res\((f; e^{3i\pi/8}) = 1/8e^{9i\pi/8}\) \\
	\item Res\((f; e^{5i\pi/8}) = 1/8e^{15i\pi/8}\) \\
	\item Res\((f; e^{7i\pi/8}) = 1/8e^{21i\pi/8}\) \\
\end{itemize}
Summing and multiplying these gives us \[\int_{-\infty}^{\infty} \frac{x^{4}}{1+x^{8}}dx = \frac{i\pi}{4}(e^{-3i\pi/8}+e^{-9i\pi/8}+e^{-15i\pi/8}+e^{-21i\pi/8}).\]

\textbf{2.} \(\int_{-\infty}^{\infty}\frac{x^{2}}{x^{4}-4x^{2}+5}dx\)\\

Let \(f(z) = \frac{z^{2}}{z^{4}-4z^{2}+5}\). This function has isolated singularities at \(z= \pm \sqrt{2+i}\) and \(z=\pm\sqrt{2-i}\). The singularities that lie in \(U\) are \(-\sqrt{2-i}\) and \(\sqrt{2+i}\). \\

By theorem, the residue at \(z_{0}\) of a rational function \(F/G\) is equal to \(F(z_{0})/G'(z_{0})\), so by letting \(F=z^{2}\) and \(G=z^{4}-4z^{2}+5\), we may calculate our residues. This formula shows that Res\((f, \sqrt{2+i}) = \frac{\sqrt{2+i}}{4i}\) and Res\((f, -\sqrt{2-i}) = \frac{\sqrt{2-i}}{4i}\). Summing and multiplying (by 2 \(\pi i\)) these values gives us

\[\int_{-\infty}^{\infty} \frac{x^{2}}{x^{4}-4x^{2}+5} = 2i\pi\left(\frac{\sqrt{2+i}+\sqrt{2-i}}{4i}\right).\]

\textbf{3.} \(\int_{-\infty}^{\infty} \frac{dx}{(x^{2}+a^{2})(x^{2}+b^{2})}, \text{ }a,b > 0\)\\

Let \(f(z) = 1/(z^{2}+a^{2})(z^{2}+b^{2})\). This function has isolated singularities at \(z = \pm bi\) and \(z = \pm ai\). As \(a,b\) are positive, \(ai\) and \(bi\) lie in \(U\). \\

By theorem, the residue at \(z_{0}\) of a rational function \(F/G\) is equal to \(F(z_{0})/G'(z_{0})\), so by letting \(F=1\) and \(G=(z^{2}+a^{2})(z^{2}+b^{2})\), we may calculate our residues. This formula gives us Res\((f, ai) = \frac{1}{2ai}\frac{1}{b^{2}-a^{2}}\) and Res\((f, bi) = \frac{1}{2bi}\frac{1}{a^{2}-b^{2}}\). Summing and multiplying (by 2\(\pi i\)) these values gives us

\[\int_{-\infty}^{\infty} \frac{dx}{(x^{2}+a^{2})(x^{2}+b^{2})} = \frac{\pi}{ab(a+b)}.\]

Use the method of Example 7 to compute these integrals. \\

\textbf{9.} \(\int_{0}^{2\pi} \frac{d\theta}{(2-\sin\theta)^{2}}\)\\

With the substitution \(z = e^{i\theta}\), it follows that \(d\theta = dz/iz\) and \(\sin\theta = \frac{1}{2i}(z-(1/z))\). This gives \[\int_{0}^{2\pi}\frac{d\theta}{(2-\sin\theta)^{2}} = \int_{|z|=1} \frac{dz}{zi(2-\frac{z}{2i}+\frac{1}{2iz})^{2}}\] \[= \frac{1}{i}\int_{|z|=1} \frac{dz}{\frac{9z}{2} - \frac{2z^{2}}{i} + \frac{2}{i} -\frac{z^{3}}{4} - \frac{1}{4z}} = 8i\pi\left\{\frac{1}{2\pi i}\int_{|z| = 1} \frac{zdz}{18iz^{2} - 8z^{3} + 8z - iz^{4} - i }\right\}.\] The integrand (call it \(f\)) of this integral has one pole within \(|z| < 1\), at \(-i(\sqrt{3}-2)\), so we have \[\int_{0}^{2\pi} \frac{d\theta}{(2-\sin\theta)^{2}} = 8i\pi\text{Res}(f, -i(\sqrt{3}-2)) = 8i\pi \frac{-i}{6\sqrt{3}} = \frac{4\pi}{3\sqrt{3}}.\]

\textbf{10.} \(\int_{0}^{2\pi} \frac{d\theta}{(1+\beta\cos\theta)^{2}}, \text{ }-1<\beta<1\) \\

With the substitution \(z = e^{i\theta}\), it follows that \(d\theta = dz/iz\) and \(\cos\theta = \frac{1}{2}(z+(1/z))\). This gives \[\int_{0}^{2\pi} \frac{d\theta}{(1+\beta\cos\theta)^{2}} = \int_{|z|=1} \frac{dz}{iz(1+\frac{\beta z}{2} + \frac{\beta}{2z})^{2}}\] \[=\frac{1}{i}\int_{|z|=1} \frac{dz}{z+\beta z^{2} + \beta + \beta^{2}z/2 + \beta^{2} z^{3}/4 + \beta^{2}/4} = \frac{4}{i} \int_{|z|=1} \frac{zdz}{\beta^{2}z^{4} + 4\beta z^{3} + (2\beta^{2} + 4)z^{2} + 4\beta z + \beta^{2}}\] \[= 8\pi \left\{ \frac{1}{2\pi i} \int_{|z| = 1} \frac{zdz}{(\beta z^{2} + 2z + \beta)^{2}}\right\}.\] The integrand \(f\) of this integral has isolated singularities in \(|z| < 1\) at \(z = \frac{-1}{\beta} \pm \sqrt{1/\beta^{2} - 1}\), which are complex conjugates, so we have \[\int_{0}^{2\pi} \frac{d\theta}{(1+\beta\cos\theta)^{2}} = 8\pi |\text{Res}(f, \frac{-1}{\beta} + \sqrt{1/\beta^{2} - 1})| = \frac{8\pi (1/\beta^{3})}{(1/\beta^{2} - 1)^{3/2}}.  \]

\textbf{12.} \(\int_{0}^{2\pi} \sin^{2k}\theta d\theta\) \\

With the substitution \(z = e^{i\theta}\), it follows that \(d\theta = dz/iz\) and \(\sin\theta = \frac{1}{2i}(z-(1/z))\). This gives \[\int_{0}^{2\pi} \sin^{2k}\theta d\theta = \int_{|z| = 1} \frac{1}{(2i)^{2k}}(z-1/z)^{2k}\]\[= \int_{|z| = 1} \frac{1}{(2i)^{k}} \frac{z^{2k}}{z^{2k}}(z-1/z)^{2k} = \int_{|z| = 1} \frac{1}{(2i)^{2k}} \left(\frac{z^{2} - 1}{z}\right)^{2k}.\] This integrand (\(f\)) has an isolated singularity at \(z = 0\), so we have  \[\int_{0}^{2\pi} \sin^{2k}\theta d\theta = 2\pi i \text{Res}(f, 0) = \frac{(2k)!\pi}{(k!)^{2}2^{2k-1}}.\]





\end{document}