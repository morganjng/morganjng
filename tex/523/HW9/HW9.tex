\documentclass[12pt,letterpaper]{article}
\usepackage[utf8]{inputenc}
\usepackage{amsmath}
\usepackage{amsfonts}
\usepackage{amssymb}
\usepackage{graphicx}
\usepackage{lastpage}
\usepackage[left=1.00in, right=1.00in, top=1.00in, bottom=1.00in]{geometry}
\usepackage{fancyhdr}
\title{Math 523 HW 9}
\author{Morgan Gribbins}
\date{}
\pagestyle{fancy}
\fancyhf{}
\lhead{Page \thepage\ of \pageref{LastPage}}
\begin{document}
	
\maketitle

Use the technique of Example 2 to determine the number of zeroes of \(f\) in the first quadrant. \\

\textbf{1.} \(f(z) = z^{2} - z + 1\) \\

We examine \(f(z)\) on the quarter circle of radius \(R >> 0\) in the first quadrant bounded by the real and imaginary axes. On the segment \(0 \leq x \leq R\), \(f(x) = x^{2} - x + 1\), which is real and positive. On the quarter circle \(z = Re^{it}, 0 \leq t \leq \pi/2\), \[f(Re^{it}) = R^{2}e^{2it}\left(1 - \frac{1}{Re^{it}} + \frac{1}{R^{2}e^{2it}}\right) = R^{2}e^{2it}(1+\gamma),\] where \(|\gamma| \leq 2/R < \epsilon\) for \(R\) large. Thus, \(\arg f(Re^{it})\) is approximately \(\arg  (e^{2it}) = 2t\) for large \(R\), so \(\arg f(Re^{it})\) increases from \(0\) to about \(\pi\) as \(t\) increases from \(0\) to \(\pi/2\). On the segment \(z=iy, R \geq y \geq 0\), \[f(iy) = -y^{2} - iy + 1.\] Thus, as \(y\) decreases from \(R\) to \(0\), \(f(iy)\) moves from the third quadrant to \(z = 1\), and \(\arg f(z)\) increases by \(\pi\), so as \(z\) traverses the contour, \(\arg f(z)\) increases by exactly \(2\pi\), and so \(f(z)\) has exactly one zero in the first quadrant. \\

\textbf{2.} \(f(z) = z^{4} - 3z^{2} + 3\) \\

We examine \(f(z)\) on the quarter circle of radius \(R >> 0\) in the first quadrant bounded by the real and imaginary axes. On the segment \(0 \leq x \leq R\), \(f(x) = x^{4} - 3x^{2} + 3\), which is real and positive. On the quarter circle \(z = Re^{it}, 0 \leq t \leq \pi/2\), \[f(Re^{it}) = R^{4}e^{4it}\left(1 - \frac{3}{R^{2}e^{2it}} + \frac{3}{R^{4}e^{4it}}\right) = R^{4}e^{4it}(1+\gamma),\] where \(|\gamma| \leq 6/R < \epsilon\) for \(R\) large. Thus, \(\arg f(Re^{it})\) is approximately \(\arg  (e^{4it}) = 4t\) for large \(R\), so \(\arg f(Re^{it})\) increases from \(0\) to about \(2\pi\) as \(t\) increases from \(0\) to \(\pi/2\). On the segment \(z=iy, R \geq y \geq 0\), \[f(iy) = y^{4} + 3iy^{2} + 3.\] Thus, as \(y\) decreases from \(R\) to \(0\), \(f(iy)\) moves from the first quadrant to \(z = 3\), and \(\arg f(z)\) increases by \(0\), so as \(z\) traverses the contour, \(\arg f(z)\) increases by exactly \(2\pi\), and so \(f(z)\) has exactly one zero in the first quadrant. \\

\textbf{3.} \(f(z) = z^{3} - 3z + 6\) \\

We examine \(f(z)\) on the quarter circle of radius \(R >> 0\) in the first quadrant bounded by the real and imaginary axes. On the segment \(0 \leq x \leq R\), \(f(x) = x^{3} - 3x + 6\), which is real and positive. On the quarter circle \(z = Re^{it}, 0 \leq t \leq \pi/2\), \[f(Re^{it}) = R^{3}e^{3it}\left(1 - \frac{3}{Re^{it}} + \frac{6}{R^{3}e^{3it}}\right) = R^{3}e^{3it}(1+\gamma),\] where \(|\gamma| \leq 7/R < \epsilon\) for \(R\) large. Thus, \(\arg f(Re^{it})\) is approximately \(\arg  (e^{3it}) = 3t\) for large \(R\), so \(\arg f(Re^{it})\) increases from \(0\) to about \(3\pi/2\) as \(t\) increases from \(0\) to \(\pi/2\). On the segment \(z=iy, R \geq y \geq 0\), \[f(iy) = -iy^{3} - 3iy + 6.\] Thus, as \(y\) decreases from \(R\) to \(0\), \(f(iy)\) moves from the fourth quadrant to \(z = 6\), and \(\arg f(z)\) increases by \(\pi/2\), so as \(z\) traverses the contour, \(\arg f(z)\) increases by exactly \(2\pi\), and so \(f(z)\) has exactly one zero in the first quadrant. \\

\textbf{4.} \(f(z) = z^{2} + iz + 2 + i\) \\

We examine \(f(z)\) on the quarter circle of radius \(R >> 0\) in the first quadrant bounded by the real and imaginary axes. On the segment \(0 \leq x \leq R\), \(f(x) = x^{2} + ix + 2 + i\), which traverses from \(2+i\) to \(R^{2} + 2 + i(R+1)\), so \(\arg f(x)\) changes by \(-\arg (2+i)\). On the quarter circle \(z = Re^{it}, 0 \leq t \leq \pi/2\), \[f(Re^{it}) = R^{2}e^{2it}\left(1 + \frac{i}{Re^{it}} + \frac{2+i}{R^{3}e^{3it}}\right) = R^{2}e^{2it}(1+\gamma),\] where \(|\gamma| \leq 4/R < \epsilon\) for \(R\) large. Thus, \(\arg f(Re^{it})\) is approximately \(\arg  (e^{2it}) = 2t\) for large \(R\), so \(\arg f(Re^{it})\) increases from \(0\) to about \(\pi\) as \(t\) increases from \(0\) to \(\pi/2\). On the segment \(z=iy, R \geq y \geq 0\), \[f(iy) = -y^{2} - y + 2 + i.\] Thus, as \(y\) decreases from \(R\) to \(0\), \(f(iy)\) moves from the third quadrant to \(z = 2+i\), and \(\arg f(z)\) increases by \(-\pi + \arg (2+i)\), so as \(z\) traverses the contour, \(\arg f(z)\) increases by exactly \(2\pi\), and so \(f(z)\) has no zeros in the first quadrant. \\

\textbf{5.} \(f(z) = z^{9} + 5z^{2} + 3\) \\

We examine \(f(z)\) on the quarter circle of radius \(R >> 0\) in the first quadrant bounded by the real and imaginary axes. On the segment \(0 \leq x \leq R\), \(f(x) = x^{9} + 5x^{2} + 3\), which is real and positive. On the quarter circle \(z = Re^{it}, 0 \leq t \leq \pi/2\), \[f(Re^{it}) = R^{9}e^{9it}\left(1 + \frac{5}{Re^{7it}} + \frac{3}{R^{9}e^{9it}}\right) = R^{9}e^{9it}(1+\gamma),\] where \(|\gamma| \leq 8/R < \epsilon\) for \(R\) large. Thus, \(\arg f(Re^{it})\) is approximately \(\arg  (e^{9it}) = 9t\) for large \(R\), so \(\arg f(Re^{it})\) increases from \(0\) to about \(9\pi/2\) as \(t\) increases from \(0\) to \(\pi/2\). On the segment \(z=iy, R \geq y \geq 0\), \[f(iy) = iy^{9} - 5y^{2} + 3.\] Thus, as \(y\) decreases from \(R\) to \(0\), \(f(iy)\) moves from the first quadrant to \(z = 3\), and \(\arg f(z)\) increases by \(-\pi/2\), so as \(z\) traverses the contour, \(\arg f(z)\) increases by exactly \(4\pi\), and so \(f(z)\) has exactly two zero in the first quadrant. \\

\textbf{18.} Extend Formula 4 to prove the following. Let \(g\) be analytic on a domain containing \(\gamma\) and its inside. Then \[\frac{1}{2\pi i} \int_{\gamma} \frac{h'(z)}{h(z)}g(z) dz = \sum_{i = 1}^{N} g(z_{i}) - \sum_{j = 1}^{M} g(w_{j}),\] where \(z_{1}, ..., z_{N}\) are the zeroes of \(h\) and \(w_{1}, ..., w_{M}\) are the poles of \(h\) inside \(\gamma\), each listed according to its multiplicity. \\

By the Residue Theorem, \[\frac{1}{2\pi i} \int_{\gamma} \frac{h'(z)}{h(z)} g(z) dz = \sum_{i = 1}^{N} \text{Res}\left(\frac{h'}{h}g ; z_{i}\right),\] where \(z_{i}\) are the poles of \(\frac{h'}{h}g\). Note that this set of poles is the union of the set of zeros of \(h(z)\) and poles of \(h(z)\) inside \(\gamma\). By theorem, a residue of \(\frac{h'(z)g(z)}{h(z)}\) at \(z_{i}\) is equal to \(\frac{h'(z_{i})g(z_{i})}{h'(z_{i}} = g(z_{i})\). Therefore, \[\frac{1}{2\pi i} \int_{\gamma} \frac{h'(z)}{h(z)}g(z) dz = \sum_{i = 1}^{N} g(z_{i}) - \sum_{j = 1}^{M} g(w_{j}),\] with the prior definition of \(z_{i}\) and \(w_{j}\).


\end{document}