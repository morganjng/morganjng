\documentclass[12pt,letterpaper]{article}
\usepackage[utf8]{inputenc}
\usepackage{amsmath}
\usepackage{amsfonts}
\usepackage{amssymb}
\usepackage{graphicx}
\usepackage{lastpage}
\usepackage[left=1.00in, right=1.00in, top=1.00in, bottom=1.00in]{geometry}
\usepackage{fancyhdr}
\title{Math 523 HW 5}
\author{Morgan Gribbins}
\date{}
\pagestyle{fancy}
\fancyhf{}
\lhead{Page \thepage\ of \pageref{LastPage}}
\begin{document}
	
\maketitle

\section*{Section 2.3} 

In Exercises (1) to (4), evaluate the given integral using Cauchy's Formula or Theorem.

\textbf{(1)} \(\int_{|z| = 1} \frac{z}{(z-2)^{2}} dz\) \\

On the domain of \(\mathbb{C}\setminus\{2\}\), the integrand of \(\int_{|z| = 1} \frac{z}{(z-2)^{2}} dz\) is analytic, and \(2\) is not in the interior of \(|z| = 1\), so \(\int_{|z| = 1} \frac{z}{(z-2)^{2}} dz = 0.\) \\

\textbf{(2)}  \(\int_{|z| = 2} \frac{e^{z}}{z(z-3)} dz\) \\

The integrand of \(\int_{|z| = 2} \frac{e^{z}}{z(z-3)} dz\) is not defined on \(z \in \{0,3\}\), which is part of the interior of \(|z| = 2\), so we cannot apply Cauchy's Theorem. Because \(z_{0} = 0\) is the interior point in this integral, we have \(f(0) = \frac{1}{2\pi i} \int_{|z|=2} \frac{e^{z}/(z-3)}{z-0}\), \(-1/3 = \frac{1}{2\pi i} \int_{|z| = 2} \frac{e^{z}}{z(z-3)} dz\), \(\int_{|z| = 2} \frac{e^{z}}{z(z-3)} dz = -2\pi i /3\). \\

\textbf{(3)} \(\int_{|z+1| = 2} \frac{z^{2}}{4-z^{2}} dz\) \\

The integrand of \(\int_{|z+1| = 2} \frac{z^{2}}{4-z^{2}} dz\) is not analytic inside of \(|z+1| = 2\), so we cannot apply Cauchy's Theorem. We can rearrange \(\int_{|z+1| = 2} \frac{z^{2}}{4-z^{2}} dz\) to \(\int_{|z+1| = 2} \frac{-z^{2}/(2+z)}{z-2} dz\), so it is clear that \(\int_{|z+1| = 2} \frac{z^{2}}{4-z^{2}} dz = 2\pi i f(2) = 2\pi i \times 2^{2}/(2+2) = 2\pi i\). \\

\textbf{(4)} \(\int_{|z| = 1} \frac{\sin z}{z} dz\) \\

By Cauchy's Formula, \(\int_{|z| = 1} \frac{\sin z}{z} dz = \int_{|z| = 1} \frac{\sin z}{z-0} = 2\pi i \sin 0 = 0\). \\

In Exercises (5) to (8), evaluate the definite trigonometric integral making use of the technique of Examples 6 and 7 in this section.

\textbf{(5)} \(\int_{0}^{2\pi} \frac{d\theta}{2+\cos \theta}\) \\

Letting \(z = e^{i\theta}\), we have \[\cos \theta = \frac{1}{2}\left(z + \frac{1}{z}\right),\] \[d\theta = \frac{1}{i}\frac{dz}{z}.\] Through substitution, we have \[\int_{0}^{2\pi} \frac{d\theta}{2+\cos \theta} = \int_{|z| = 1} \frac{1}{2+\frac{1}{2}\left(z+ \frac{1}{z}\right)}\frac{1}{i}\frac{dz}{z} = \int_{|z|=1} \frac{2dz}{4iz+iz^{2}+i}.\] This factors to \[\frac{2}{i}\int_{|z|=1} \frac{dz}{(z-(\sqrt{3}-2))(z+(\sqrt{3}+2))}.\] Setting \(p = \sqrt{3} - 2\) and \(q = -\sqrt{3}-2\) note that \(p\) is within the radius 1 circle, and \(q\) is not. The function \((z-q)^{-1}\) is analytic within this circle, so Cauchy's Formula states \[\frac{1}{2\pi i} \int_{|z| = 1} \frac{dz}{(z-q)(z-p)} = \frac{1}{p-q} = \frac{1}{2\sqrt{3}},\] so our integral gives us \[\int_{0}^{2\pi} \frac{d\theta}{2+\cos \theta} = \frac{2\pi}{\sqrt{3}}.\] \\

\textbf{(6)} \(\int_{0}^{2\pi} \frac{d\theta}{3+\sin\theta+\cos\theta}\) \\

Letting \(z = e^{i\theta}\), we have \[\cos \theta = \frac{1}{2}\left(z + \frac{1}{z}\right),\] \[\sin \theta = \frac{1}{2i}\left(z-\frac{1}{z}\right)\], \[d\theta = \frac{1}{i}\frac{dz}{z}.\] Through substitution, we have \[\int_{0}^{2\pi} \frac{d\theta}{3+\sin\theta+\cos\theta} = \int_{|z|=1} \frac{dz}{iz\left(3+\frac{1}{2}\left(z+1/z\right) + \frac{1}{2i}\left(z-1/z\right)\right)} = \int_{|z|=1} \frac{2dz}{z^{2}+iz^{2}+6iz-1+1}\]\[=\frac{2}{1+i}\int_{|z|=1} \frac{dz}{z^{2}+3z+3iz+i}.\] The roots of this are \[\frac{1}{2}\left(-3-3i \pm \sqrt{7} + i \sqrt{7}\right),\] so by the logic of the previous question, we have \[\int_{0}^{2\pi} \frac{d\theta}{3+\sin\theta+\cos\theta} = \frac{2\pi}{\sqrt{7}}.\] \\

\textbf{(8)} \(\int_{0}^{\pi} \frac{d\theta}{1+\sin^{2} \theta} \) \\

Letting \(z = e^{i\theta}\), we have \[d\theta = \frac{1}{i}\frac{dz}{z},\] \[\sin^{2}\theta = \left(\frac{1}{2i}\left(z-1/z\right)\right)^{2} = -\frac{1}{4}\left(z^{2}-2+1/z^{2}\right).\] Therefore, \[\int_{0}^{\pi} \frac{d\theta}{1+\sin^{2} \theta} = \int_{|z| = 1, \text{Re } z \geq 0} \frac{4dz}{iz^{3} + 2iz - \frac{1}{z}} = \frac{4}{i} \int \frac{zdz}{z^{4}+2z^{2}-1} = \int \frac{zdz}{(}.\] The roots of this are 

In Exercises (9) to (12), evaluate the given integral using the technique of Example 10; indicate which theorem or device you used to obtain your answer.

\textbf{(9)} \(\int_{\gamma} \frac{dz}{z^{2}}\), where \(\gamma\) is any curve in Re \(z > 0\) joining \(1 - i\) to \(1 + i\). \\

The integrand \(f(z) = 1/z^{2}\) is the derivative of \(F(z) = -1/z\). This is valid everywhere except \(z=0\), so our domain is valid. The integral \[\int_{\gamma} f(z)dz = \int_{\gamma} F'(z)dz\]\[=F(1+i)-F(1-i)\]\[=\frac{1}{1+i} - \frac{1}{1-i}\]\[=-i.\] \\

\textbf{(10)} \(\int_{\gamma}\left(z+\frac{1}{z}\right)dz\), where \(\gamma\) is any curve in Im \(z > 0\) joining \(-4+i\) to \(6+2i\). \\

The integrand \(f(z) = z + 1/z\) is the derivative of \(F(z) = \frac{1}{2}z^{2} + \log z\), which is a valid antiderivative outside of \(z=0\), which means our curve is valid. We then have \[\int_{\gamma}f(z)dz = \int_{\gamma} F'(z)dz\] \[= F(6+2i) - F(-4+1)\]\[= 16+12i + \log 6+2i - 15/2 +4i - \log -4+i\] \[=17/2 + 16 i + \log \left(\frac{6+2i}{-4+1}\right).\]\\

\textbf{(11)} \(\int_{\gamma} e^{z} dz\), where \(\gamma\) is the semicircle from \(-1\) to \(1\) passing through \(i\). \\

The integrand \(f(z) = e^{z}\) is the derivative of the function \(F(z) = e^{z}\), which is valid everywhere, so we have \[\int_{\gamma} f(z)dz = \int_{\gamma} F'(z)dz\] \[= F(1) - F(-1)\] \[=e - 1/e.\] \\

\textbf{(12)} \(\int_{\gamma} \sin z dz\), where \(\gamma\) is any curve joining \(i\) to \(\pi\). \\

The integrand \(f(z) = \sin z\) is the derivative of the function \(F(z) = -\cos z\), which is analytic everywhere. We then have \[\int_{\gamma} f(z)dz = \int_{\gamma} F'(z)dz\] \[= F(\pi) - F(i)\] \[=-1-\cos i.\]



\end{document}