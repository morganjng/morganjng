\documentclass[12pt,letterpaper]{article}
\usepackage[utf8]{inputenc}
\usepackage{amsmath}
\usepackage{amsfonts}
\usepackage{amssymb}
\usepackage{graphicx}
\usepackage{lastpage}
\usepackage[left=1.00in, right=1.00in, top=1.00in, bottom=1.00in]{geometry}
\usepackage{fancyhdr}
\pagestyle{fancy}
\fancyhf{}
\title{Complex Functions}
\author{Morgan Gribbins}
\date{}
\lhead{Page \thepage\ of \pageref{LastPage}}
\begin{document}
	
\maketitle
\tableofcontents
\pagebreak

\section{The Complex Plane}

\subsection{The Complex Numbers and the Complex Plane}

The complex plane is how we visualize the separate real and imaginary components of the complex variable \( z \). A \textbf{complex number} is an expression of the form \[z = x + iy,\] where \( x \) and \( y \) are real numbers and \( i \) satisfies the rule \[(i)^{2} = (i)(i) = -1.\] The number \(x\) is called the \textbf{real part} of \(z\) and is written \[x = \text{Re } z.\] The number \(y\), despite the fact that it is also a real number, is called the \textbf{imaginary part} of \(z\) and is written \[z = \text{Im } z.\] The \textbf{modulus}, or \textbf{absolute value} of \(z\) is defined by \[|z| = \sqrt{x^{2} + y^{2}},\ z = x+ iy.\] A complex number \(z = x + iy\) corresponds to the point \(P(x,y)\) in the \(xy\)-plane. The modulus of \(z\), then, is just the distance from the point \(P(x,y)\) to the origin, which is \(0\). These three inequalities hold true: \[|x| \leq |z|,\ |y| \leq |z|,\ |z| \leq |x| + |y|.\] The \textbf{complex conjugate} of \(z = x + iy\) is given by \[\bar{z} = x - iy.\] Addition, subtraction, multiplication, and division of complex numbers follow the ordinary rules of arithmetic. For \[z = x + iy\ \text{and}\ w = s + it\] we have \[z + w = (x + s) + i(y + t),\] \[z - w = (x - s) + i(y - t),\] \[zw = (xs - yt) + i(xt + ys),\ \text{and}\] \[\frac{z}{w} = \frac{\bar{w}z}{\bar{w}w} = \frac{(xs + yt) + i(ys - xt)}{s^{2} + t^{2}}.\] We can also represent complex numbers in the complex plane via polar coordinates---\[x = r \text{cos }\theta\ \text{and}\ y = r \text{ sin }\theta.\] Since \(r = |z|\) by definition of the modulus of a complex number, so \[z = |z|(\text{cos }\theta + i \text{ sin }\theta).\] This is called the \textbf{polar representation} of \(z\). For two complex numbers, \[z = |z|(\text{cos }\theta + i\text{ sin }\theta)\ \text{and}\ w = |w|(\text{cos }\phi + i\text{ sin }\phi),\] we have \[zw = |z||w|(\text{cos }(\theta + \phi) + i\text{ sin }(\theta + \phi))\ \text{and}\] \[\frac{z}{w} = \left( \frac{|z|}{|w|}\right) (\text{cos }(\theta - \phi) + i\text{ sin }(\theta - \phi)).\] From this, we have \textbf{De Moivre's Theorem:} \[(\text{cos }\theta + i\text{ sin }\theta)^{n} = \text{cos }n\theta + i\text{ sin }n\theta.\] We define an \textbf{argument} of the nonzero complex number \(z\) to be any angle \(\theta\) for which \[z = |z|(\text{cos }\theta + i\text{ sin }\theta),\] whether or not it lies in the range \([0, 2\pi)\); we write \(\theta = \text{arg } z\). We define \(\textbf{Arg } z\) to be the number \(\theta_{0} \in [-\pi, \pi)\) such that \[z = |z|(\text{cos }\theta_{0} + i\text{ sin }\theta_{0}).\] We then have \[\textbf{Arg } (zw) = \textbf{Arg } z + \textbf{Arg } w\ (\text{mod } 2\pi).\] In summary, the \(xy\)-plane is a natural interpretation of a complex variable, and the rules of the \(xy\)-plane can be made to fit this depiction. We then call this plane the \textbf{complex plane}, the \(x\)-axis the \textbf{real axis}, and the \(y\)-axis the \textbf{imaginary axis}.

\subsection{Complex Roots and Circles in the Complex Plane}


Complex roots follow directly from rules established in the previous section. A complex number \(z\) that satisfies the equation \(z^{n} = w\) (with \(z = (|z|, \theta)\) and \(w = (|w|, \phi)\)) is called the \textbf{\textit{n}th root of \textit{w}}. This \(n\)th root of \(w\) follows these equations: \[|z|^{n} = |w|,\ \text{cos } n\theta = \text{cos } \phi,\ \text{sin } n\theta = \text{sin } \phi.\] A circle around a point \(p\) of radius \(r\) is given by the equation \[|z - p| = r,\] as this is the set of points \(r\) distance from \(p\).

\subsection{Subsets of the Plane}

The set consisting of all points \(z\) satisfying \(|z - z_{0}| < R\) is called the \textbf{open disc} of radius \(R\) centered at \(z_{0}\). A point \(w_{0}\) in a set \(D\) in the complex plane is called an \textbf{interior point} of \(D\) if there is some open disc centered at \(w_{0}\) that lies entirely within \(D\). A set \(D\) is called \textbf{open} if all of its points are interior points. A point \(p\) is a \textbf{boundary point} of a set \(S\) if every open disc centered at \(p\) containing both points of \(S\) and not of (S). The set of all boundary points of a set \(S\) is called the \textbf{boundary} of \(S\). A set \(C\) is \textbf{closed} if it contains its boundary. \\

\textbf{Theorem: } A set \(D\) is open if and only if it contains no point of its boundary. A set \(C\) is closed if and only if its complement \(D = \{z: z \notin C\}\) is open. \\

A \textbf{polygonal curve} is the union of a finite number of directed line segments \(\textbf{P}_{1}, \textbf{P}_{2}, ..., \textbf{P}_{n}\), where the terminal point of one is the initial point of the next (excluding the last). An open set \(D\) is \textbf{connected} if each pair \(p, q\) of points in \(D\) may be joined by a polygonal curve lying entirely in \(D\). An open connected set is called a \textbf{domain}. A set \(S\) is \textbf{convex} if the line segment \textbf{pq} joining each \(p,q\) in \(S\) also lies in \(S\). An \textbf{open half-plane} is the set of points strictly to one side of a straight line. A \textbf{closed half-plane} is an open half-plane, with the inclusion of the defining straight line.

\subsection{Functions ad Limits}

A \textbf{function} of the complex variable \(z\) is a rule that assigns a complex number to each \(z\) within some specified set \(D\); \(D\) is called the \textbf{domain of definition} of the function. The collection of all possible values of the function is called the \textbf{range} of the function. We frequently write \(w = f(z)\) to distinguish the dependent and independent variables. \\

Let \(\{z_{n}\}_{n=1}^{\infty}\) be a sequence of complex numbers. We say that \(\{z_{n}\}\) has the complex number \(A\) as a \textbf{limit}, or that \(\{z_{n}\}\) \textbf{converges} to \(A\), and we write \[\lim_{n \to \infty} z_{n} = A,\ \text{or}\ z_{n} \to A\] if, given any positive number \(\epsilon\), there is an integer \(N\) such that \[|z_{n} - A| < \epsilon,\ \text{for all } n \geq N.\] A sequence that does not converge, for any reason whatsoever, is called \textbf{divergent}. Additionally, for \(z_{n} = x_{n} + iy_{n}\) and \(A = s + it\), then \(z_{n} \to A\) if and only if \(x_{n} \to s\) and \(y_{n} \to t\). Additionally, if \(z_{n} \to A\), then \(|z_{n}| \to |A|\), and for two convergent series \(z_{n} \to A\) and \(w_{n} \to B \neq 0\), \(\forall \alpha,\beta \in \mathbb{C},\ \alpha z_{n} + \beta w_{n} \to \alpha A + \beta B,\ \alpha z_{n} w_{n} \to \alpha AB,\ \alpha z_{n}/\beta w_{n} \to \alpha A / \beta B\). \\

We say that a function \(f\) defined on a subset \(S \subseteq \mathbb{C}\) has a \textbf{limit} \(L\) at the point \(z_{0} \in S\) or in the boundary of \(S\), and we write \[\lim_{z \to z_{0}} f(z) = L\ \text{or}\ f(z) \to L\ \text{as}\ z \to z_{0}\] if, given \(\epsilon > 0\), there is a \(\delta > 0\) such that \[|f(z) - L| < \epsilon\ \text{whenever}\ z \in S\ \text{and}\ |z - z_{0}| < \delta.\] We say that a function \(f\) has a \textbf{limit L at} \(\infty\), and we write \[\lim_{z \to \infty} f(z) = L\] if, given \(\epsilon > 0\), there is a large number \(M\) such that \[|f(z) - L| < \epsilon\ \text{whenever}\ z \geq M.\] Note that this only requires that \(|z|\) be large; there is no restriction on arg \(z\). The arithmetic rules with constants and limits of sequences hold with limits of functions. \\

Suppose again that \(f\) is a function defined on a subset \(S\) on the complex plane. If \(z_{0} \in S\), then \(f\) is \textbf{continuous} at \(z_{0}\) if \[\lim_{z \to z_{0}} f(z) = f(z_{0}).\] That is, \(f\) is continuous at \(z_{0}\) if the values of \(f(z)\) get arbitrarily close to the value \(f(z_{0})\), so long as \(z\) is in \(S\) and \(z\) is sufficiently close to \(z_{0}\). If it happens that \(f\) is continuous at all points on \(S\), we say \(f\) is \textbf{continuous on \textit{S}}. The function \(f\) is continuous at \(\infty\) if \(f(\infty)\) is defined and \(\lim_{z \to \infty} f(z) = f(\infty)\). \\

The sum of an \textbf{infinite series of complex numbers} is 
practically the same as the sum of an infinite series of real numbers. We define the \textbf{\textit{n}th partial sum by} \[s_{n} = \sum\limits_{j = 1}^{n} = z_{1} + ... + z_{n},\ n = 1,2,... .\]  If the sequence \(\{s_{n}\}\) has a limit \(s\), then we say that the infinite series \(\sum\limits_{j = 1}^{\infty} z_{j}\) \textbf{converges} and has sum s; this is written \[\sum\limits_{j = 1}^{\infty} z_{j} = s.\] If this does not have a limit, we say that this series \textbf{diverges}. As with sequences, the real and imaginary parts of the sum must converge to the real and imaginary parts of the limits of the sum.

\subsection{The Exponential, Logarithm, and Trigonometric Function}

The exponential function is one of the most important functions in complex analysis. Its definition is this: \[e^{x + iy} = e^{x}(\text{cos }y + i\text{ sin }y).\] The form \(\text{exp}(z)\) is sometimes used, especially if \(z\) is particularly complicated. For any two complex numbers, \(z\) and \(w\), we have \[e^{z + w} = e^{z}e^{w}.\] Additionally, \[|e^{z}| = e^{\text{Re } z},\ \text{and}\ \textbf{arg } e^{z} = \text{Im } z.\] \\

The inverse of the exponential function is the logarithm function. For a nonzero complex number \(z\), we define \textbf{log} \(z\) to be any complex number \(w\) with \(e^{w} = z\). This is given by \[\text{log } z = \text{ln } |z| + i \text{ arg } z.\] This provides a set of answers, due to the argument of a complex number being a set of numbers varying by \(2\pi\). To avoid the ambiguity of mod \(2\pi\), we define \textbf{Log \(z\)} as \[\text{Log } z = \text{ln } |z| + i \textbf{ Arg } z.\] The definition of log \(z\) allows us to complete the discussion of roots for complex powers. For a nonzero complex number \(a\), we define \(a^{z}\) by the rule \[a^{z} = e^{z\text{ log }a}.\] \\

The trigonometric functions of \(z\) are defined in terms of the exponential function. We begin with the cosine and sine of \(z\): \[\text{cos } z = \frac{1}{2}(e^{iz} + e^{-iz})\] \[\text{sin } z = \frac{1}{2i}(e^{iz}-e^{-iz}).\] The other four trigonometric functions are defined in terms of sin \(z\) and cos \(z\): \[\text{tan } z= \frac{\text{sin } z}{\text{cos } z},\ \text{cot }z = \frac{\text{cos }z}{\text{sin }z},\ \text{sec }z = \frac{1}{\text{cos }z},\ \text{csc }z = \frac{1}{\text{sin }z}.\] The rules of these trigonometric functions over the reals hold for the complex definitions. \\

The inverse trigonometric functions are defined in this way: \\ 

Let \(w = \text{sin } z = \frac{1}{2i}(e^{iz} - e^{-iz}),\) so \(e^{2iz} - 2iwe^{iz} - 1 = 0,\) which is a quadratic of the variable \(e^{iz}\). By the quadratic equation, we have \(e^{iz} = iw + \sqrt{1-w^{2}},\) or \(z = -i\text{log}(iw + \sqrt{1-w^{2}}),\)---uniquely, we can have \[\text{Arcsin } z = -i\text{Log}(iz + \sqrt{1-z^{2}})\] and \[\text{Arccos } z = -i\text{Log}(z + \sqrt{z^{2} - 1})\] and \[\text{Arctan }z = \frac{i}{2}\text{Log}(\frac{1-iz}{1+iz}),\] with proper logarithms and roots.  

\subsection{Line Integrals and Green's Theorem}

A \textbf{curve} \(\gamma\) is a continuous complex-valued function \(\gamma(t)\) defined for \(t\) is some interval \([a,b]\) in the real axis. The curve \(\gamma\) is \textbf{simple} if \(\gamma(t_{1}) \neq \gamma(t_{2})\) whenever \(a \leq t_{1} < t_{2} < b\), and it is \textbf{closed} if \(\gamma(a) = \gamma(b)\). The \textbf{Jordan Curve Theorem} asserts that the complement of the range of a curve that is both simple and closed consists of two disjoint open connected sets, one bounded and the other unbounded. The bounded piece is the \textbf{inside} of the curve and the outside is the \textbf{outside}. Suppose \(\gamma\) is a curve; separating the complex number \(\gamma(t)\) into its real and imaginary parts and write \(\gamma(t) = x(t) + iy(t), t\in [a,b]\) These two functions may or may not be differentiable. If they are differentiable at some \(t_{0}\), we say \(\gamma(t)\) differentiable at \(t_{0}\), and we can say \(\gamma'(t_{0}) = x'(t_{0}) + iy'(t_{0})\).  A curve \(\gamma\) is \textbf{smooth} if \(\gamma'(t)\) exists and is continuous on \([a,b]\), with the derivatives at \(a\) and \(b\) being taken from the right and left, respectively. A curve is \textbf{piecewise smooth} if it is composed of a finite number of smooth curves, the end of one coinciding with the beginning of the next. Each curve \(\gamma\) is \textbf{oriented} by increasing \(t\). The \textbf{reverse orientation} of \(\gamma\) begins at the end of \(\gamma\) and ends at the beginning of \(\gamma\)---this is denoted by \(-\gamma\) and is parametrized by \(-\gamma(t) = \gamma(a + b - t),\ a \leq t \leq b\). \\

Suppose \(g(t) = x(t) + iy(t)\) is a continuous complex function on \([a,b]\). We define the \textbf{integral of g over [a,b]} by \[\int_{a}^{b} g(t)dt = \int_{a}^{b} x(t)dt + i\int_{a}^{b} y(t)dt.\] Suppose that \(\gamma\) is a smooth curve and \(u\) is a continuous function on the range of \(\gamma\). We define the \textbf{line integral of \(u\) along \(\gamma\)} by \[\int_{\gamma} u(z)dz = \int_{a}^{b} u(\gamma(t))\gamma'(t)dt.\] Line integrals have the same properties that definite integrals on the real numbers have. We define the line integral on a piecewise smooth curve \(\gamma = \gamma_{1}+\gamma_{2}+...+\gamma_{n}\) of the same function to be \[\int_{\gamma} u(z)dz = \sum_{j=i}^{n} \int_{\gamma_{j}} u(z)dz,\] with the integral inside the sum being computed in the same way as previously demonstrated. The length of a curve \(\gamma\) is given by \[\text{length }(\gamma) = \int_{a}^{b} |\gamma'(t)|dt.\] From this, we have the inequality \[\left|\int_{\gamma} u(z)dz\right| \leq \left(\max_{z \in \gamma}|u(z)|\right) \text{length }(\gamma).\] \\

Let \(\Omega\) be a domain bounded by a boundary \(\Gamma = \gamma_{1}+...+\gamma_{n}\). The line integral over the boundary is given by \[\int_{\Gamma} f(z)dz = \sum_{j=1}^{n} \int_{\gamma_{j}} f(z)dz.\] Green's Theorem relates this line integral to the integral of a related function over \(\Omega\). Ve assume that there is an open set containing both \(\Omega\) and \(\Gamma\) where \(f\) has continuous partial derivatives with respect to \(x\) and \(y\). That is, if \(f = u(x,y) + iv(x,y)\) then \[\frac{\partial f}{\partial x} = \frac{\partial u}{\partial x} + i\frac{\partial v}{\partial x},\ \frac{\partial f}{\partial y} = \frac{\partial u}{\partial y} + i\frac{\partial v}{\partial y},\] where all of these are continuous on the open set. Green's Theorem states \[\int_{\Gamma} f(z)dz = i\int \int_{\Omega} \left( \frac{\partial f}{\partial x} + \frac{\partial f}{\partial y}\right)dxdy.\] \\

\pagebreak

\section{Basic Properties of Analytic Functions}

\subsection{Analytic and Harmonic Functions; the Cauchy-Riemann Equations}

A function \(f\) defined for \(z\) in a domain \(D\) is \textbf{differentiable} at a point \(z_{0}\) in \(D\) if \[\lim_{z \to z_{0}} \frac{f(z)-f(z_{0}}{z-z_{0}} = \lim_{h \to 0} \frac{f(z_{0}+h) - f(z_{0}}{h}\] exists; the limit, if it exists, is denoted by \(f'(z_{0})\). If \(f\) is differentiable at each point of the domain \(D\), then \(f\) is called \textbf{analytic} in \(D\). A function analytic on the whole complex plane is called \textbf{entire}. The properties of differentiation established in real calculus hold here. \\

The \textbf{Cauchy-Riemann equations} state that if \(f = u + iv\) is analytic on a domain \(D\), then throughout \(D\), \[\frac{\partial u}{\partial x} = \frac{\partial v}{\partial y}\ \text{ and }\ \frac{\partial u}{\partial y} = -\frac{\partial v}{\partial x}.\] \textbf{Laplace's equation} is \[\Delta u = \frac{\partial^{2} u}{\partial x^{2}}\ + \frac{\partial^{2} u}{\partial y^{2}} = 0.\] If a function satisfies this equation on \(D\), then it is called \textbf{harmonic} on \(D\)---the real and imaginary parts of an analytic function are always harmonic. A function \(x\), which for a harmonic function \(y\) satisfies \(x-y = C\) is called a \textbf{harmonic conjugate} of \(y\). Additionally, if a function is analytic on a disc centered around some point, then said function is differentiable at that point.

\subsection{Power Series}

A \textbf{power series} in \(z\) is an infinite series of the form \[\sum_{n=0}^{\infty} a_{n}(z-z_{0})^{n},\] where \(a_{0}, a_{1},...\) are complex numbers, called the \textbf{coefficients} of the series; \(z_{0}\) is fixed and is called the \textbf{center} of the series. If for some \(z_{1} \neq z_{1}\) has \[\sum_{n=0}^{\infty} a_{n}(z_{1}-z_{0})^{n}\] converges,  then all \(z\) that satisfies \(|z-z_{0}| < |z_{1}-z_{0}|\) have \[\sum_{n=0}^{\infty} a_{n}(z-z_{0})^{n}\] absolutely convergent. For some power series \(\sum a_{n}(z-z_{0})^{n}\), there is a unique number \(R\) such that \[|z-z_{0}| < R \implies \sum_{n=0}^{\infty} a_{n}(z-z_{0})^{n} \text{ converges }\] and \[|z-z_{0}| > R \implies \sum_{n=0}^{\infty} a_{n}(z-z_{0})^{n} \text{ diverges.}\] The number \(R\) is called the \textbf{radius of convergence} for said power series. The derivative of a power series \(f(z) = \sum_{n=0}^{\infty} a_{n}(z-z_{0})^{n}\) is given by \[f'(z) = \sum_{n=1}^{\infty} na_{n}(z-z_{0})^{n-1},\] and this is defined for  (and analytic) inside of the disc of convergence. 

\subsection{Cauchy's Theorem and Cauchy's Formula}

\textbf{Cauchy's theorem:} Suppose that \(f\) is analytic on a domain \(D\). Let \(\gamma\) be a piecewise smooth simple closed curve in \(D\) whose interior \(\Omega\) also lies in \(D\). Then \[\int_{\gamma} f(z)dz = 0.\] \\
	
A curve is \textbf{simply-connected} if, whenever \(\gamma\) is a simple closed curve in \(D\), the inside of \(\gamma\) is also a subset of \(D\). This, in clear terms, means there are no ``holes" in \(D\). \\

A number of results of this theorem follow. \\

Let \(D\) be a simply-connected domain and \(\Gamma\) a closed curve in \(D\) that is composed of a finite number of horizontal and vertical line segments. If \(f\) is analytic in \(D\), then \[\int_{\Gamma} f(z)dz = 0.\] \\

If \(f\) is analytic in a simply-connected domain \(D\), then there is an analytic function \(F\) on \(D\) with \(F' = f\) throughout \(D\). \\

Let \(f\) be analytic on a simply-connected domain \(D\), and let \(\gamma\) be a piecewise smooth closed curve in \(D\). Then \[\int_{\gamma} f(z)dz = 0.\] \\

\textbf{Cauchy's formula:} Suppose that \(f\) is analytic on a domain \(D\) and that \(\gamma\) is a piecewise smooth, positively oriented simple closed curve in \(D\) whose inside \(\Omega\) also lies in \(D\). Then \[f(z) = \frac{1}{2\pi i}\int_{\gamma} \frac{f(\zeta)}{\zeta - z}d\zeta, \text{ for all } z \in \Omega.\]


\end{document}