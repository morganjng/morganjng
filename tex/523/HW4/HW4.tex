\documentclass[12pt,letterpaper]{article}
\usepackage[utf8]{inputenc}
\usepackage{amsmath}
\usepackage{amsfonts}
\usepackage{amssymb}
\usepackage{graphicx}
\usepackage[left=1.00in, right=1.00in, top=1.00in, bottom=1.00in]{geometry}
\usepackage{fancyhdr}
\usepackage{lastpage}
\pagestyle{fancy}
\title{Math 523 HW 4}
\author{Morgan Gribbins}
\date{}
\fancyhf{}
\lhead{Page \thepage\ of \pageref{LastPage}}
\begin{document}
	
\maketitle

\section*{Section 2.1}

Use the rules for differentiation to find the derivatives of \textbf{(2)} and \textbf{(4)}.

\textbf{(2)} \(z^{2} + 10z\) \\

\[(z^{2} + 10z)' = \lim_{h \to 0} \frac{(z+h)^{2} + 10(z+h) - z^{2} - 10z}{h} = \lim_{h \to 0} \frac{z^{2} +2zh + h^{2} + 10z + 10 h - z^{2} - 10 z}{h} \] \[= \lim_{h \to 0} \frac{2zh + h^{2} + 10h}{h} = \lim_{h \to 0} 2z + h + 10 = 2z + 10.\] \\

\textbf{(4)} \([\text{cos } (z^{2})]^{3}\) \\

Note that \[[\text{cos }(z^{2})]^{3} = \frac{1}{8} \left(e^{iz^{2}} + e^{-iz^{2}}\right)^{3},\] so \[\left\{[\text{cos }(z^{2})]^{3}\right\}' = \left\{\frac{1}{8} \left(e^{iz^{2}} + e^{-iz^{2}}\right)^{3}\right\}'\] \[\implies \left\{[\text{cos }(z^{2})]^{3}\right\}' = \frac{3}{8}\left(e^{iz^{2}} + e^{-iz^{2}}\right)^{2}\times \left(2zie^{iz^{2}} - 2zie^{-iz^{2}}\right)\] \[ \implies \left\{[\text{cos }(z^{2})]^{3}\right\}' = \frac{3zi}{4} \times 4[\text{cos }(z^{2})]^{2} \times 2i\text{sin }(z^{2}) \] \[ \implies \left\{[\text{cos }(z^{2})]^{3}\right\}' = -6z\text{cos}^{2}\text{ } (z^{2}) \text{ sin }(z^{2}). \] \\

For each function \(f\) listed in Exercises \textbf{(8)} and \textbf{(10)}, find an analytic function \(F\) with \(F' = f\).

\textbf{(8)} \(f(z) = z - 2\) \\

\(F(z) = \frac{1}{2}z^{2} -2 z\) is analytic with \(F' = f\). \\

\textbf{(10)} \(f(z) = \text{sin } z \text{ cos } z \) \\

\(F(z) = \frac{1}{2}\left(\text{sin } z\right)^{2}\) is analytic with \(F' = f\). \\

\textbf{(14)} Let \(P(z) = A(z - z_{1})...(z-z_{n})\), where \(A\) and \(z_{1},...z_{n}\) are complex numbers and \(A \neq 0\). Show that \[\frac{P'(z)}{P(z)} = \sum_{j=1}^{n}\frac{1}{z-z_{j}},\ z \neq z_{1},...,z_{n}.\] \\



\textbf{(16)} Find the derivative of the \textbf{linear fractional transformation} \(T(z) = (az+b)/(cz+d), ad \neq bc.\) In what way does the condition \(ad - bc \neq 0\) enter? Conclude that \(T'(z)\) is never zero, \(z \neq -d/c\). \\

\[T'(z) = \left( \frac{az+b}{cz+d} \right)' = \lim_{h \to 0} \frac{\frac{a(z+h) + b}{c(z+h) + d} - \frac{az+b}{cz+d}}{h} = \lim_{h \to 0} \frac{\frac{az+ah + b}{cz+ch + d} - \frac{az+b}{cz+d}}{h}\] \[= \lim_{h \to 0} \frac{\frac{(az+ah + b)(cz+d)-(az+b)(cz+ch+d)}{(cz+ch + d)(cz+d)}}{h} = \lim_{h \to 0} \frac{\frac{(acz^{2}+aczh + bcz + adz + adh + bd )-(acz^{2} + aczh + adz + bcz + bch + bd )}{(cz+ch + d)(cz+d)}}{h}\] \[= \lim_{h \to 0} \frac{\frac{adh-bch}{(cz+ch + d)(cz+d)}}{h} = \lim_{h \to 0} \frac{ad - bc}{(cz + ch + d)(cz+d)} = \frac{ad-bc}{(cz+d)^{2}}.\] The condition \(ad-bc \neq 0\) enters when taking the limit as \(h \to 0\)---if \(ad-bc = 0\), then there would be additional \(h\) in the denominator of the limit, and the limit would not converge. Therefore, \(T'(z)\) can never be zero, and \(z \neq -d/c\), as that would cause \(T'(z)\) to not converge to a limit. \\

\textbf{(18)} Show that \(h(z) = \bar{z}\) is not analytic on any domain. (\textbf{Hint:} check the Cauchy-Riemann equations.) \\

If \(h(z) = \bar{z}\), where \(h(z) = u(x,y) + iv(x,y)\) were analytic, we would have \(u_{x} = v_{y}\) and \(u_{y} = -v_{x}\). We also have \(h(z) = x - iy\), so \(u_{x} = 1 \neq v_{y}\), as \(v_{y} = -1\). Therefore, \(h(z) = \bar{z}\) cannot be analytic. \\

\textbf{(20)} Let \(f = u + iv\) be analytic. In each of the following, find \(v\) given \(u\). \\

\textbf{(20a)} \(u = x^{2} - y^{2}\) \\

\(u_{x} = 2x,\ u_{y} = -2y,\) so \(v_{y} = 2x,\ v_{x} = 2y\). Therefore, \(v = 2xy\). \\

\textbf{(20b)} \(u = \frac{x}{x^{2} + y^{2}}\) \\
	

	
\section*{Section 2.2} 

In exercises \textbf{(2)} and \textbf{(4)}, use Theorem 2 or Example 4 to find the radius of convergence of the following power series.

\textbf{(2)} \(\sum_{k=0}^{\infty}\frac{(k!)^{2}}{(2k)!}(z-2)^{k}\) \\

\[\frac{1}{R} = \lim_{k \to \infty} \left| \frac{a_{k+1}}{a_{k}} \right| = \lim_{k \to \infty} \left| \frac{((k+1)!)^{2}/(2k+2)!}{(k!)^{2}/((2k)!)} \right|\] \[ = \lim_{k \to \infty} \left| \frac{(k+1)^{2}}{(2k+2)(2k+1)} \right| = \lim_{k \to \infty} \left| \frac{k^{2}+2k+1}{4k^{2}+6k+2} \right| = \frac{1}{4} \implies R = 4.\]

\textbf{(4)} \(\sum_{k=0}^{\infty}(-1)^{k}z^{2k}\) \\

Letting \(w = z^{2}\), we have \(\sum_{k=0}^{\infty} (-1)^{k}z^{2k} = \sum_{k=0}^{\infty}(-1)^{k}w^{k}\), so we can say \[\frac{1}{R} = \lim_{n\to \infty} \left| \frac{(-1)^{k+1}}{(-1)^{k}} \right| = 1 \implies R = 1.\]

In exercises \textbf{(8)} and \textbf{(10)}, find the power series about the origin for the given function.

\textbf{(8)} \(z^{2}\text{cos } z\) \\

\[z^{2}\text{sin } z = z^{2} \sum_{n=0}^{\infty} (-1)^{n}z^{2n+1}/(2n+1)! = \sum_{n=0}^{\infty} (-1)^{n}z^{2n+3}/(2n+1)!.\] \\

\textbf{(10)} \(\frac{1+z}{1-z},\ |z|<1\) \\



In exercises \textbf{(14)}, \textbf{(16)}, and \textbf{(18)}, find a ``closed form" (that is, a simple expression) for each of the given power series. 

\textbf{(14)} \(\sum_{n=0}^{\infty} \frac{z^{2n}}{n!}\) \\



\textbf{(16)} \(\sum_{n=1}^{\infty} n(z-1)^{n-1} \) \\



\textbf{(18)} \(\sum_{n = 2}^{\infty} n(n-1)z^{n} \) \\



\textbf{(22)} \\

\textbf{(22a)} If \(f(z) = \sum_{n=0}^{\infty} a_{n}(z-z_{0})^{n}\) has radius of convergence \(R > 0 \) and if \(f(z) = 0\) for all \(z, |z-z_{0}| < r \leq R \), show that \(a_{0} = a_{1} = ... = 0\). \\



\textbf{(22b)} If \(F(z) = \sum_{n=0}^{\infty} a_{n}(z - z_{0})^{n}\) and \(G(z) = \sum_{n=0}^{\infty} b_{n}(z-z_{0})^{n}\) are equal on some disc \(|z - z_{0}|< r\), show that \(a_{n} = b_{n}\) for all \(n\). \\





\textbf{}

\end{document}